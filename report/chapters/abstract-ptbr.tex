% brazilian portuguese abstract
\setlength{\absparsep}{18pt} % ajusta o espaçamento dos parágrafos do resumo
\begin{resumo}

Ao observar e conversar com donos de estacionamentos privados, assim como com motoristas, percebeu-se dois grandes problemas enfrentados por estes grupos. Os donos e gerentes de estacionamentos privados reportaram a necessidade de aumentar a eficiência e diminuir custos em seus negócios. Os motoristas apontam o tempo gasto com a busca de vagas de garagem e a falta de possibilidade de reservar um vaga como os principais motivos para a insatisfação com o uso de serviços oferecidos por estacionamentos privados. 

Observando-se estes problemas, percebeu-se que eles complementam um ao outro, e para resolver o problema de um é preciso resolver o problema do outro. Este fato levou à percepção de que a criação de um espécie de leilão de vagas para estacionamentos pode ser uma boa oportunidade de negócio. O plano de projeto para implementação desta idéia é o proposto neste trabalho. 

A solução encontrada aborda a implementação de dois aplicativos. Um para ser utilizado pelo motorista, onde é possível realizar reservas e encontrar estacionamentos vagos. O outro aplicativo foca no dono ou gerente do estacionamento privado, e permite verificar a ocupação do estacionamento, seus horários de pico, controlar entrada e saída de clientes, gerar relatórios de tendências, entre outras funcionalidades.

Este plano de projeto foi baseado nos conhecimentos propostos na quinta edição do \cite{project2013guia}. A organização deste trabalho segue conforme os grupos de processos e as áreas de conhecimento para cada documento criado. O plano foca em apresentar e descrever os processos utilizados durante o gerenciamento deste projeto.

 \textbf{Palavras-chave}: vaga-livre, estacionamento privado, plano de projeto, PMBOK.
\end{resumo}