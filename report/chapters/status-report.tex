\begin{landscape}

\chapter{Relatório de acompanhamento do projeto}
\label{ch:status-report}

\section{Identificação do projeto}

\begin{longtable}{ p{0.2\textwidth} p{1.25\textwidth} }
    \toprule
    \endhead
	\multicolumn{2}{c}{{\textit{Continua na próxima página.}}} \\
	\caption{Identificação do projeto.}
	\endfoot
	\endlastfoot
    \textbf{Projeto:} & \foreign{\{Vaga Livre\}} \\
    \midrule
    \textbf{Data:} & \foreign{\{Data em que o relatório foi desenvolvido\}} \\
    \midrule
    \textbf{Responsável:} & \foreign{\{Nome de quem elaborou o relatório\}} \\
    \bottomrule
    \caption{Identificação do projeto.}
    \centering
\end{longtable}

\section{Marcos do projeto}

\begin{longtable}{ >{\centering\arraybackslash}p{0.65\textwidth} >{\centering\arraybackslash}p{0.25\textwidth} >{\centering\arraybackslash}p{0.25\textwidth} >{\centering\arraybackslash}p{0.25\textwidth} }
    \toprule
    \thead[c]{\textbf{Entrega}} & \thead[c]{\textbf{Data Planejada}} & \thead[c]{\textbf{Data Realizada}} & \thead[c]{\textbf{Status}} \\
    \midrule
    \endhead
	\multicolumn{4}{c}{{\textit{Continua na próxima página.}}} \\
	\caption{Marcos do projeto.}
	\endfoot
	\endlastfoot

    \foreign{\{Descrever atividade ou entrega\}} & \foreign{\{Data prevista\}} & \foreign{\{Data de realização da atividade\}} & \foreign{\{Percentual completado da atividade.\}} \\
    \midrule
    &&&\\
    \midrule
    &&&\\
    \midrule
    &&&\\
    \midrule
    &&&\\

    \caption{Marcos do projeto.}
    \centering
\end{longtable}

\section{Análise desempenho}

O relatório de análise de valor agregado, gerado pelo Microsoft Project (ver apêndice \ref{project-monitoring-report}), deverá ser anexado a este documento.

\section{Principais riscos}

\begin{longtable}{ >{\centering\arraybackslash}p{0.4\textwidth} >{\centering\arraybackslash}p{0.2\textwidth} >{\centering\arraybackslash}p{0.4\textwidth} >{\centering\arraybackslash}p{0.4\textwidth} }
    \toprule
    \thead[c]{\textbf{Risco}} & \thead[c]{\textbf{Exposição}} & \thead[c]{\textbf{Plano de Resposta}} & \thead[c]{\textbf{Comentários}} \\
    \midrule
    \endhead
	\multicolumn{4}{c}{{\textit{Continua na próxima página.}}} \\
	\caption{Marcos do projeto.}
	\endfoot
	\endlastfoot

    \foreign{\{Descrever risco\}} & \foreign{\{Exposição ao risco\}} & \foreign{\{Ação de resposta ao risco\}} & \foreign{\{Comentários sobre o risco\}} \\
    \midrule
    &&&\\
    \midrule
    &&&\\
    \midrule
    &&&\\
    \midrule
    &&&\\

    \caption{Marcos do projeto.}
    \centering
\end{longtable}

\subsection{Avaliação}

\begin{longtable}{ p{0.2\textwidth} p{1.25\textwidth} }
    \toprule
    \endhead
	\multicolumn{2}{c}{{\textit{Continua na próxima página.}}} \\
	\caption{Avaliação do andamento do projeto.}
	\endfoot
	\endlastfoot
    \textbf{Observações:} & \foreign{\{Observações sobre o andamento do projeto\}} \\
    \midrule
    \textbf{Responsável:} & \foreign{\{Nome de quem realizou a avaliação\}} \\
    \midrule
    \textbf{Assinatura:} & \foreign{\{Assinatura de quem realizou a avaliação\}} \\
    \bottomrule
    \caption{Avaliação do andamento do projeto.}
    \centering
\end{longtable}

\section{Anexos opcionais}

Os seguintes documentos poderão ser anexados a este relatório conforme necessidade:

\begin{itemize}
    \item Registro das questões (ver apêndice \ref{ch:question-register}).
    \item Mudanças no nível de engajamento das partes interessadas.
    \item Planos de ação para recuperação de atraso, engajamento das partes interessadas e problemas esperados.
    \item Métricas de qualidade.
\end{itemize}

\end{landscape}