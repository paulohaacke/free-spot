%Descrição do processo de definição das atividades (base na decomposição dos pacotes da EAP normalmente)
%Descrição do processo e técnicas de sequenciamento das atividades
%Descrição do processo de estimativa de recursos para as atividades
%Descrição do método de gerenciamento do tempo que será empregado no projeto (PERT/CPM, corrente crítica, SCRUM, ...) e a justificativa para sua aplicação no projeto
%Elementos visuais:
%   Visão executiva do cronograma em formato gráfico (Gantt, Burndown, ...), deve ocupar apenas uma página
%   Principais marcos do projeto na visão do cliente final, patrocinador ou gerência imediata / tem no escopo
%   Cronograma detalhado do projeto com identificação do caminho crítico no formato Gantt contendo: Atividade, Duração, Data de início, Data de fim, Percentual de progresso
%   Lista de atividades do caminho crítico com datas de início e fim
%Apresentação do processo empregado para gerenciamento da linha de base de tempo
%Descrição do processo de controle do cronograma
%   Frequência das ações de controle
%   Eventos listados na Matriz de Comunicações
%   Remete ao Controle Integrado de Mudanças em caso de divergência planejado x realizado

\chapter{Plano de gerenciamento do tempo}

\section{Descrição dos processos de gerenciamento do tempo}

\begin{itemize}
	\item O gerenciamento de tempo será realizado a partir da alocação de precentual completo nas atividades do projeto através da utilização do software \projectManagementSoftwareName{}.
	\item Utilizando o método do caminho crítico o gerente de projeto criou o cronograma do projeto. O gráfico de Gannt representando o cronograma encontra-se no apêndice \ref{ch:gannt-chart}.
	\item A atualização dos prazos do projeto será realizada no sofwtare \projectManagementSoftwareName{} através da publicação no repositório de documentos do projeto dos seguintes relatórios:
	      \begin{itemize}
		      \item Gráfico de Gannt;
		      \item Percentual completo.
	      \end{itemize}
	\item A avaliação de desempenho do projeto será realizada através da análise de valor agregado, onde o custo e o prazo do projeto são acompanhados em um único processo de controle.
	\item Todas atividades com folga menor ou igual a 3 dias serão consideradas críticas.
	\item Todas mudanças no prazo inicialmente previsto para o projeto devem ser avaliadas e classificadas de acordo com o sistema de controle integrado de mudanças (ver seção \ref{sec:change-control-system}).
	\item Atrasos decorrentes de medidas corretivas e que influenciam no sucesso do projeto serão integrados ao plano. Novos recursos terão seus prazos negociados na reunião do CCM.
	\item Atualizações na linha de base do projeto somente serão permitidas com autorização expressa do gerente de projeto e do patrocinador. A linha de base anterior deve ser arquivada, documentada e publicada para fins de lições aprendidas.
	\item Solicitações de mudanças no crognorama definido deverão ser realizadas através do formulário de requisição de mudança (ver apêndice \ref{ch:change-request-form}), conforme previsto no plano de gerenciamento das mudanças (ver capítulo \ref{ch:change-management-plan})
	\item A estimativa de tempo para cada atividade deve ser determinada utilizando a técnica de PERT, conforme equação \ref{eq:time-pert}.

	      \begin{equation}\label{eq:time-pert}
		      DE = \frac{DO+4DMP+DP}{6}
	      \end{equation}

	      \begin{description}
		      \item[Duração Estimada (DE):] a melhor estimativa da duração necessária para completar a atividade, levando em consideração o fato de que o projeto nem sempre corre conforme o planejado.
		      \item[Duração Mais Provável (DMP):] a melhor estimativa da duração necessária para completar a atividade, assumindo que tudo ocorrerá normalmente.
		      \item[Duração Otimista (DO):] o menor tempo em que é possível completar a atividade, assumindo que tudo ocorra melhor do que é normalmente esperado.
		      \item[Duração Pessimista (DP):] a maior duração possível para completar a atividade, considerando que tudo que possa dar errado aconteça (excluindo o caso de catástrofes extremamente raras).
	      \end{description}
		  
\end{itemize}

\section{Priorização das mudanças nos prazos e respostas}

As mudanças nos prazos deverão ser classificadas de acordo com o modelo de priorização integrada de mudanças (ver seção \ref{sec:integrated-change-priorization}).

\section{Sistema de controle de mudanças de prazos}

Todas as mudanças nos prazos e atrasos/adiantamentos do projeto devem ser tratados segundo o fluxo apresentado pelo sistema de controle integrado das mudanças (ver seção \ref{sec:change-control-system}).

%\section{Mecanismo adotado para conflitos de recursos}

\section{Buffer de tempo do projeto}

O projeto adotará o conceito de caminho crítico, sendo assim o projeto não prevê a criação de uma folga ou atraso no término do projeto baseado nos conceitos de corrente crítica.

%Deverão ser criados pulmões de caminho para proteger uma corrente crítica e também pulmões de projeto, para proteger a data final do projeto contra possíveis problemas.

%Para gerenciar estes pulmões o gerente de projeto deverá dividir cada pulmão em três partes. Enquanto o uso estiver no primeiro terço se considera que está controlado. No segundo terço %o gerente de projeto volta sua atenção para as atividades relacionadas a este pulmão e busca descobrir se existem indícios de que a atividade não irá terminar no prazo, se encontrar %tais indícios o gerente de projeto pode trazer os mesmos para serem discutidos durante a reunião semanal do CCM. Se já estiver utilizando o último terço do pulmão o gerente de projeto %deve acionar o CCM solicitando uma mudança de prazo ou escopo.

\section{Frequência de avaliação dos prazos do projeto}

Os prazos do projeto deverão ser avaliados e atualizados semanalmente, sendo os resultados publicados no repositório de documentos do projeto e apresentados durante a reunião semanal do CCM, pervista no plano de gerenciamento das comunicações.

\section{Alocação financeira para o gerenciamento de tempo}

As mudanças de recuperação de atraso no projeto que requerem gastos adicionais podem ser alocadas dentro das reservas gerenciais do projeto de acordo com as necessidades do gerente de projeto.

Conforme descrito no plano de gerenciamento dos custos (ver capítulo \ref{ch:cost-management-plan}), em caso que haja necesidade de medidas prioritárias para a recuperação de prazos, em momentos que não existam mais reservas gerenciais disponíveis, deverá ser acionado o patrocinador, já que o gerente de projeto não possui autonomia para decidir utilizar a reserva de contingência de riscos para a recuperação de atrasos.

\section{Administração do plano de gerenciamento de tempo}

\subsection{Responsável}

\begin{itemize}
	\item \projectManagerName{}, gerente de projeto, será o responsável direto pelo plano de gerenciamento do tempo.
\end{itemize}

\subsection{Frequência de atualização}

O plano de gerenciamento do tempo será reavaliado mensalmente durante a reunião do CCM, juntamente com os outros planos de gerenciamento do projeto.

\section{Outros assuntos relacionados ao gerenciamento do tempo do projeto não previstos neste plano}

Solicitações não previstas neste plano deverão ser submetidas a reunião do CCM para aprovação. O plano de gerenciamento do tempo deverá ser atualizado com as devidas alterações realizadas.

\section{Controle de Versão}

\begin{table}[H]
	\begin{tabularx}{\textwidth}{| c | c | X | X |}
		\hline
		\textbf{Versão} & \textbf{Data} & \textbf{Autor}      & \textbf{Notas de Revisão} \\
		\hline
		1                &               & \projectManagerName{} & Criação do documento     \\
		\hline
	\end{tabularx}
	\centering
\end{table}

\section{Aprovações}

\begin{table}[H]
	\begin{tabularx}{\textwidth}{| c | c | X | c |}
		\hline
		\textbf{Função}  & \textbf{Nome}       & \textbf{Assinatura}      & \textbf{Data} \\
		\hline
		Patrocinador       & \projectSponsorName{} & \projectSponsorSignature{} &               \\
		\hline
		Gerente de projeto & \projectManagerName{} & \projectManagerSignature{} &               \\
		\hline
	\end{tabularx}
	\centering
\end{table}