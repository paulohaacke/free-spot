%Tempo
%Descrição do processo de definição das atividades (base na decomposição dos pacotes da EAP normalmente)
%Descrição do processo e técnicas de sequenciamento das atividades
%Descrição do processo de estimativa de recursos para as atividades
%Descrição do método de gerenciamento do tempo que será empregado no projeto (PERT/CPM, corrente crítica, SCRUM, ...) e a justificativa para sua aplicação no projeto
%Elementos visuais:
%   Visão executiva do cronograma em formato gráfico (Gantt, Burndown, ...), deve ocupar apenas uma página
%   Principais marcos do projeto na visão do cliente final, patrocinador ou gerência imediata / tem no escopo
%   Cronograma detalhado do projeto com identificação do caminho crítico no formato Gantt contendo: Atividade, Duração, Data de início, Data de fim, Percentual de progresso
%   Lista de atividades do caminho crítico com datas de início e fim
%Apresentação do processo empregado para gerenciamento da linha de base de tempo
%Descrição do processo de controle do cronograma
%   Frequência das ações de controle
%   Eventos listados na Matriz de Comunicações
%   Remete ao Controle Integrado de Mudanças em caso de divergência planejado x realizado

\chapter{Plano de gerenciamento do cronograma}

\todo[inline,color=red]{Criar plano de gerenciamento do cronograma.}

\section{Descrição dos processos de gerenciamento de tempo}

\section{Priorização das mudanças nos prazos e respostas}

\section{Sistema de controle de mudanças de prazos}

\section{Mecanismo adotado para conflitos de recursos}

\section{Buffer de tempo do projeto}

\section{Frequência de avaliação dos prazos do projeto}

\section{Alocação financeira para o gerenciamento de tempo}

\section{Administração do plano de gerenciamento de tempo}

\subsection{Responsável}

\begin{itemize}
	\item \projectManagerName, gerente de projeto, será o responsável direto pelo plano de gerenciamento do cronograma.
\end{itemize}
\todo[inline,color=orange]{Verificar necessidade de adicionar suplente responsável.}

\subsection{Frequência de atualização}

O plano de gerenciamento do cronograma será reavaliado mensalmente durante a reunião do CCM, juntamente com os outros planos de gerenciamento do projeto.

\section{Outros assuntos relacionados ao gerenciamento do cronograma do projeto não previstos neste plano}

\section{Controle de Versão}

\begin{table}[H]
	\begin{tabularx}{\textwidth}{| c | c | X | X |}
		\hline
		\textbf{Versão} & \textbf{Data} & \textbf{Autor}      & \textbf{Notas de Revisão} \\
		\hline
		1                &               & \projectManagerName & Criação do documento     \\
		\hline
	\end{tabularx}
	\centering
\end{table}

\section{Aprovações}

\begin{table}[H]
	\begin{tabularx}{\textwidth}{| c | c | X | c |}
		\hline
		\textbf{Função}  & \textbf{Nome}       & \textbf{Assinatura}      & \textbf{Data} \\
		\hline
		Gerente de projeto & \projectManagerName & \projectManagerSignature &               \\
		\hline
	\end{tabularx}
	\centering
\end{table}