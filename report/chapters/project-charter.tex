\section{Termo de Abertura}

\subsection{Nome do Projeto}

Vaga Livre

\subsection{Patrocinador}

\projectSponsorName

%Poderia colocar a minha empresa ou uma empresa contratante, provisoriamente deixei como sendo minha empresa
%Nome e autoridade do patrocinador ou outra(s) pessoa(s) que autoriza(m) o termo de abertura do projeto.

\subsection{Gerente do Projeto}

\projectManagerName

%Gerente do projeto, responsabilidade, nível de autoridade designados.

\subsection{Descrição}

Habilitar gerentes de estacionamentos privados a controlar a ocupação de vagas em seu estacionamento, incluindo visualizar o histórico de ocupação e demanda atual, oferecer um serviço de reserva com rígido controle.
O consumidor será beneficiado através de um aplicativo móvel, onde poderá realizar reservas e encontrar vagas em estacionamentos diversos.

%Descrição de alto nível do projeto e seus limites.

\subsection{Justificativa}

O mercado de estacionamentos privados tem evoluído muito no Brasil e no Mundo, essa evolução trouxe grandes benefícios para o usuário, entretanto ainda existe demanda para a melhoria nos serviços em áres pouco exploradas, como a busca de vagas e o congestionamento das cidades.
Por outro lado estacionamentos privados encontram dificuldades em alocar de forma eficiente seus estacionamentos.
\todo[inline,color=red]{Melhorar descrição da oportunidade de mercado.}
Através do desenvolvimento deste sistema pretende-se:
\begin{itemize}
	\item Maximizar a ocupação dos estacionamentos;% O que significa isso?
	\item Maximizar o lucro;
	\item Atingir objetivos estratégicos;
	\item Desenvolver estratégias de marketing de acordo com ocupação em períodos anteriores;
	\item Oferecer um melhor serviço aos seus consumidores: permitindo a reserva e busca de vagas.
\end{itemize}

%Finalidade ou justificativa do projeto.

\subsection{Objetivos}

%%Diminuir o tempo médio de espera em fila para 10 minutos. Diminuir o número de vagas não ocupadas nos piores horários para 50.

\begin{itemize}
	\item Desenvolver um aplicativo de celular portável tanto para Android quanto para IOS para encontrar e reservar vagas de estacionamento;
	\item Desenvolver um software que permita aos estacionamento privados controlar a demanda e ocupação de seus estacionamentos;
	\item O aplicativo e o software devem estar prontos até \maximumDeadline;
	\item O aplicativo deve suportar até \minimumUsersAmount usuários;
	\item O orçamento é de até \maximumBudget;
\end{itemize}

%Objetivos mensuráveis do projeto e critérios de sucesso relacionados.

\subsection{Requisitos}

Requisitos de alto nível.

\subsection{Premissas}

Premissas e restrições.

\subsection{Restrições}

Premissas e restrições.

\subsection{Riscos}

Riscos de Alto Nível

\subsection{Marcos}

Resumo do Cronograma de Marcos

\subsection{Orçamento}

Resumo do Orçamento

\subsection{Partes Interessadas}

Lista das Partes Interessadas.

\subsection{Requisitos para aprovação do projeto}

Requisitos para aprovação do projeto (ou seja, o que constitui o sucesso do projeto, quem decide se
o projeto é bem sucedido e quem assina o projeto).

\subsection{Controle de Versão}

\subsection{Aprovações}

\todo[inline,color=red]{Finalizar termo de abertura.}