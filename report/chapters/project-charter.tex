% A fazer
% Duração, Data de início e fim
% Premissas e restrições

\section{Termo de Abertura}

\subsection{Nome do Projeto}

Vaga Livre

\subsection{Patrocinador}

\projectSponsorName

%Poderia colocar a minha empresa ou uma empresa contratante, provisoriamente deixei como sendo minha empresa
%Nome e autoridade do patrocinador ou outra(s) pessoa(s) que autoriza(m) o termo de abertura do projeto.

\subsection{Gerente do Projeto}

\projectManagerName é o gerente do projeto. Sua autoridade é total em relação ao desenvolvimento do produto, podendo contratar, realizar compras e gerenciar o pessoal de acordo com seus próprios critérios.

Em relação ao aspecto financeiro, a autoridade do gerente de projeto estará limitada a determinadas autonomias, a serem definidas no plano de gerenciamento de custos.

%Gerente do projeto, responsabilidade, nível de autoridade designados.

\subsection{Descrição}

Habilitar gerentes de estacionamentos privados a controlar a ocupação de vagas em seu estacionamento, incluindo visualizar o histórico de ocupação e demanda atual, oferecer um serviço de reserva com rígido controle.

O consumidor será beneficiado através de um aplicativo móvel, onde poderá realizar reservas e encontrar vagas em estacionamentos diversos.

%Descrição de alto nível do projeto e seus limites.

\subsection{Justificativa}

O mercado de estacionamentos privados tem evoluído muito no Brasil e no Mundo, essa evolução trouxe grandes benefícios para o usuário, entretanto ainda existe demanda para a melhoria nos serviços em áres pouco exploradas, como a busca de vagas e o congestionamento das cidades.

Por outro lado estacionamentos privados encontram dificuldades em alocar de forma eficiente seus estacionamentos.

\todo[color=orange]{Melhorar descrição da oportunidade de mercado.}

Através do desenvolvimento deste sistema pretende-se:
\begin{itemize}
	\item Maximizar a ocupação dos estacionamentos.% O que significa isso?
	\item Maximizar o lucro.
	\item Atingir objetivos estratégicos.
	\item Desenvolver estratégias de marketing de acordo com ocupação em períodos anteriores.
	\item Oferecer um melhor serviço aos seus consumidores: permitindo a reserva e busca de vagas.
\end{itemize}

\subsection{Objetivos}

%%Diminuir o tempo médio de espera em fila para 10 minutos. Diminuir o número de vagas não ocupadas nos piores horários para 50.

\begin{itemize}
	\item Desenvolver um aplicativo de celular portável tanto para Android quanto para IOS para encontrar e reservar vagas de estacionamento.
	\item Desenvolver um software que permita aos estacionamento privados controlar a demanda e ocupação de seus estacionamentos.
	\item O aplicativo e o software devem estar prontos até \maximumDeadline.
	\item O aplicativo deve suportar até \minimumUsersAmount usuários.
\end{itemize}

%Objetivos mensuráveis do projeto e critérios de sucesso relacionados.

%\subsection{Requisitos}

%Requisitos de alto nível.

\subsection{Premissas Iniciais}

\begin{itemize}
	\item A equipe está motivida para trabalhar no projeto.
	\item Os estacionamentos privados de Porto Alegre estão prontos para aderir a este tipo de tecnologia.
	\item Utilizar base de dados em cloud.
	\item Utilizar um framework multi-plataforma para o desenvolvimento do aplicativo para dispositivos móveis.
	\item Disponibilidade do patrocinador: durante o planejamento no mínimo 40\% do tempo, e durante a execução no mínimo 80\% do tempo.
\end{itemize}

\subsection{Restrições Iniciais}

\begin{itemize}
	\item O orçamento é limitado a \maximumBudget.
	\item O modo de funcionamento básico do sistema não deve depender de qualquer tipo de tecnologia de ponta.
	\item Data de conclusão do projeto: \maximumDeadline.
\end{itemize}

\subsection{Riscos Iniciais}

\begin{itemize}
	\item Segurança das informações: dados de usuários e transações financeiras.
	\item Atraso nas entregas.
	\item Equipe insuficiente ou indisponível.
	\item Falta de aderência dos estacionamentos privados. %VER
\end{itemize}

\subsection{Marcos}

\todo[color=orange]{Escrever marcos}

A tabela \ref{tab:marcos} descreve os principais marcos de entrega do projeto.

\begin{table}[h]
	\begin{tabularx}{.9\textwidth}{| X | c | c |}
		\hline
		\textbf{Marco}                                                   & \textbf{Prazo} & \textbf{Custo} \\
		\hline
		Detalhamento de requisitos e mapeamento de processos nananananan & 1 mês         & R\$30.000,00   \\
		\hline
		Detalhamento de requisitos e mapeamento de processos nananananan & 1 mês         & R\$30.000,00   \\
		\hline
		Detalhamento de requisitos e mapeamento de processos nananananan & 1 mês         & R\$30.000,00   \\
		\hline
	\end{tabularx}
	\centering
	\caption{Principais datas de marcos para entregas do projeto.}
	\label{tab:marcos}
\end{table}

\subsection{Partes Interessadas Iniciais}

\begin{itemize}
	\item Equipe do projeto.
	\item Patrocinador.
	\item Pessoas (jurídicas e físicas) que utilizam de serviços de estacionamentos privados.
	\item Motoristas que compartilham a estrada com carros em busca de estacionamento.
	\item Funcionários e donos de empreendimentos na área de estacionamentos privados.
	\item Organizações que possuem interesse ou sofrem influência de políticas relacionadas ao meio ambiente.
	\item Fabricantes de automóveis.
	\item Empresas de transporte coletivo.
\end{itemize}

%\subsection{Requisitos para aprovação do projeto}

%Requisitos para aprovação do projeto (ou seja, o que constitui o sucesso do projeto, quem decide se
%o projeto é bem sucedido e quem assina o projeto).

\subsection{Controle de Versão}

\begin{table}[H]
	\begin{tabularx}{.9\textwidth}{| c | c | X | X |}
		\hline
		\textbf{Versão} & \textbf{Data} & \textbf{Autor}      & \textbf{Notas de Revisão} \\
		\hline
		1                & 2017-04-29    & Paulo André Haacke & Criação do documento     \\
		\hline
	\end{tabularx}
	\centering
	\caption{Tabela para controle de versão do termo de abertura.}
\end{table}

\subsection{Aprovações}

\begin{table}[H]
	\begin{tabularx}{\textwidth}{| c | c | X | c |}
		\hline
		\textbf{Função} & \textbf{Nome}       & \textbf{Assinatura}      & \textbf{Data} \\
		\hline
		Patrocinador      & \projectSponsorName & \projectSponsorSignature &               \\
		\hline
	\end{tabularx}
	\centering
\end{table}