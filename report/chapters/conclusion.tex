\part{Considerações finais}

\chapter[Conclusão]{Conclusão}

Este trabalho apresentou o planejamento do projeto \projectName{} com base nos conhecimentos apresentados em \cite{project2013guia}. Foram criados planos cobrindo cada uma das 10 áreas de conhecimento, utilizando as ferramentas e técnicas conforme necessário. 

O projeto aqui apresentado tratou sobre a criação de uma solução de software que envolve a criação de dois aplicativos voltados para a área de estacionamentos privados: um focado no dono do estacionamento e outro focado no cliente - o motorista em busca de uma vaga para estacionar seu carro. 

Inicialmente foi criado o termo de abertura e identificadas as principais partes interessadas. Após a aprovação do patrocinador em 20/04/2017, foi realizada a reunião de \foreign{kick-off} com equipe e projeto estava iniciado. Os próximos passos foram a criação dos planos de projeto e da linha de base do escopo. Os planos e o escopo foram aprovados pelo patrocinador em 06/06/2017. Iniciou-se então a execução do plano.

Este trabalho contemplou apenas o planejamento do projeto e uma simulação parcial da execução. Todos os planos e os artefatos necessários foram devidamente documentados. Caso o projeto fosse executado conforme os planos há indícios que os objetivos definidos inicialmente seriam atingidos.

Não é possível concluir que o plano desenvolvido para este projeto obtenha 100\% de sucesso, mesmo que aplicadas as técnicas de gestão adequadas. Entretanto, considerando o fato de todos os planos estarem integrados e ter sido aplicada consolidadas práticas de gestão, é possível afirmar que as possibilidades de falha serão minimizadas.

Portanto, pode-se concluir que, utilizar práticas de gestão consolidadas e desenvolver um plano de gerenciamento adequado são técnicas importantes para alcançar o sucesso do projeto, nesse contexto é essencial a presença de um gestor capacitado.

