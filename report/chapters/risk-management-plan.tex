% Descrição do processo de gerenciamento de riscos empregado no projeto
% Descrição do processo de identificação dos riscos

% Lista de riscos do projeto
% Priorização dos riscos do projeto
% Apresentar a EAR do projeto
% Apresentar a matriz de probabilidade e impactos
% Apresentar a relação de ameaças e oportunidades do projeto
% Desenvolver uma análise numérica dos riscos do projeto: Análise qualitativa, Análise quantitativa
% Desenvolver o cálculo do VME com os 3 cenários, refletindo o valor do cenário provável como Reservas de Contingência nos Custos
% Apresentar o plano de respostas aos riscos do projeto
% Processo de monitoramento e controle dos riscos do projeto
%   Métodos de monitoramento
%   Frequência das ações de monitoramento
%   Eventos listados na Matriz de Comunicações
%   Remete ao Controle Integrado de Mudanças em caso de divergência planejado x realizado 

\chapter{Plano de gerenciamento dos riscos}

\todo[inline,color=green]{Criar plano de gerenciamento dos riscos.}

\section{Descrição dos processos de gerenciamento de riscos}

\begin{itemize}
%\item Atividades de alto risco deverão tornar-se prioritárias, de modo que sejam desenvolvidas o mais cedo possível durante a execução do projeto.
%\item Projetos de software com ciclo de vida adaptativo escolhem requisitos e histórias de usuário do backlog, o qual pode sofrer frequentes repriorizações; o que permite o gerenciamento de risco o mais cedo possível no projeto, minimizando efeitos de atraso e composição.
\item Os riscos do projeto serão identificados pelo time do projeto, utilizando-se da EAR (ver seção \ref), sugere-se a utilização das seguintes técnicas:
\begin{itemize} 
\item Brainstorming. 
\item Grupo nominal. 
\item Slip de Crawford.
\todo[inline,color=red]{Descrever técnicas de identificação de risco.}
\end{itemize} 
\item Os riscos do projeto serão gerenciados com base nos riscos identificados previamente, assim como no monitoramento e controle de novos riscos que podem não ter sido identificados anteriormente.
\item Todos os riscos não previstos no plano devem ser incluídos ao projeto de acordo com sistema de controle de mudanças de riscos (ver seção \ref{sec:risk-change-control-system}).
\item A identificação, a avaliação e o monitoramento de riscos devem ser feitos por escrito ou através de e-mail, conforme descrito no plano de gerenciamento das comunicações (ver capítulo \ref{ch:communication-management-plan}).
\item Durante a reunião do CCM serão avaliados e monitorados os riscos do projeto.
\end{itemize}

\section{Estrutura analítica de riscos para identificação dos riscos}

Para auxiliar na compreensão, gerenciamento e comunicação de riscos em projetos, a melhor forma de apresentar as informações referentes aos riscos de um projeto, é de forma estruturada \cite{hillson2002risk}, sistemática e explícita \cite{gusmao2007modelo}.

Com o objetivo de auxiliar o gerente de projetos na identificação dos riscos será utilizado o modelo de estrutura analítica de riscos proposta em \cite{dorofee1996continuous}, a qual pode ser vista na figura \ref{fig:rbs}. Esta estrutura ajuda a equipe do projeto a considerar muitas fontes a partir dos quais os riscos podem surgir.

\begin{figure}[h]
\centering
\begin{tikzpicture}[node distance = 0.3cm and 0.7cm, auto]

\node (prodeng) [wbsblock] {Engenharia de produto};
\node (devenv) [wbsblock, right = of prodeng, xshift=2.5em] {Ambiente de desenvolvimento};
\node (constraints) [wbsblock, right = of devenv, xshift=2.5em] {Restrições};
% Engenharia de Produto
\node (req) [wbsblock, below right = of prodeng, xshift=-6em] {Requisitos};
\node (design) [wbsblock, below = of req] {Projeto};
\node (code) [wbsblock, below = of design] {Código e testes unitários};
\node (integ) [wbsblock, below = of code] {Testes de integração};
\node (engattr) [wbsblock, below = of integ] {Propriedades de Engenharia};
% Ambiente de Desenvolvimento
\node (devproc) [wbsblock, below right = of devenv, xshift=-6em] {Processo de desenvolvimento};
\node (devsys) [wbsblock, below = of devproc] {Sistema de desenvolvimento};
\node (mngtproc) [wbsblock, below = of devsys] {Processo de gerenciamento};
\node (mngtmet) [wbsblock, below = of mngtproc] {Métodos de gerenciamento};
\node (workenv) [wbsblock, below = of mngtmet] {Ambiente de trabalho};
% Restrições
\node (resources) [wbsblock, below right = of constraints, xshift=-6em] {Recursos};
\node (contracts) [wbsblock, below = of resources] {Contratos};
\node (interfaces) [wbsblock, below = of contracts] {Interfaces};
% RBS root
\node (projrisk) [wbsblock, above = of devenv, yshift=3em] {Risco do Projeto};

\path [simpleline] (projrisk.south) |- ++(0,-1cm) -| (prodeng);
\path [simpleline] (projrisk.south) |- ++(0,-1cm) -| (devenv);
\path [simpleline] (projrisk.south) |- ++(0,-1cm) -| (constraints);

\path [simpleline] (prodeng.215) |- (req);
\path [simpleline] (prodeng.215) |- (design);
\path [simpleline] (prodeng.215) |- (code);
\path [simpleline] (prodeng.215) |- (integ);
\path [simpleline] (prodeng.215) |- (engattr);

\path [simpleline] (devenv.215) |- (devproc);
\path [simpleline] (devenv.215) |- (devsys);
\path [simpleline] (devenv.215) |- (mngtproc);
\path [simpleline] (devenv.215) |- (mngtmet);
\path [simpleline] (devenv.215) |- (workenv);

\path [simpleline] (constraints.215) |- (resources);
\path [simpleline] (constraints.215) |- (contracts);
\path [simpleline] (constraints.215) |- (interfaces);

\end{tikzpicture}
\caption{Estrutura analítica de riscos.}
\label{fig:rbs}
\end{figure}

\section{Riscos identificados}

Os riscos identificados no projeto, segundo a EAP do projeto e a EAR apresentada, estão listados a seguir.

\begin{enumerate}
\item Projeto Vaga Livre
\begin{enumerate}
\item Gerenciamento do projeto
\item Infraestrutura
\item Banco de dados
\item Aplicativo do motorista
\item Aplicativo do estacionamento
\item Auditoria
\item Encerramento
\end{enumerate}
\end{enumerate}

\section{Qualificação dos riscos}

Os riscos identificados serão classificados de acordo com sua probabilidade de ocorrência e impacto de seus resultados, conforme descrito a seguir.

\subsection{Probabilidade de ocorrência}

\begin{description}
\item [Baixa] A probabilidade de o risco ocorrer é considerada pequena, menor que 25\%.
\item [Média] Existe razoável probabilidade de o risco ocorrer, entre 25\% e 60\%.
\item [Alta] O risco possui alta probabilidade de ocorrência, maior que 60\%. 
\end{description}

\subsection{Impacto}

\begin{description}
\item [Baixo] Caso o evento venha a ocorrer, o impacto no projeto é insignificante, podendo ser facilmente resolvido.
\item [Médio] A ocorrência do evento de risco pode causar impacto relevante no projeto, sendo necessário um gerenciamento mais preciso, sob pena de prejudicar o resultado do projeto.
\item [Alto] A ocorrência do evento de risco impacta de forma extremamente significativa no projeto, exigindo ação precisa e imediata por parte da equipe do projeto, os resultados estarão comprometidos.
\end{description}

\section{Quantificação dos riscos}

\section{Sistema de controle de mudanças de riscos}
\label{sec:risk-change-control-system}

\section{Respostas planejadas aos riscos}

\section{Reservas de contingência}

\section{Frequência de avaliação dos riscos do projeto}

\section{Alocação financeira para o gerenciamento de riscos}

\section{Administração do plano de gerenciamento de riscos}

\subsection{Responsável}

\begin{itemize}
	\item \projectManagerName{}, gerente de projeto, será o responsável direto pelo plano de gerenciamento de riscos.
\end{itemize}

\section{Outros assuntos relacionados ao gerenciamento de riscos do projeto não previstos neste plano}

Solicitações não previstas neste plano deverão passar pela aprovação do CCM. Após aprovada o plano deve ser atualizado pelo gerente do projeto.

\section{Controle de Versão}

\begin{table}[H]
	\begin{tabularx}{\textwidth}{| c | c | X | X |}
		\hline
		\textbf{Versão} & \textbf{Data} & \textbf{Autor}      & \textbf{Notas de Revisão} \\
		\hline
		1                &               & \projectManagerName{} & Criação do documento     \\
		\hline
	\end{tabularx}
	\centering
\end{table}

\section{Aprovações}

\begin{table}[H]
	\begin{tabularx}{\textwidth}{| c | c | X | c |}
		\hline
		\textbf{Função}  & \textbf{Nome}       & \textbf{Assinatura}      & \textbf{Data} \\
		\hline
		Patrocinador       & \projectSponsorName{} & \projectSponsorSignature{} &               \\
		\hline
		Gerente de projeto & \projectManagerName{} & \projectManagerSignature{} &               \\
		\hline
	\end{tabularx}
	\centering
\end{table}