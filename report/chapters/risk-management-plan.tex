% Descrição do processo de gerenciamento de riscos empregado no projeto 10
% Descrição do processo de identificação dos riscos 8

% Lista de riscos do projeto
% Priorização dos riscos do projeto
% Apresentar a EAR do projeto 9
% Apresentar a matriz de probabilidade e impactos
% Apresentar a relação de ameaças e oportunidades do projeto
% Desenvolver uma análise numérica dos riscos do projeto: Análise qualitativa, Análise quantitativa
% Desenvolver o cálculo do VME com os 3 cenários, refletindo o valor do cenário provável como Reservas de Contingência nos Custos
% Apresentar o plano de respostas aos riscos do projeto
% Processo de monitoramento e controle dos riscos do projeto
%   Métodos de monitoramento
%   Frequência das ações de monitoramento
%   Eventos listados na Matriz de Comunicações
%   Remete ao Controle Integrado de Mudanças em caso de divergência planejado x realizado 

\chapter{Plano de gerenciamento dos riscos}

\section{Descrição dos processos de gerenciamento de riscos}

\begin{itemize}
	%\item Atividades de alto risco deverão tornar-se prioritárias, de modo que sejam desenvolvidas o mais cedo possível durante a execução do projeto.
	%\item Projetos de software com ciclo de vida adaptativo escolhem requisitos e histórias de usuário do backlog, o qual pode sofrer frequentes repriorizações; o que permite o gerenciamento de risco o mais cedo possível no projeto, minimizando efeitos de atraso e composição.
	\item Os riscos do projeto serão identificados pelo time do projeto, utilizando-se da estrutura analítica de riscos (EAR) (ver seção \ref{sec:rbs}), sugere-se a utilização das seguintes técnicas:
	      \begin{itemize}
		      \item \foreign{Brainstorming}.
		      \item Grupo nominal.
	      \end{itemize}
	\item Os riscos do projeto serão gerenciados com base nos riscos identificados previamente, assim como no monitoramento e controle de novos riscos que podem não ter sido identificados oportunamente.
	\item Todos os riscos não previstos no plano devem ser incluídos ao projeto de acordo com sistema de controle de mudanças de riscos (ver seção \ref{sec:risk-change-control-system}).
	\item A identificação, a avaliação e o monitoramento de riscos devem ser feitos por escrito ou através de e-mail, conforme descrito no plano de gerenciamento das comunicações (ver capítulo \ref{ch:communication-management-plan}).
	\item Durante a reunião do CCM serão avaliados e monitorados os riscos do projeto.
	\item Mudanças relacionadas ao plano de gerenciamento de riscos, que não são a inclusão ou atualização de riscos,  deverão ser tratadas conforme o plano de gerenciamento das mudanças, apresentado no capítulo \ref{ch:change-management-plan}.
\end{itemize}

\section{Estrutura analítica de riscos para identificação dos riscos}
\label{sec:rbs}

Para auxiliar na compreensão, gerenciamento e comunicação de riscos em projetos, a melhor forma de apresentar as informações referentes aos riscos de um projeto, é de forma estruturada \cite{hillson2002risk}, sistemática e explícita \cite{gusmao2007modelo}.

Com o objetivo de auxiliar o gerente de projetos na identificação dos riscos será utilizado o modelo de estrutura analítica de riscos proposta em \cite{dorofee1996continuous}, a qual pode ser vista na figura \ref{fig:rbs}. Esta estrutura ajuda a equipe do projeto a considerar muitas fontes a partir dos quais os riscos podem surgir.

\begin{figure}[h]
	\centering
	\begin{tikzpicture}[node distance = 0.3cm and 0.7cm, auto]

		\node (prodeng) [wbsblock] {Engenharia de produto};
		\node (devenv) [wbsblock, right = of prodeng, xshift=2.5em] {Ambiente de desenvolvimento};
		\node (constraints) [wbsblock, right = of devenv, xshift=2.5em] {Restrições};
		% Engenharia de Produto
		\node (req) [wbsblock, below right = of prodeng, xshift=-6em] {Requisitos};
		\node (design) [wbsblock, below = of req] {Projeto};
		\node (code) [wbsblock, below = of design] {Código e testes unitários};
		\node (integ) [wbsblock, below = of code] {Testes de integração};
		\node (engattr) [wbsblock, below = of integ] {Propriedades de Engenharia};
		% Ambiente de Desenvolvimento
		\node (devproc) [wbsblock, below right = of devenv, xshift=-6em] {Processo de desenvolvimento};
		\node (devsys) [wbsblock, below = of devproc] {Sistema de desenvolvimento};
		\node (mngtproc) [wbsblock, below = of devsys] {Processo de gerenciamento};
		\node (mngtmet) [wbsblock, below = of mngtproc] {Métodos de gerenciamento};
		\node (workenv) [wbsblock, below = of mngtmet] {Ambiente de trabalho};
		% Restrições
		\node (resources) [wbsblock, below right = of constraints, xshift=-6em] {Recursos};
		\node (contracts) [wbsblock, below = of resources] {Contratos};
		\node (interfaces) [wbsblock, below = of contracts] {Interfaces};
		% RBS root
		\node (projrisk) [wbsblock, above = of devenv, yshift=3em] {Risco do Projeto};

		\path [simpleline] (projrisk.south) |- ++(0,-1cm) -| (prodeng);
		\path [simpleline] (projrisk.south) |- ++(0,-1cm) -| (devenv);
		\path [simpleline] (projrisk.south) |- ++(0,-1cm) -| (constraints);

		\path [simpleline] (prodeng.215) |- (req);
		\path [simpleline] (prodeng.215) |- (design);
		\path [simpleline] (prodeng.215) |- (code);
		\path [simpleline] (prodeng.215) |- (integ);
		\path [simpleline] (prodeng.215) |- (engattr);

		\path [simpleline] (devenv.215) |- (devproc);
		\path [simpleline] (devenv.215) |- (devsys);
		\path [simpleline] (devenv.215) |- (mngtproc);
		\path [simpleline] (devenv.215) |- (mngtmet);
		\path [simpleline] (devenv.215) |- (workenv);

		\path [simpleline] (constraints.215) |- (resources);
		\path [simpleline] (constraints.215) |- (contracts);
		\path [simpleline] (constraints.215) |- (interfaces);

	\end{tikzpicture}
	\caption{Estrutura analítica de riscos.}
	\label{fig:rbs}
\end{figure}

\section{Riscos identificados}
\label{sec:identified-risks}

Os riscos identificados no projeto, segundo a EAP do projeto e a EAR apresentada, estão listados a seguir.

\begin{description}
	\item[Projeto Vaga Livre] \hfill
	\begin{enumerate}[label=\arabic*.]
		\item \textbf{Gerenciamento do projeto}
		      \begin{enumerate}[label*=\arabic*.]
			      \item Recusa do patrocinador em aprovar o termo de abertura.
			      \item Atraso nas entregas dos planos ou da linha de base do escopo.
			      \item Planos e linha de base do escopo não aprovados.
			      \item Termo de encerramento não aprovado.
			      \item Indisponibilidade do patrocinador para as reuniões e acompanhamento do \foreign{status} do projeto.
			      \item Férias de algum membro da equipe.
			      \item Indisponibilidade de membro da equipe devido a doença.
			      \item Estouro do orçamento.
			      \item Atrasos nos prazos de execução do projeto.
			      \item Mudanças de governo, ocasionando reformas que influenciam nos planos de projeto.
			      \item Controles e grenciamento do projeto inadequados ou insuficientes.
			      \item Problemas de comunicação entre a equipe e o gerente do projeto.
			      \item Mudança constante de requisitos.
			      \item Corte no orçamento.
			      \item Cancelamento ou suspensão do projeto.
			      \item Plano de comunicação débil.
			      \item Responsabilidades da equipe não delineadas.
		      \end{enumerate}
		\item \textbf{Infraestrutura}
		      \begin{enumerate}[label*=\arabic*.]
			      \item Identificação de novas aquisições necessárias para os ambientes de desenvolvimento, testes ou produção, ocasionando gastos não previstos na linha de base dos custos.
			      \item Atraso na contratação do servidor em nuvem.
			      \item Quebra de contrato do servidor em nuvem.
			      \item Aumento do custo do servidor em nuvem, visto que não é um contrato de preço fixo.
			      \item Indisponibilidade dos membros da equipe para a análise de necessidades de cada ambiente.
			      \item Ambiente de desenvolvimento insuficiente em relação às necessidades dos membros da equipe.
		      \end{enumerate}
		\item \textbf{Web API}
		      \begin{enumerate}[label*=\arabic*.]
			      \item Pouca disponibilidade de fornecedores preparados para oferecer o servidor em nuvem conforme requisitos desejados.
			      \item Custos com fornecedor do servidor em nuvem maior que o orçado.
			      \item Indisponibilidade de membros da equipe para suporte no levantamento requisitos para estrutura de banco de dados.
			      \item Grande quantidade de erros encontrados nos testes.
			      \item Atraso nos testes.
			      \item Reaproveitamento de código e ferramentas já desenvolvidas.
		      \end{enumerate}
		\item \textbf{Aplicativo do motorista}
		      \begin{enumerate}[label*=\arabic*.]
			      \item Performance do aplicativo do motorista insuficiente.
			      \item Grande quantidade de erros encontrados nos testes.
			      \item Atraso nos testes.
			      \item Reaproveitamento de código e ferramentas já desenvolvidas.
		      \end{enumerate}
		\item \textbf{Aplicativo do estacionamento}
		      \begin{enumerate}[label*=\arabic*.]
			      \item Performance do aplicativo do estacionamento insuficiente.
			      \item Grande quantidade de erros encontrados nos testes.
			      \item Atraso nos testes.
			      \item Indisponibilidade de documentação para desenvolver integração com DETRAN.
			      \item Integração com DETRAN muito complexa.
			      \item Integração com sistema de automação muito complexa.
			      \item Reaproveitamento de código e ferramentas já desenvolvidas.
		      \end{enumerate}
		\item \textbf{Auditoria da qualidade}
		      \begin{enumerate}[label*=\arabic*.]
			      \item Auditoria não representa corretamente os processos de qualidade aplicados pelo projeto.
		      \end{enumerate}
	\end{enumerate}
\end{description}

\section{Qualificação dos riscos}
\label{sec:risk-qualification}

Os riscos identificados serão classificados de acordo com sua probabilidade de ocorrência e impacto de seus resultados, conforme descrito abaixo:

\subsection{Probabilidade de ocorrência}

\begin{description}
	\item [Baixa] A probabilidade de o risco ocorrer é considerada pequena, menor que 25\%.
	\item [Média] Existe razoável probabilidade de o risco ocorrer, entre 25\% e 60\%.
	\item [Alta] O risco possui alta probabilidade de ocorrência, maior que 60\%.
\end{description}

\subsection{Impacto}

\begin{description}
	\item [Baixo] Caso o evento venha a ocorrer, o impacto no projeto é insignificante, podendo ser facilmente resolvido.
	\item [Médio] A ocorrência do evento de risco pode causar impacto relevante no projeto, sendo necessário um gerenciamento mais preciso, sob pena de prejudicar o resultado do projeto.
	\item [Alto] A ocorrência do evento de risco impacta de forma extremamente significativa no projeto, exigindo ação precisa e imediata por parte da equipe do projeto, os resultados estarão comprometidos.
\end{description}

As análises para os riscos classificados como oportunidade ou ameaça podem ser vistas nas figuras \ref{fig:probability-impact-matrix-oportunity} e \ref{fig:probability-impact-matrix-threats}, respectivamente.

\begin{figure}[h]
	\begin{tabularx}{\textwidth}{ c | >{\centering\arraybackslash}X >{\centering\arraybackslash}X >{\centering\arraybackslash}X l}
		\textbf{Probabilidade} &                                     &                                       &                                  &                  \\
		\cellcolor{red!30!}\textbf{Alta} &
		% Probabilidade Alta e Impacto Baixo
		\cellcolor{yellow!10!} &
		% Probabilidade Alta e Impacto Médio
		\cellcolor{red!10!} &
		% Probabilidade Alta e Impacto Alto
		\cellcolor{red!10!}3.6 &   \\
		\cellcolor{yellow!30!}\textbf{Média}&
		% Probabilidade Média e Impacto Baixo
		\cellcolor{green!10!} &
		% Probabilidade Média e Impacto Médio
		\cellcolor{yellow!10!} &
		% Probabilidade Média e Impacto Alto
		\cellcolor{red!10!} 4.4 &   \\
		\cellcolor{green!30!}\textbf{Baixa}&
		% Probabilidade Baixa e Impacto Baixo
		\cellcolor{green!10!} &
		% Probabilidade Baixa e Impacto Médio
		\cellcolor{green!10!} &
		% Probabilidade Baixa e Impacto Alto
		\cellcolor{yellow!10!} 5.7 &   \\
		\hline
		                       & \cellcolor{green!30!}\textbf{Baixo} & \cellcolor{yellow!30!}\textbf{Médio} & \cellcolor{red!30!}\textbf{Alto} & \textbf{Impacto} \\
	\end{tabularx}
	\caption{Matriz de probabilidade e impacto para oportunidades.}
	\label{fig:probability-impact-matrix-oportunity}
	\centering
\end{figure}

\begin{figure}[h]
	\begin{tabularx}{\textwidth}{ c | >{\centering\arraybackslash}X >{\centering\arraybackslash}X >{\centering\arraybackslash}X l}
		\textbf{Probabilidade} &                                     &                                       &                                  &                  \\
		\cellcolor{red!30!}\textbf{Alta} &
		% Probabilidade Alta e Impacto Baixo
		\cellcolor{yellow!10!} &
		% Probabilidade Alta e Impacto Médio
		\cellcolor{red!10!} 1.6, 5.6  &
		% Probabilidade Alta e Impacto Alto
		\cellcolor{red!10!} 1.12  &   \\
		\cellcolor{yellow!30!}\textbf{Média}&
		% Probabilidade Média e Impacto Baixo
		\cellcolor{green!10!} 2.6 &
		% Probabilidade Média e Impacto Médio
		\cellcolor{yellow!10!} 1.7, 1.9, 1.10, 1.16, 2.5, 3.3, 5.4, 5.5 &
		% Probabilidade Média e Impacto Alto
		\cellcolor{red!10!} 1.3, 1.5, 1.17, 2.1, 2.4, 3.4 &   \\
		\cellcolor{green!30!}\textbf{Baixa}&
		% Probabilidade Baixa e Impacto Baixo
		\cellcolor{green!10!} 2.2, 3.5 &
		% Probabilidade Baixa e Impacto Médio
		\cellcolor{green!10!} 1.11, 1.13, 2.3, 3.1, 3.2, 4.2, 4.3, 5.1, 5.2, 5.3, 6.1 &
		% Probabilidade Baixa e Impacto Alto
		\cellcolor{yellow!10!} 1.1, 1.2, 1.4, 1.8, 1.12, 1.14, 1.15, 4.1 &   \\
		\hline
		                       & \cellcolor{green!30!}\textbf{Baixo} & \cellcolor{yellow!30!}\textbf{Médio} & \cellcolor{red!30!}\textbf{Alto} & \textbf{Impacto} \\
	\end{tabularx}
	\caption{Matriz de probabilidade e impacto para ameaças.}
	\label{fig:probability-impact-matrix-threats}
	\centering
\end{figure}

\section{Quantificação dos riscos}
\label{sec:risk-quantification}

A quantificação dos riscos do projeto foi realizada utilizando a técnica do valor monetário esperado (VME), confrome pode ser visto na tabela \ref{tab:risk-answers}.

\newcounter{prob}
\newcounter{cost}
\newcounter{total}
\newcounter{parSum}

\begin{longtable}{ c p{0.35\textwidth} c c c }
	\toprule
	\thead[c]{\textbf{Item}} & \thead[c]{\textbf{Risco}}                                                                                                                                                     & \thead[c]{\textbf{Probabilidade}}    & \thead[c]{\textbf{Impacto}}                 & \thead[c]{\textbf{VME}}                                                                                                  \\
	\midrule
	\endhead
	\multicolumn{5}{c}{{\textit{Continua na próxima página.}}} \\
	\caption{Análise quantitativa dos riscos: VME.}
	\endfoot
	\endlastfoot

	1.1                      & Recusa do patrocinador em aprovar o termo de abertura.                                                                                                                        & \setcounter{prob}{15}\arabic{prob}\% & R\$ \setcounter{cost}{60*70*2055/10000}\arabic{cost},00 & R\$ \setcounter{total}{\value{prob}*\value{cost}/100}\arabic{total},00                                                   \\
	\midrule
	1.2                      & Atraso nas entregas dos planos ou da linha de base do escopo.                                                                                                                 & \setcounter{prob}{25}\arabic{prob}\% & R\$ \setcounter{cost}{60*65*11926/10000}\arabic{cost},00 & R\$ \setcounter{parSum}{\value{prob}*\value{cost}/100}\setcounter{total}{\value{total}+\value{parSum}}\arabic{parSum},00 \\
	\midrule
	1.3                      & Planos e linha de base do escopo não aprovados.                                                                                                                              & \setcounter{prob}{35}\arabic{prob}\% & R\$ \setcounter{cost}{60*75*11926/10000}\arabic{cost},00 & R\$ \setcounter{parSum}{\value{prob}*\value{cost}/100}\setcounter{total}{\value{total}+\value{parSum}}\arabic{parSum},00 \\
	\midrule
	1.4                      & Termo de encerramento não aprovado.                                                                                                                                          & \setcounter{prob}{10}\arabic{prob}\% & R\$ \setcounter{cost}{60*80*90000/10000}\arabic{cost},00 & R\$ \setcounter{parSum}{\value{prob}*\value{cost}/100}\setcounter{total}{\value{total}+\value{parSum}}\arabic{parSum},00 \\
	\midrule
	1.5                      & Indisponibilidade do patrocinador para as reuniões e acompanhamento do \foreign{status} do projeto.                                                                          & \setcounter{prob}{40}\arabic{prob}\% & R\$ \setcounter{cost}{60*75*19000/10000}\arabic{cost},00 & R\$ \setcounter{parSum}{\value{prob}*\value{cost}/100}\setcounter{total}{\value{total}+\value{parSum}}\arabic{parSum},00 \\
	\midrule
	1.6                      & Férias de algum membro da equipe.                                                                                                                                            & \setcounter{prob}{90}\arabic{prob}\% & R\$ \setcounter{cost}{60*35*7000/10000}\arabic{cost},00 & R\$ \setcounter{parSum}{\value{prob}*\value{cost}/100}\setcounter{total}{\value{total}+\value{parSum}}\arabic{parSum},00 \\
	\midrule
	1.7                      & Indisponibilidade de membro da equipe devido a doença.                                                                                                                       & \setcounter{prob}{60}\arabic{prob}\% & R\$ \setcounter{cost}{60*30*1500/10000}\arabic{cost},00 & R\$ \setcounter{parSum}{\value{prob}*\value{cost}/100}\setcounter{total}{\value{total}+\value{parSum}}\arabic{parSum},00 \\
	\midrule
	1.8                      & Estouro do orçamento.                                                                                                                                                        & \setcounter{prob}{15}\arabic{prob}\% & R\$ \setcounter{cost}{60*75*15000/10000}\arabic{cost},00 & R\$ \setcounter{parSum}{\value{prob}*\value{cost}/100}\setcounter{total}{\value{total}+\value{parSum}}\arabic{parSum},00 \\
	\midrule
	1.9                      & Atrasos nos prazos de execução do projeto.                                                                                                                                  & \setcounter{prob}{50}\arabic{prob}\% & R\$ \setcounter{cost}{60*50*10000/10000}\arabic{cost},00 & R\$ \setcounter{parSum}{\value{prob}*\value{cost}/100}\setcounter{total}{\value{total}+\value{parSum}}\arabic{parSum},00 \\
	\midrule
	1.10                     & Mudanças de governo, ocasionando reformas que influenciam nos planos de projeto.                                                                                             & \setcounter{prob}{60}\arabic{prob}\% & R\$ \setcounter{cost}{60*50*15000/10000}\arabic{cost},00 & R\$ \setcounter{parSum}{\value{prob}*\value{cost}/100}\setcounter{total}{\value{total}+\value{parSum}}\arabic{parSum},00 \\
	\midrule
	1.11                     & Controles e grenciamento do projeto inadequados ou insuficientes.                                                                                                             & \setcounter{prob}{10}\arabic{prob}\% & R\$ \setcounter{cost}{60*50*9456/10000}\arabic{cost},00 & R\$ \setcounter{parSum}{\value{prob}*\value{cost}/100}\setcounter{total}{\value{total}+\value{parSum}}\arabic{parSum},00 \\
	\midrule
	1.12                     & Problemas de comunicação entre a equipe e o gerente do projeto.                                                                                                             & \setcounter{prob}{70}\arabic{prob}\% & R\$ \setcounter{cost}{60*75*7500/10000}\arabic{cost},00 & R\$ \setcounter{parSum}{\value{prob}*\value{cost}/100}\setcounter{total}{\value{total}+\value{parSum}}\arabic{parSum},00 \\
	\midrule
	1.13                     & Mudança constante de requisitos.                                                                                                                                             & \setcounter{prob}{20}\arabic{prob}\% & R\$ \setcounter{cost}{60*50*8500/10000}\arabic{cost},00 & R\$ \setcounter{parSum}{\value{prob}*\value{cost}/100}\setcounter{total}{\value{total}+\value{parSum}}\arabic{parSum},00 \\
	\midrule
	1.14                     & Corte no orçamento.                                                                                                                                                          & \setcounter{prob}{15}\arabic{prob}\% & R\$ \setcounter{cost}{60*70*2350/10000}\arabic{cost},00 & R\$ \setcounter{parSum}{\value{prob}*\value{cost}/100}\setcounter{total}{\value{total}+\value{parSum}}\arabic{parSum},00 \\
	\midrule
	1.15                     & Cancelamento ou suspensão do projeto.                                                                                                                                        & \setcounter{prob}{3}\arabic{prob}\% & R\$ \setcounter{cost}{60*100*10000/10000}\arabic{cost},00 & R\$ \setcounter{parSum}{\value{prob}*\value{cost}/100}\setcounter{total}{\value{total}+\value{parSum}}\arabic{parSum},00 \\
	\midrule
	1.16                     & Plano de comunicação débil.                                                                                                                                                & \setcounter{prob}{40}\arabic{prob}\% & R\$ \setcounter{cost}{60*60*8345/10000}\arabic{cost},00 & R\$ \setcounter{parSum}{\value{prob}*\value{cost}/100}\setcounter{total}{\value{total}+\value{parSum}}\arabic{parSum},00 \\
	\midrule
	1.17                     & Responsabilidades da equipe não delineadas.                                                                                                                                  & \setcounter{prob}{25}\arabic{prob}\% & R\$ \setcounter{cost}{60*70*9250/10000}\arabic{cost},00 & R\$ \setcounter{parSum}{\value{prob}*\value{cost}/100}\setcounter{total}{\value{total}+\value{parSum}}\arabic{parSum},00 \\
	\midrule
	2.1                      & Identificação de novas aquisições necessárias para os ambientes de desenvolvimento, testes ou produção, ocasionando gastos não previstos na linha de base dos custos. & \setcounter{prob}{30}\arabic{prob}\% & R\$ \setcounter{cost}{60*75*10000/10000}\arabic{cost},00 & - R\$ \setcounter{parSum}{\value{prob}*\value{cost}/100}\setcounter{total}{\value{total}+\value{parSum}}\arabic{parSum},00 \\
	\midrule
	2.2                      & Atraso na contratação do servidor em nuvem.                                                                                                                                  & \setcounter{prob}{20}\arabic{prob}\% & R\$ \setcounter{cost}{60*20*9500/10000}\arabic{cost},00 & - R\$ \setcounter{parSum}{\value{prob}*\value{cost}/100}\setcounter{total}{\value{total}+\value{parSum}}\arabic{parSum},00 \\
	\midrule
	2.3                      & Quebra de contrato do servidor em nuvem.                                                                                                                                      & \setcounter{prob}{5}\arabic{prob}\% & R\$ \setcounter{cost}{60*50*9750/10000}\arabic{cost},00 & - R\$ \setcounter{parSum}{\value{prob}*\value{cost}/100}\setcounter{total}{\value{total}+\value{parSum}}\arabic{parSum},00 \\
	\midrule
	2.4                      & Aumento do custo do servidor em nuvem, visto que não é um contrato de preço fixo.                                                                                          & \setcounter{prob}{30}\arabic{prob}\% & R\$ \setcounter{cost}{60*70*8500/10000}\arabic{cost},00 & - R\$ \setcounter{parSum}{\value{prob}*\value{cost}/100}\setcounter{total}{\value{total}+\value{parSum}}\arabic{parSum},00 \\
	\midrule
	2.5                      & Indisponibilidade dos membros da equipe para a análise de necessidades de cada ambiente.                                                                                     & \setcounter{prob}{30}\arabic{prob}\% & R\$ \setcounter{cost}{60*65*9650/10000}\arabic{cost},00 & - R\$ \setcounter{parSum}{\value{prob}*\value{cost}/100}\setcounter{total}{\value{total}+\value{parSum}}\arabic{parSum},00 \\
	\midrule
	2.6                      & Ambiente de desenvolvimento insuficiente em relação às necessidades dos membros da equipe.                                                                                 & \setcounter{prob}{40}\arabic{prob}\% & - R\$ \setcounter{cost}{60*15*10000/10000}\arabic{cost},00 & R\$ \setcounter{parSum}{\value{prob}*\value{cost}/100}\setcounter{total}{\value{total}+\value{parSum}}\arabic{parSum},00 \\
	\midrule
	3.1                      & Pouca disponibilidade de fornecedores preparados para oferecer o servidor em nuvem conforme requisitos desejados.                                                             & \setcounter{prob}{20}\arabic{prob}\% & R\$ \setcounter{cost}{60*40*9625/10000}\arabic{cost},00 & - R\$ \setcounter{parSum}{\value{prob}*\value{cost}/100}\setcounter{total}{\value{total}+\value{parSum}}\arabic{parSum},00 \\
	\midrule
	3.2                      & Custos com fornecedor do servidor em nuvem maior que o orçado.                                                                                                               & \setcounter{prob}{5}\arabic{prob}\% & R\$ \setcounter{cost}{60*50*9900/10000}\arabic{cost},00 & - R\$ \setcounter{parSum}{\value{prob}*\value{cost}/100}\setcounter{total}{\value{total}+\value{parSum}}\arabic{parSum},00 \\
	\midrule
	3.3                      & Indisponibilidade de membros da equipe para suporte no levantamento requisitos para estrutura de banco de dados.                                                              & \setcounter{prob}{30}\arabic{prob}\% & R\$ \setcounter{cost}{60*60*8900/10000}\arabic{cost},00 & - R\$ \setcounter{parSum}{\value{prob}*\value{cost}/100}\setcounter{total}{\value{total}+\value{parSum}}\arabic{parSum},00 \\
	\midrule
	3.4                      & Grande quantidade de erros encontrados nos testes.                                                                                                                            & \setcounter{prob}{40}\arabic{prob}\% & R\$ \setcounter{cost}{60*70*9400/10000}\arabic{cost},00 & - R\$ \setcounter{parSum}{\value{prob}*\value{cost}/100}\setcounter{total}{\value{total}+\value{parSum}}\arabic{parSum},00 \\
	\midrule
	3.5                      & Atraso nos testes.                                                                                                                                                            & \setcounter{prob}{15}\arabic{prob}\% & R\$ \setcounter{cost}{60*15*7950/10000}\arabic{cost},00 & - R\$ \setcounter{parSum}{\value{prob}*\value{cost}/100}\setcounter{total}{\value{total}+\value{parSum}}\arabic{parSum},00 \\
	\midrule
	3.6                      & Reaproveitamento de código e ferramentas já desenvolvidas.                                                                                                                  & \setcounter{prob}{70}\arabic{prob}\% & R\$ \setcounter{cost}{60*70*16494/10000}\arabic{cost},00 & + R\$ \setcounter{parSum}{\value{prob}*\value{cost}/100}\setcounter{total}{\value{total}-\value{parSum}}\arabic{parSum},00 \\
	\midrule
	4.1                      & Performance do aplicativo do motorista insuficiente.                                                                                                                          & \setcounter{prob}{15}\arabic{prob}\% & R\$ \setcounter{cost}{60*80*20000/10000}\arabic{cost},00 & - R\$ \setcounter{parSum}{\value{prob}*\value{cost}/100}\setcounter{total}{\value{total}+\value{parSum}}\arabic{parSum},00 \\
	\midrule
	4.2                      & Grande quantidade de erros encontrados nos testes.                                                                                                                            & \setcounter{prob}{20}\arabic{prob}\% & R\$ \setcounter{cost}{60*50*9400/10000}\arabic{cost},00 & - R\$ \setcounter{parSum}{\value{prob}*\value{cost}/100}\setcounter{total}{\value{total}+\value{parSum}}\arabic{parSum},00 \\
	\midrule
	4.3                      & Atraso nos testes.                                                                                                                                                            & \setcounter{prob}{15}\arabic{prob}\% & R\$ \setcounter{cost}{60*55*7950/10000}\arabic{cost},00 & - R\$ \setcounter{parSum}{\value{prob}*\value{cost}/100}\setcounter{total}{\value{total}+\value{parSum}}\arabic{parSum},00 \\
	\midrule
	4.4                      & Reaproveitamento de código e ferramentas já desenvolvidas.                                                                                                                  & \setcounter{prob}{50}\arabic{prob}\% & R\$ \setcounter{cost}{60*70*25590/10000}\arabic{cost},00 & - R\$ \setcounter{parSum}{\value{prob}*\value{cost}/100}\setcounter{total}{\value{total}-\value{parSum}}\arabic{parSum},00 \\
	\midrule
	5.1                      & Performance do aplicativo do estacionamento insuficiente.                                                                                                                     & \setcounter{prob}{18}\arabic{prob}\% & R\$ \setcounter{cost}{60*50*20000/10000}\arabic{cost},00 & - R\$ \setcounter{parSum}{\value{prob}*\value{cost}/100}\setcounter{total}{\value{total}+\value{parSum}}\arabic{parSum},00 \\
	\midrule
	5.2                      & Grande quantidade de erros encontrados nos testes.                                                                                                                            & \setcounter{prob}{20}\arabic{prob}\% & R\$ \setcounter{cost}{60*40*9400/10000}\arabic{cost},00 & - R\$ \setcounter{parSum}{\value{prob}*\value{cost}/100}\setcounter{total}{\value{total}+\value{parSum}}\arabic{parSum},00 \\
	\midrule
	5.3                      & Atraso nos testes.                                                                                                                                                            & \setcounter{prob}{15}\arabic{prob}\% & R\$ \setcounter{cost}{60*45*7950/10000}\arabic{cost},00 & - R\$ \setcounter{parSum}{\value{prob}*\value{cost}/100}\setcounter{total}{\value{total}+\value{parSum}}\arabic{parSum},00 \\
	\midrule
	5.4                      & Indisponibilidade de documentação para desenvolver integração com DETRAN.                                                                                                 & \setcounter{prob}{40}\arabic{prob}\% & R\$ \setcounter{cost}{60*50*9950/10000}\arabic{cost},00 & - R\$ \setcounter{parSum}{\value{prob}*\value{cost}/100}\setcounter{total}{\value{total}+\value{parSum}}\arabic{parSum},00 \\
	\midrule
	5.5                      & Integração com DETRAN muito complexa.                                                                                                                                       & \setcounter{prob}{40}\arabic{prob}\% & R\$ \setcounter{cost}{60*50*9600/10000}\arabic{cost},00 & - R\$ \setcounter{parSum}{\value{prob}*\value{cost}/100}\setcounter{total}{\value{total}+\value{parSum}}\arabic{parSum},00 \\
	\midrule
	5.6                      & Integração com sistema de automação muito complexa.                                                                                                                       & \setcounter{prob}{65}\arabic{prob}\% & R\$ \setcounter{cost}{60*40*9230/10000}\arabic{cost},00 & - R\$ \setcounter{parSum}{\value{prob}*\value{cost}/100}\setcounter{total}{\value{total}+\value{parSum}}\arabic{parSum},00 \\
	\midrule
	5.7                      & Reaproveitamento de código e ferramentas já desenvolvidas.                                                                                                                  & \setcounter{prob}{20}\arabic{prob}\% & R\$ \setcounter{cost}{60*70*39619/10000}\arabic{cost},00 & R\$ \setcounter{parSum}{\value{prob}*\value{cost}/100}\setcounter{total}{\value{total}-\value{parSum}}\arabic{parSum},00 \\
	\midrule
	6.1                      & Auditoria não representa corretamente os processos de qualidade aplicados pelo projeto.                                                                                      & \setcounter{prob}{10}\arabic{prob}\% & R\$ \setcounter{cost}{60*50*7250/10000}\arabic{cost},00 & - R\$ \setcounter{parSum}{\value{prob}*\value{cost}/100}\setcounter{total}{\value{total}+\value{parSum}}\arabic{parSum},00 \\
	\bottomrule
	                         &                                                                                                                                                                               &                                      & \thead[r]{\textbf{Total:}}                  & \textbf{- R\$ \arabic{total},00}                                                                                                    \\
	\bottomrule
	\caption{Análise quantitativa dos riscos: VME.}
	\centering
	\label{tab:risk-answers}
\end{longtable}

\section{Sistema de controle de mudanças de riscos}
\label{sec:risk-change-control-system}

Toda identificação de novos riscos ou alterações em riscos já identificados devem ser incluídas na lista de riscos identificados (ver seção \ref{sec:identified-risks}), em seguida deve ser atualizada a qualificação dos riscos (ver seção \ref{sec:risk-qualification}), a resposta aos riscos (ver seção \ref{risk-answers}) e a análise de VME (ver seção \ref{sec:risk-quantification}). Esta análise deve ser apresentada durante a reunião semanal do CCM, em conjunto com os detalhes sobre os riscos atualizados ou incluídos.


%\begin{figure}[h]
%	\centering
%	\begin{tikzpicture}[node distance = 0.3cm and 0.7cm, auto]
%
%		\node (begin) [startstop] {Início};
%
%		\node (end) [startstop] {Fim};
%
%	\end{tikzpicture}
%	\caption{Sistema de controle de mudanças de riscos.}
%	\label{fig:risk-change-control-system}
%\end{figure}

\section{Respostas planejadas aos riscos}
\label{risk-answers}

Para os riscos identificados e qualificados, optou-se por diferentes estratégias de resposta para cada necessidade, conforme pode ser visto na tabela \ref{tab:risk-answers}.

\begin{longtable}{ c p{0.4\textwidth} c p{0.35\textwidth} }
	\toprule
	\thead[c]{\textbf{Item}} & \thead[c]{\textbf{Risco}}                                                                                                                                                     & \thead[c]{\textbf{Ação}} & \thead[c]{\textbf{Descrição da Ação}}                                                                                              \\
	\midrule
	\endhead
	\multicolumn{4}{c}{{\textit{Continua na próxima página.}}} \\
	\caption{Respostas planejadas aos riscos.}
	\endfoot
	\endlastfoot
	1.1                      & Recusa do patrocinador em aprovar o termo de abertura.                                                                                                                        & Mitigar                    & Alinhar, periodicamente, as informações contidas no plano com o patrocinador.                                                        \\
	\midrule
	1.2                      & Atraso nas entregas dos planos ou da linha de base do escopo.                                                                                                                 & Aceitar                    & Risco não será respondido e verba de contingência será utilizada em caso de necessidade.                                           \\
	\midrule
	1.3                      & Planos e linha de base do escopo não aprovados.                                                                                                                              & Aceitar                    & Risco não será respondido e verba de contingência será utilizada em caso de necessidade.                                           \\
	\midrule
	1.4                      & Termo de encerramento não aprovado.                                                                                                                                          & Mitigar                    & Incentivar a cultura de qualidade durante toda execução do projeto. Definir claramente critérios de aceitação.                    \\
	\midrule
	1.5                      & Indisponibilidade do patrocinador para as reuniões e acompanhamento do \foreign{status} do projeto.                                                                          & Mitigar                    & Alinhar com patrocinador sua dispoibilidade no projeto.                                                                                \\
	\midrule
	1.6                      & Férias de algum membro da equipe.                                                                                                                                            & Mitigar                    & Buscar definir as férias dos recursos o quanto antes.                                                                                 \\
	\midrule
	1.7                      & Indisponibilidade de membro da equipe devido a doença.                                                                                                                       & Aceitar                    & Risco não será respondido e verba de contingência será utilizada em caso de necessidade.                                           \\
	\midrule
	1.8                      & Estouro do orçamento.                                                                                                                                                        & Mitigar                    & Acompanhamento diário do orçamento.                                                                                                  \\
	\midrule
	1.9                      & Atrasos nos prazos de execução do projeto.                                                                                                                                  & Aceitar                    & Risco não será respondido e verba de contingência será utilizada em caso de necessidade.                                           \\
	\midrule
	1.10                     & Mudanças de governo, ocasionando reformas que influenciam nos planos de projeto.                                                                                             & Aceitar                    & Risco não será respondido e verba de contingência será utilizada em caso de necessidade.                                           \\
	\midrule
	1.11                     & Controles e grenciamento do projeto inadequados ou insuficientes.                                                                                                             & Mitigar                    & Periodicamente, alinhar com a equipe os questionamentos relacionados ao gerenciamento e monitoramento do projeto.                      \\
	\midrule
	1.12                     & Problemas de comunicação entre a equipe e o gerente do projeto.                                                                                                             & Mitigar                    & Diariamente, o gerente do projeto deve buscar saber os obstáculo encontrados pela equipe.                                             \\
	\midrule
	1.13                     & Mudança constante de requisitos.                                                                                                                                             & Mitigar                    & Garantir que o plano integrado de mudanças trate destes casos de forma adequada, prevendo as mudanças de custo e prazo necessárias. \\
	\midrule
	1.14                     & Corte no orçamento.                                                                                                                                                          & Aceitar                    & Risco não será respondido e verba de contingência será utilizada em caso de necessidade.                                           \\
	\midrule
	1.15                     & Cancelamento ou suspensão do projeto.                                                                                                                                        & Aceitar                    & Risco não será respondido e verba de contingência será utilizada em caso de necessidade.                                           \\
	\midrule
	1.16                     & Plano de comunicação débil.                                                                                                                                                & Mitigar                    & Verificar periodicamente o plano de gerenciamento das comunicações.                                                                  \\
	\midrule
	1.17                     & Responsabilidades da equipe não delineadas.                                                                                                                                  & Mitigar                    & Durante a reunião de kick-off apresentar os membros da equipe e suas responsabilidades.                                               \\
	\midrule
	2.1                      & Identificação de novas aquisições necessárias para os ambientes de desenvolvimento, testes ou produção, ocasionando gastos não previstos na linha de base dos custos. & Mitigar                    & Gerente do projeto deve alinhar com todos membros da equipe as aquisições necessárias no início do projeto.                        \\
	\midrule
	2.2                      & Atraso na contratação do servidor em nuvem.                                                                                                                                 & Aceitar                    & Risco não será respondido e verba de contingência será utilizada em caso de necessidade.                                           \\
	\midrule
	2.3                      & Quebra de contrato do servidor em nuvem.                                                                                                                                      & Aceitar                    & Risco não será respondido e verba de contingência será utilizada em caso de necessidade.                                           \\
	\midrule
	2.4                      & Aumento do custo do servidor em nuvem, visto que não é um contrato de preço fixo.                                                                                          & Aceitar                    & Risco não será respondido e verba de contingência será utilizada em caso de necessidade.                                           \\
	\midrule
	2.5                      & Indisponibilidade dos membros da equipe para a análise de necessidades de cada ambiente.                                                                                     & Mitigar                    & Gerente do projeto deve verificar agenda dos recursos ao criar o cronograma.                                                           \\
	\midrule
	2.6                      & Ambiente de desenvolvimento insuficiente em relação às necessidades dos membros da equipe.                                                                                 & Aceitar                    & Risco não será respondido e verba de contingência será utilizada em caso de necessidade.                                           \\
	\midrule
	3.1                      & Pouca disponibilidade de fornecedores preparados para oferecer o servidor em nuvem conforme requisitos desejados.                                                             & Aceitar                    & Risco não será respondido e verba de contingência será utilizada em caso de necessidade.                                           \\
	\midrule
	3.2                      & Custos com fornecedor do servidor em nuvem maior que o orçado.                                                                                                               & Aceitar                    & Risco não será respondido e verba de contingência será utilizada em caso de necessidade.                                           \\
	\midrule
	3.3                      & Indisponibilidade de membros da equipe para suporte no levantamento de requisitos para estrutura de banco de dados.                                                           & Mitigar                    & Gerente do projeto deve verificar agenda dos recursos no início do projeto e criar o cronograma de acordo.                            \\
	\midrule
	3.4                      & Grande quantidade de erros encontrados nos testes.                                                                                                                            & Mitigar                    & Incentivar a cultura de qualidade no processo de desenvolvimento.                                                                      \\
	\midrule
	3.5                      & Atraso nos testes.                                                                                                                                                            & Aceitar                    & Risco não será respondido e verba de contingência será utilizada em caso de necessidade.                                           \\
	\midrule
	3.6                      & Reaproveitamento de código e ferramentas já desenvolvidas.                                                                                                                  & Explorar                   & Incentivar a cultura de código reaproveitável na equipe.                                                                             \\
	\midrule
	4.1                      & Performance do aplicativo do motorista insuficiente.                                                                                                                          & Mitigar                    & Revisar código periodicamente em busca de gargalos de performance.                                                                    \\
	\midrule
	4.2                      & Grande quantidade de erros encontrados nos testes.                                                                                                                            & Mitigar                    & Incentivar a cultura de qualidade no processo de desenvolvimento.                                                                      \\
	\midrule
	4.3                      & Atraso nos testes.                                                                                                                                                            & Aceitar                    & Risco não será respondido e verba de contingência será utilizada em caso de necessidade.                                           \\
	\midrule
	4.4                      & Reaproveitamento de código e ferramentas já desenvolvidas.                                                                                                                  & Explorar                   & Incentivar a cultura de código reaproveitável na equipe.                                                                             \\
	\midrule
	5.1                      & Performance do aplicativo do estacionamento insuficiente.                                                                                                                     & Aceitar                    & Risco não será respondido e verba de contingência será utilizada em caso de necessidade.                                           \\
	\midrule
	5.2                      & Grande quantidade de erros encontrados nos testes.                                                                                                                            & Mitigar                    & Incentivar a cultura de qualidade no processo de desenvolvimento.                                                                      \\
	\midrule
	5.3                      & Atraso nos testes.                                                                                                                                                            & Aceitar                    & Risco não será respondido e verba de contingência será utilizada em caso de necessidade.                                           \\
	\midrule
	5.4                      & Indisponibilidade de documentação para desenvolver integração com DETRAN.                                                                                                 & Mitigar                    & Entrar em contato com DETRAN o quanto antes em busca de documentação para API.                                                       \\
	\midrule
	5.5                      & Integração com DETRAN muito complexa.                                                                                                                                       & Mitigar                    & Realizar a integração o mais cedo possível.                                                                                         \\
	\midrule
	5.6                      & Integração com sistema de automação muito complexa.                                                                                                                       & Mitigar                    & Realizar a integração o mais cedo possível.                                                                                         \\
	\midrule
	5.7                      & Reaproveitamento de código e ferramentas já desenvolvidas.                                                                                                                  & Explorar                   & Incentivar a cultura de código reaproveitável na equipe.                                                                             \\
	\midrule
	6.1                      & Auditoria não representa corretamente os processos de qualidade aplicados pelo projeto.                                                                                      & Mitigar                    & Reavaliar a lista de verificação de qualidade após cada processo de auditoria.                                                      \\
	\bottomrule
	\caption{Respostas planejadas aos riscos.}
	\centering
	\label{tab:risk-answers}
\end{longtable}

\section{Reservas de contingência}

Conforme descrito no plano de gerenciamento de custos (ver capítulo \ref{ch:cost-management-plan}), as reservas de contingências estão reservadas para uso exclusivo do processo de gerenciamento de riscos, para os eventos de riscos aceitos ou mitigados.


As reservas de contingência totalizam \contingencyBudget{}, o gerente de projeto possui as autonomias descritas no plano de gerenciamento de custos para utilização das reservas.
\section{Frequência de avaliação dos riscos do projeto}

Os riscos identificados no projeto devem ser avaliados durante a reunião do CCM, previsto no plano de gerenciamento das comunicações (ver capítulo \ref{ch:communication-management-plan}).

\section{Alocação financeira para o gerenciamento de riscos}

\section{Administração do plano de gerenciamento de riscos}

\subsection{Responsável}

\begin{itemize}
	\item \projectManagerName{}, gerente de projeto, será o responsável direto pelo plano de gerenciamento de riscos.
\end{itemize}

\section{Outros assuntos relacionados ao gerenciamento de riscos do projeto não previstos neste plano}

Solicitações não previstas neste plano deverão passar pela aprovação do CCM. Após aprovada o plano deve ser atualizado pelo gerente do projeto.

\section{Controle de Versão}

\begin{table}[H]
	\begin{tabularx}{\textwidth}{| c | c | X | X |}
		\hline
		\textbf{Versão} & \textbf{Data} & \textbf{Autor}        & \textbf{Notas de Revisão} \\
		\hline
		1                &    23/04/2017           & \projectManagerName{} & Criação do documento     \\
		\hline
		2                &    27/04/2017           & \projectManagerName{} & Identificação dos riscos     \\
		\hline
		3                &    02/05/2017           & \projectManagerName{} & Adicionando novos riscos     \\
		\hline
		4                &    05/05/2017           & \projectManagerName{} & Realizando qualificação dos riscos     \\
		\hline
		5                &    17/05/2017           & \projectManagerName{} & Realizando quantificação dos riscos      \\
		\hline
		6                &    24/05/2017           & \projectManagerName{} & Planejando respostas aos riscos     \\
		\hline
		7                &    05/06/2017           & \projectManagerName{} & Revisão do documento     \\
		\hline
	\end{tabularx}
	\centering
\end{table}

\section{Aprovações}

\begin{table}[H]
	\begin{tabularx}{\textwidth}{| c | c | X | c |}
		\hline
		\textbf{Função}  & \textbf{Nome}         & \textbf{Assinatura}        & \textbf{Data} \\
		\hline
		Patrocinador       & \projectSponsorName{} & \projectSponsorSignature{} &      06/06/2017         \\
		\hline
		Gerente de projeto & \projectManagerName{} & \projectManagerSignature{} &      06/06/2017         \\
		\hline
	\end{tabularx}
	\centering
\end{table}