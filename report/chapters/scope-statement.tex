%---Escopo
%Descrição dos processos de definição do escopo com base nos requisitos coletados
%Priorização dos requisitos
%Apresentação da declaração de escopo do projeto, identificação dos itens fora do escopo do projeto
%Descrição das premissas e restrições do projeto
%Matriz de rastreabilidade de requisitos
%---

\chapter{Declaração de escopo}

\section{Equipe do projeto}

\begin{itemize}
	\item Gerente de projeto: \projectManagerName.
	\item Desenvolvedor Mobile:
	\item Desenvolvedor Web Front-End:
	\item Desenvolvedor Web Back-End:
	\item Engenheiro de software:
	\item Arquiteto de solução:
	\item Arquiteto de software:
	\item Engenheiro eletricista:
	\item Patrocinador: \projectSponsorName.
	\item Analista de Testes:
\end{itemize}

% Já estão presentes no termo de abertura, verificar necessidade de estarem aqui também
%\section{Descrição do projeto}
%\section{Objetivo do projeto}
%\section{Justificativa do projeto}

\section{Documentação dos requisitos}

O escopo deste projeto inclui a especificação de duas soluções de software. A primeira prevista através de um aplicativo para smartphone focado no cliente de estacionamentos privados, chamaremos esta solução de \userMobileAppName. A segundo solução prevê a criação de um aplicativo web focado no uso do dono ou gerente de estacionamentos privados, referenciaremos esta solução pelo nome \parkingWebAppName.

Além das soluções de software entra no escopo deste projeto a especificação de dois tipos de projetos de implantação do sistema de sensores.

A seguir estão especificados os requisitos que deverão ser implementados para cada uma destas soluções.

%\subsection{Requisitos de negócio}

%\subsection{Requisitos das partes interessadas}

\subsection{Requisitos de solução}

\subsubsection{Requisitos funcionais}

\todo[inline,color=orange]{Revisar requisitos funcionais.}

\begin{itemize}
		% Android App
	\item Qualquer pessoa deve poder criar um conta usando o aplicativo \projectName. Para se registrar o consumidor deve entrar com seu endereço de e-mail, uma senha, o primeiro e último nomes, a placa do veículo e um cartão de crédito válido antes que possam fazer qualquer reserva.
	\item Um consumidor registrado pode realizar reservas nos estacionamentos listados pelo aplicativo, contanto que realize o pagamento de forma antecipada.
	\item Um consumidor registrado deve ocupar sua vaga no estacionamento para que uma reserva seja considerada realizada.
	\item Uma câmera deve reconhecer a placa do carro na entrada, e entregar um bilhete contendo a identificação da vaga disponibilizada.
	\item Um consumidor pode visualizar se existem vagas nos estacionamentos cadastrados.
	\item O usuário deve ter a possibilidade de verificar o dinheiro gasto até o momento.
	\item O sistema deve ser capaz de processar pagamentos.
	\item O consumidor pode cancelar sua reserva em até 1 hora antes do início da mesma.
	    % Web app
	\item O sistema deve permitir ao gerente ou dono do estacionamento visualizar o perfil do cliente.
	\item O sistema deve permitir a criação de um cadastro do estacionamento através de um registro.
	\item O sistema deve permitir a definição de usuários com permissão para acessar as telas de gerenciamento do estacionamento.
	\item O sistema deve processar o pagamento baseado no tempo em que o usuário ficou estacionado.
	\item O sistema deve oferecer opções para gerenciar o estacionamento: definir preço por hora ou mês, multas entre outras definições sobre as características do mesmo.
	\item O sistema deve identificar tendências em relação a ocupação do estacionamento.
	\item %O sistema deve possuir integração com aplicativos web e mobile.
	\item O sistema deve estar integrado com o banco de dados do detran para buscar informações de licença do motorista.
	\item O administrador do sistema pode verificar a ocupação das vagas no estacionamento.
	    % Sistema de sensores
	\item O sistema deve escanear e reconhecer placas veiculares.
	\item O sistema deve reconhecer usuários cadastrados através da placa do veículo.
	\item O sistema de sensores deve oferecer um software que centralize as informações e as envie de forma segura para a nuvem.
	\item Caso o estacionamento esteja sem acesso a internet deve existir a capacidade de guardar as informações e enviá-las para a nuvem assim que o acesso for reestabelecido.
	\item Caso o sistema de sensores identifique um usuário que já tenha realizado o pagamento é preciso abrir a cancela sem qualquer interação com o usuário.
	\item Problemas na identificação da placa podem ocorrer, nestes casos é preciso oferecer um sistema para interação manual.
	\item Caso um cliente registrado chegue ao estacionamento e não existam vagas disponíveis o sistema deve avisar o mesmo, em caso de reservas realizadas que não poderão ser realizadas o sistema deve devolver o dinheiro para o cartão de crédito do mesmo.
    \item Deve existir um sistema de \foreign{overbooking} de modo a permitir a máxima ocupação do estacionamento.
\end{itemize}

\subsubsection{Requisitos não funcionais}

\todo[inline,color=orange]{Criar requisitos não funcionais.}

%\subsubsection{Requisitos tecnológicos e de conformidade com padrões}

%\subsubsection{Requisitos de suporte e treinamento}

%\subsubsection{Requisitos de qualidade}

%\subsection{Requisitos do projeto}

%\subsubsection{Níveis de serviço, desempenho, segurança, conformidade, etc...}

%\subsubsection{Critérios de aceitação}

\subsection{Premissas}

\begin{itemize}
	\item O cliente irá manter o registro de seu perfil, incluindo a placa do veículo, devidamente atualizados.
    \item O sistema de leitura de placas irá sempre identificar de maneira correta a placa dos veículo, retornando erro caso a mesma não possa ser identificada, o que pode acontecer devido a sujeira ou falta de placa.
    \item Em casos que o sistema não possa acomodar o usuário ou o usuário não concorde com os termos de uso e condições do sistema, o usuário será aconselhado a deixar o estacionamento. Nestes casos assume-se que o usuário sempre obedecerá e sairá do estacionamento.
    \item Um consumidor estaciona sempre na vaga correta, e nunca ocupará duas vagas ao mesmo tempo.
    \item Os sensores de vagas sempre funcionarão de maneira correta. Os sensores reconhecerão uma vaga como ocupada apenas quando houver um carro na vaga.
    \item O usuário sairá do estacionamento imediatamente após retirar seu carro da vaga utilizada.
    % From project-charter
    \item A equipe está motivida para trabalhar no projeto.
	\item Os estacionamentos privados de Porto Alegre estão prontos para aderir a este tipo de tecnologia.
	\item Utilizar base de dados em cloud.
	\item Utilizar um framework multi-plataforma para o desenvolvimento do aplicativo para dispositivos móveis.
	\item Disponibilidade do patrocinador: durante o planejamento no mínimo 40\% do tempo, e durante a execução no mínimo 80\% do tempo.
\end{itemize}

\subsection{Restrições}

\begin{itemize}
	\item O orçamento é limitado a \maximumBudget.
	\item O modo de funcionamento básico do sistema não deve depender de qualquer tipo de tecnologia de ponta. \todo[inline,color=orange]{Verificar se esta restrição é plausível.}
	\item Data de conclusão do projeto: \maximumDeadline.
\end{itemize}

\subsection{Limites do projeto e exclusões específicas}

\begin{itemize}
\item O projeto não tem como objetivo a implantação da solução, apenas a criação de especificações técnicas que possibilitem a implantação do sistema.
\item Questões de marketing e divulgação do produto não serão tratados neste projeto.
\end{itemize}

\section{Estrutura Analítica do Projeto}

Verificar apêndice \ref{ch:wbs}.

\section{Dicionário da EAP}

Verificar apêndice \ref{ch:wbs-dictionary}.

\section{Orçamento do projeto}

O orçamento previsto pelo projeto é de \maximumBudget, incluindo reservas gerenciais. Despesas com pessoal e recursos internos estão incluídas dentro deste orçamento.

\section{Principais entregas e critérios de aceitação}

\todo[inline,color=orange]{Criar principais entregas.}

\section{Controle de Versão}

\begin{table}[H]
	\begin{tabularx}{\textwidth}{| c | c | X | X |}
		\hline
		\textbf{Versão} & \textbf{Data} & \textbf{Autor}      & \textbf{Notas de Revisão} \\
		\hline
		1                &               & \projectManagerName & Criação do documento     \\
		\hline
	\end{tabularx}
	\centering
\end{table}

\section{Aprovações}

\begin{table}[H]
	\begin{tabularx}{\textwidth}{| c | c | X | c |}
		\hline
		\textbf{Função}  & \textbf{Nome}       & \textbf{Assinatura}      & \textbf{Data} \\
        \hline
		Patrocinador       & \projectSponsorName & \projectSponsorSignature &               \\
		\hline
		Gerente de projeto & \projectManagerName & \projectManagerSignature &               \\
		\hline
	\end{tabularx}
	\centering
\end{table}