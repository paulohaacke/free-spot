%---Escopo
%Descrição dos processos de definição do escopo com base nos requisitos coletados
%Priorização dos requisitos
%Apresentação da declaração de escopo do projeto, identificação dos itens fora do escopo do projeto
%Descrição das premissas e restrições do projeto
%Matriz de rastreabilidade de requisitos
%---

\chapter{Declaração de escopo}

\section{Equipe do projeto}

\begin{itemize}
	\item \projectManagerName{} - Gerente de projeto
	\item \mobDevOneName{} - Desenvolvedor mobile 1
	\item \mobDevTwoName{} - Desenvolvedor mobile 2
	\item \frontWebDevName{} - Desenvolvedor web front-end
	\item \backWebDevName{} - Desenvolvedor web back-end
	\item \softEngName{} - Engenheiro de software
	\item \softArcName{} - Arquiteto de software
	\item \testAnalOneName{} - Analista de Testes 1
	\item \testAnalTwoName{} - Analista de Testes 2
	\item \dbAnalName{} - Analista de Banco de Dados
	\item \projectSponsorName{} - Patrocinador
\end{itemize}

% Já estão presentes no termo de abertura, verificar necessidade de estarem aqui também
%\section{Descrição do projeto}
%\section{Objetivo do projeto}
%\section{Justificativa do projeto}

\section{Documentação dos requisitos}

O escopo deste projeto inclui a especificação de duas soluções de software. A primeira prevista através de um aplicativo para smartphone focado no cliente de estacionamentos privados, chamaremos esta solução de \userMobileAppName{}. A segundo solução prevê a criação de um aplicativo web focado no uso do dono ou gerente de estacionamentos privados, referenciaremos esta solução pelo nome \parkingWebAppName{}.

A seguir estão especificados os requisitos que deverão ser implementados para entregar esta solução.

\subsection{Requisitos de solução}

\subsubsection{Requisitos funcionais}

\begin{itemize}
	% Android App
	\item Qualquer pessoa deve poder criar um conta usando o aplicativo \projectName{}. Para se registrar o consumidor deve entrar com seu endereço de e-mail, uma senha, o primeiro e último nomes, a placa do veículo e um cartão de crédito válido antes que possa fazer qualquer reserva.
	\item Um consumidor registrado pode realizar reservas nos estacionamentos listados pelo aplicativo, contanto que realize o pagamento de forma antecipada.
	\item Um consumidor registrado deve ocupar sua vaga no estacionamento para que uma reserva seja considerada realizada.
	\item Um consumidor pode visualizar se existem vagas nos estacionamentos cadastrados.
	\item O usuário deve ter a possibilidade de verificar o dinheiro gasto até o momento.
	\item O sistema deve ser capaz de processar pagamentos.
	\item O consumidor pode cancelar sua reserva em até 1 hora antes do início da mesma.
	      % Web app
	\item O sistema deve permitir ao gerente ou dono do estacionamento visualizar o perfil do cliente.
	\item O sistema deve permitir a criação de um cadastro do estacionamento através de um registro.
	\item O sistema deve permitir a definição de usuários com permissão para acessar as telas de gerenciamento do estacionamento.
	\item O sistema deve processar o pagamento baseado no tempo em que o usuário ficou estacionado.
	\item O sistema deve oferecer opções para gerenciar o estacionamento: definir preço por hora ou mês, multas entre outras definições sobre as características do mesmo.
	\item O sistema deve identificar tendências em relação a ocupação do estacionamento.
	      %\item O sistema deve possuir integração com aplicativos web e mobile.
	\item O sistema deve estar integrado com a base de dados do Departamento Estadual de Trânsito (DETRAN) para buscar informações de licença do motorista.
	\item O administrador do sistema pode verificar a ocupação das vagas no estacionamento.
	\item O sistema deve estar integrado com um sistema de automação presente no mercado.
\end{itemize}

\subsubsection{Requisitos não funcionais}

\begin{itemize}
	\item Utilizar base de dados em cloud.
	\item A solução deverá utilizar uma plataforma em nuvem para o banco de dados e servidor web.
	\item O aplicativo móvel deverá rodar sobre a plataforma Android versão 4.0 ou superior.
	\item O aplicativo web para o estacionamento deverá ser acessado através dos seguintes navegadores: firefox e chrome.
\end{itemize}

\subsection{Premissas}

\begin{itemize}
	\item O cliente irá manter o registro de seu perfil, incluindo a placa do veículo, devidamente atualizados.
	\item Em casos que o sistema não possa acomodar o usuário ou o usuário não concorde com os termos de uso e condições do sistema, o usuário será aconselhado a deixar o estacionamento. Nestes casos assume-se que o usuário sempre obedecerá e sairá do estacionamento.
	\item Um consumidor estaciona sempre na vaga correta, e nunca ocupará duas vagas ao mesmo tempo.
	\item O usuário sairá do estacionamento imediatamente após retirar seu carro da vaga utilizada.
	      % From project-charter
	\item Os estacionamentos privados de Porto Alegre estão prontos para aderir a este tipo de tecnologia.
	\item Disponibilidade do patrocinador: durante o planejamento no mínimo 40\% do tempo, e durante a execução no mínimo 80\% do tempo.
\end{itemize}

\subsection{Restrições}

\begin{itemize}
	\item O orçamento é limitado a \maximumBudget{}.
	      %\item O modo de funcionamento básico do sistema não deve depender de qualquer tipo de tecnologia de ponta. \todo[inline,color=orange]{Verificar se esta restrição é plausível.}
	\item Data de conclusão do projeto: \maximumDeadline{}.
\end{itemize}

\subsection{Limites do projeto e exclusões específicas}

\begin{itemize}
	\item O projeto não tem como objetivo a implantação da solução, apenas o desenvolvimento das soluções de software.
	\item Questões de marketing e divulgação do produto não serão tratados neste projeto.
	\item Investimentos em infraestrutura para desenvolvimento estão fora do escopo.
\end{itemize}

\section{Estrutura Analítica do Projeto (EAP)}

Verificar apêndice \ref{ch:wbs}.

\section{Dicionário da EAP e Critérios de Aceitação}

O dicionário da EAP, bem como os critérios de aceitação do projeto, podem ser encontrados no apêndice \ref{ch:wbs-dictionary}.

\section{Orçamento do projeto}

O orçamento previsto pelo projeto é de \realBudget{}, incluindo reservas gerenciais, conforme descrito no plano de gerenciamento de custos (ver capítulo \ref{ch:cost-management-plan}).

\section{Principais entregas e critérios de aceitação}

A tabela \ref{tab:entregas} descreve os principais marcos de entrega do projeto.

\begin{table}[h]
	\begin{tabularx}{1.1\textwidth}{| X | X | c |}
		\hline
		\textbf{Marco}                        & \textbf{Critério de Aceitação}                                                                   & \textbf{Prazo} \\
		\hline
		Planejamento realizado                & Planos de projeto criados                                                                           & 05/06/2017     \\
		\hline
		Ambiente de desenvolvimento preparado & Ambiente para desenvolvimento do aplicativo móvel, web e do software de centralização preparados & 13/06/2017     \\
		\hline
		Ambiente de testes preparado          & Ambiente para testes para o aplicativo móvel, web e software de centralização preparados         & 15/06/2017     \\
		\hline
		Ambiente de produção preparado      & Ambiente de produção para o aplicativo móvel, web e o software de centralização preparados     & 19/07/2017     \\
		\hline
		Plataforma de nuvem contratada        & Selecionada e contratada o servidor em nuvem                                                        & 22/06/2017     \\
		\hline
		Banco de dados criado                 & Banco de dados planejado e criado                                                                   & 12/07/2017     \\
		\hline
		Web API pronta                        & Web API desenvolvida e devidamente testada                                                          & 29/08/2017     \\
		\hline
		Aplicativo do motorista pronto        & Aplicativo do motorista desenvolvido e devidamente testado                                          & 20/10/2017     \\
		\hline
		Aplicativo do estacionamento pronto   & Aplicativo do estacionamento desenvolvido e devidamente testado                                     & 25/12/2017     \\
		\hline
		Projeto Finalizado                    & Projeto executado, entregas realizadas, encerramento realizado e aprovado                           & 01/01/2018     \\
		\hline
	\end{tabularx}
	\centering
	\caption{Principais datas de entregas do projeto.}
	\label{tab:entregas}
\end{table}

\section{Controle de Versão}

\begin{table}[H]
	\begin{tabularx}{\textwidth}{| c | c | X | X |}
		\hline
		\textbf{Versão} & \textbf{Data} & \textbf{Autor}        & \textbf{Notas de Revisão} \\
		\hline
		1                & 23/04/2017    & \projectManagerName{} & Criação do documento     \\
		\hline
		2                & 05/05/2017    & \projectManagerName{} & Revisão de requisitos     \\
		\hline
	\end{tabularx}
	\centering
\end{table}

\section{Aprovações}

\begin{table}[H]
	\begin{tabularx}{\textwidth}{| c | c | X | c |}
		\hline
		\textbf{Função}  & \textbf{Nome}         & \textbf{Assinatura}        & \textbf{Data} \\
		\hline
		Patrocinador       & \projectSponsorName{} & \projectSponsorSignature{} & 06/06/2017    \\
		\hline
		Gerente de projeto & \projectManagerName{} & \projectManagerSignature{} & 06/06/2017    \\
		\hline
	\end{tabularx}
	\centering
\end{table}