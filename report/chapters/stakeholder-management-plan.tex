% Processo de análise das partes interessadas
%   frequência da análise, 
%   mapeamento dos interesses e influências nos projetos
%   Remete ao Controle Integrado de Mudanças em caso de divergência planejado x realizado 
% Análise dos stakeholders com no mínimo:
%   Descrição 
%   Interesses
%   Expectativas
% Identificação de meios e Habilidades de Comunicação adequados a cada parte interessada
% Matriz de Posicionamento (interesse e poder)
% Apresentação dos processos de gerenciamento e controle do engajamento das partes interessadas
%   Métodos de gerenciamento
%   Frequência das ações de gerenciamento
%   Eventos listados na Matriz de Comunicações
%   Remete ao Controle Integrado de Mudanças em caso de divergência planejado x realizado 

\chapter{Plano de gerenciamento das partes interessadas}

\section{Descrição dos processos de gerenciamento das partes interessadas}

\todo[inline,color=red]{Revisar processos para gerenciamento das partes interessadas.}

\begin{itemize}
	\item As partes interessadas relacionadas ao projeto, de forma direta ou indireta, devem ser identificadas e suas informações registradas no documento de registro das partes interessadas (ver capítulo \ref{ch:stakeholder-register}).
	\item A avaliação e identificação das partes interessadas deverá ser realizada utilizando-se o suporte da opinião especializada.
    \item O modelo classificatório do grau de poder/interesse, o qual pode ser encontrado na seção \ref{sec:power-interest-grade}, deverá ser utilizado para guiar e priorizar as ações e estratégias utilizadas para gerenciar as partes interessadas. \todo[inline,color=red]{Verificar necessidade de utilizar outro modelo.}
	\item Os níveis de engajamento das partes interessadas serão documentados conforme matriz de avaliação do nível de engajamento das partes interessadas (ver seção \ref{sec:stakeholder-engagement}). A partir desta matriz, utilizando-se a opinião especializada, deve-se planejar as ações e comunicações necessárias para fechar as lacunas entre os níveis de engajamento atual e o desejado.
    \item Para as partes interessadas priorizadas deverão ser planejadas estratégias para ganhar mais suporte ou reduzir resistência conforme documento de estratégias para gerenciamento das partes interessadas (ver apêndice \ref{}).
    \item Serão realizadas reuniões mensais entre o gerente do projeto e as principais partes interessadas com o objetivo de alinhar expectativas, identificar e avaliar problemas, reavaliar interesses e priroização das partes interessadas.
	\item Utilizando a opinião especializada deverão ser documentados os requisitos de comunicação das partes interessadas, listando as informações que devem ser distribuídas para cada parte interessada, conforme documento descrito no apêndice \ref{ch:stakeholder-communication-requirements}.
\end{itemize}

\section{Mudanças no gerenciamento das partes interessadas}

Mudanças nas análises realizadas sobre as partes interessadas do projeto devem ser tratadas de acordo com o sistema de controle integrado das mudanças (ver seção \ref{sec:change-control-system}). Seus resultados devem ser apresentados na reunião semanal do CCM com suas conclusões, prioridades e ações relacionadas.

\section{Frequência de análise das partes interessadas}

\todo[inline,color=red]{Descrever frequência de análise das partes interessadas.}

\section{Administração do plano de gerenciamento das partes interessadas}

\subsection{Responsável pelo plano}

\begin{itemize}
	\item \projectManagerName{}, gerente de projeto, será o responsável direto pelo plano de gerenciamento das partes interessadas.
\end{itemize}

\subsection{Frequência de atualização do plano de gerenciamento das partes interessadas}

O plano de gerenciamento das partes interessadas será reavaliado mensalmente durante a reunião do CCM, juntamente com os outros planos de gerenciamento do projeto.

\section{Outros assuntos relacionados ao gerenciamento das partes interessadas do projeto não previstos neste plano}

Solicitações não previstas neste plano deverão passar pela aprovação do CCM. Após aprovada o plano deve ser atualizado pelo gerente do projeto.

\section{Controle de versão}

\begin{table}[H]
	\begin{tabularx}{\textwidth}{| c | c | X | X |}
		\hline
		\textbf{Versão} & \textbf{Data} & \textbf{Autor}        & \textbf{Notas de Revisão} \\
		\hline
		1                &               & \projectManagerName{} & Criação do documento     \\
		\hline
	\end{tabularx}
	\centering
\end{table}

\section{Aprovações}

\begin{table}[H]
	\begin{tabularx}{\textwidth}{| c | c | X | c |}
		\hline
		\textbf{Função}  & \textbf{Nome}         & \textbf{Assinatura}        & \textbf{Data} \\
		\hline
		Patrocinador       & \projectSponsorName{} & \projectSponsorSignature{} &               \\
		\hline
		Gerente de projeto & \projectManagerName{} & \projectManagerSignature{} &               \\
		\hline
	\end{tabularx}
	\centering
\end{table}