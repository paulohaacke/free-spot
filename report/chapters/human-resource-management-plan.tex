% Descrição do tipo de estrutura organizacional da empresa onde o projeto está inserido: Projetizada, matricial forte, funcional, Identificar prós e contras do modelo vigente
% Descrição das necessidades de recursos humanos do projeto: Papéis, Competências, Responsabilidades, Quantidades, Relações hierárquicas
% Organograma do projeto (ideal é ter nome + papel)
% Matriz RACI
% Descrição do plano de mobilização da equipe do projeto: Processo de alocação ou contratação, Datas de alocação e saída do projeto
% Descrição do plano de desenvolvimento de competências da equipe do projeto
% Plano de gerenciamento das pessoas do projeto:
%   Formato de acompanhamento e monitoramento da equipe
%   Frequência do acompanhamento
%   Eventos listados na Matriz de Comunicações
%   Possíveis técnicas de resolução de problemas de acordo com a cultura da empresa
%   Processo de avaliação e feedback
%   Remete ao Controle Integrado de Mudanças em caso de divergência planejado x realizado
% Descrever o modelo de gerenciamento de segurança do projeto: Integridade física, Confidencialidade das informações

\chapter{Plano de gerenciamento dos recursos humanos}
\label{ch:human-resource-management-plan}

\section{Visão geral}

\todo[inline,color=red]{Visão geral do plano de RH.}

\section{Organograma do projeto}

O organograma do projeto é apresentado na figura \ref{fig:organization-chart}.

\begin{figure}[h]
	\centering
	\begin{tikzpicture}[node distance = 0.7cm and 0.7cm, auto]
		\node (sponsor) [process] {Patrocinador};
		\node (project-manager) [process, below= of sponsor] {Gerente do Projeto};
		% Left side
		\node (mobile-dev-one) [process, below left= of project-manager] {Desenvolvedor mobile 1};
		\node (mobile-dev-two) [process, below= of mobile-dev-one] {Desenvolvedor mobile 2};
		\node (front-end-web-dev) [process, below= of mobile-dev-two] {Desenvolvedor web front-end};
		\node (back-end-web-dev) [process, below= of front-end-web-dev] {Desenvolvedor web back-end};
		\node (software-engineer) [process, below= of back-end-web-dev] {Engenheiro de software};
		\node (system-developer) [process, below= of software-engineer] {Desenvolvedor de sistema};
		% Right side
		\node (solution-architect) [process, below right= of project-manager] {Arquiteto de solução};
		\node (software-architect) [process, below= of solution-architect] {Arquiteto de software};
		\node (eletric-engineer) [process, below= of software-architect] {Engenheiro eletricista};
		\node (test-analyst-1) [process, below= of eletric-engineer] {Analista de testes 1};
		\node (test-analyst-2) [process, below= of test-analyst-1] {Analista de testes 2};
		\node (database-analyst) [process, below= of test-analyst-2] {Analista de banco de dados};

		\path [line] (sponsor) -- (project-manager);
		\path [line] (project-manager) |- (mobile-dev-one);
		\path [line] (project-manager) |- (mobile-dev-two);
		\path [line] (project-manager) |- (front-end-web-dev);
		\path [line] (project-manager) |- (back-end-web-dev);
		\path [line] (project-manager) |- (software-engineer);
		\path [line] (project-manager) |- (system-developer);
		\path [line] (project-manager) |- (solution-architect);
		\path [line] (project-manager) |- (software-architect);
		\path [line] (project-manager) |- (eletric-engineer);
		\path [line] (project-manager) |- (test-analyst-1);
		\path [line] (project-manager) |- (test-analyst-2);
		\path [line] (project-manager) |- (database-analyst);
	\end{tikzpicture}
	\caption{Organograma do projeto.}
	\label{fig:organization-chart}
\end{figure}

\section{Diretório do time do projeto}

\begin{longtable}{ l p{0.25\textwidth} p{0.13\textwidth} p{0.28\textwidth} p{0.2\textwidth} }
	\toprule
	\thead[c]{\textbf{No}} & \thead[c]{\textbf{Nome}} & \thead[c]{\textbf{Área}} & \thead[c]{\textbf{E-mail}} & \thead[c]{\textbf{Telefone}} \\
	\midrule
	1                      & \ceoName{}               & CEO                       & \email{}                   & \phone{}                     \\
	2                      & \projectManagerName{}    & Gerente do Projeto        & \email{}                   & \phone{}                     \\
	3                      & \mobDevOneName{}         & Membro da equipe          & \email{}                   & \phone{}                     \\
	4                      & \mobDevTwoName{}         & Membro da equipe          & \email{}                   & \phone{}                     \\
	5                      & \frontWebDevName{}       & Membro da equipe          & \email{}                   & \phone{}                     \\
	6                      & \backWebDevName{}        & Membro da equipe          & \email{}                   & \phone{}                     \\
	7                      & \softEngName{}           & Membro da equipe          & \email{}                   & \phone{}                     \\
	8                      & \sysDevName{}            & Membro da equipe          & \email{}                   & \phone{}                     \\
	9                      & \solArcName{}            & Membro da equipe          & \email{}                   & \phone{}                     \\
	10                     & \softArcName{}           & Membro da equipe          & \email{}                   & \phone{}                     \\
	11                     & \elecEngName{}           & Membro da equipe          & \email{}                   & \phone{}                     \\
	12                     & \testAnalOneName{}       & Membro da equipe          & \email{}                   & \phone{}                     \\
	13                     & \testAnalTwoName{}       & Membro da equipe          & \email{}                   & \phone{}                     \\
	14                     & \dbAnalName{}            & Membro da equipe          & \email{}                   & \phone{}                     \\
	\bottomrule
	\caption{Diretório do time do projeto.}
	\centering
\end{longtable}

\section{Matriz de responsabilidades}

\todo[inline,color=red]{Preencher matriz de responsabilidades}

\begin{longtable}{ l p{0.025\textwidth} p{0.025\textwidth}  p{0.025\textwidth}  p{0.025\textwidth}  p{0.025\textwidth} p{0.025\textwidth} p{0.025\textwidth} p{0.025\textwidth}  p{0.025\textwidth}  p{0.025\textwidth}  p{0.025\textwidth} p{0.025\textwidth} p{0.025\textwidth} p{0.025\textwidth} }
	\toprule
	\textbf{Atividade} & \rot{\textbf{\parbox{6cm}{Patrocinador}}} & \rot{\textbf{\parbox{6cm}{Gerente do Projeto}}} & \rot{\textbf{\parbox{6cm}{Desenvolvedor mobile 1}}} & \rot{\textbf{\parbox{6cm}{Desenvolvedor mobile 2}}} & \rot{\textbf{\parbox{6cm}{Desenvolvedor web front-end}}} & \rot{\textbf{\parbox{6cm}{Desenvolvedor web back-end}}} & \rot{\textbf{\parbox{6cm}{Engenheiro de software}}} & \rot{\textbf{\parbox{6cm}{Desenvolvedor de sistema}}} & \rot{\textbf{\parbox{6cm}{Arquiteto de solução}}} & \rot{\textbf{\parbox{6cm}{Arquiteto de software}}} & \rot{\textbf{\parbox{6cm}{Engenheiro eletricista}}} & \rot{\textbf{\parbox{6cm}{Analista de testes 1}}} & \rot{\textbf{\parbox{6cm}{Analista de testes 2}}} & \rot{\textbf{\parbox{6cm}{Analista de banco de dados}}} \\
	\midrule
	Atividade 1 & R &   &   &   &   &   &   &   &   &   &   &   &  &   \\
	Atividade 2 & A &   &   &   &   &   &   &   &   &   &   &   &  &   \\
	Atividade 3 &   & R & A & R &   &   &   &   &   &   &   &   &  &   \\
	\bottomrule
	\caption{Matriz de responsabilidades.}
	\centering
\end{longtable}

\section{Novos recursos, re-alocação e substituição dos membros do time}

O gerente de projeto deve buscar manter a permanência de todos integrantes da equipe do projeto e, por isso, é o responsável por este plano do projeto.

Em caso de re-alocação de um profissional integrante da equipe do projeto, caberá ao gerente do projeto, juntamente com o restante da equipe do projeto, encontrar e identificar um substituto em comum acordo com as diretrizes do projeto e a função a ser exercida. 

Cabe ao patrocinador autorizar a contratação de novos recursos.

\section{Treinamento}

Não estão previstos treinamentos para a equipe do projeto. O gerente de projeto deve autorizar qualquer necessidade de treinamento, de modo que os custos serão alocados dentro das reservas gerenciais de acordo com as orientações do plano de gerenciamento de custos (ver capítulo \ref{ch:cost-management-plan}).

\section{Avaliação de resultados}

A avaliação de resultados será realizada sobre a equipe do projeto e de forma individual. 

A avaliação da equipe do projeto será realizada trimestralment através da utilização do formulário de avaliação de desempenho da equipe (ver apêndice \ref{ch:team-evaluation-form}), o qual deve ser preenchido por cada membro da equipe do projeto.

Mensalmente será realizada também uma avaliação do desempenho individual de cada membro da equipe por seu superior hierárquico no organograma do projeto. Os membros da equipe serão avaliados pelo gerente do projeto. Já o gerente do projeto será avaliado pelo patrocinador. Este tipo de avaliação será realizada através de reuniões individuais com cada membro da equipe.

\section{Bonificação}

\todo[inline,color=red]{Método de bonificação do plano de RH.}

\section{Frequência de avaliação consolidada dos resultados do time}

Os resultados obtidos nas avaliações trimestrais da equipe devem ser compilados e apresentados durante a última reunião do CCM, conforme previsto no plano de grenciamento das comunicações (ver capítulo \ref{ch:communication-management-plan}).

\section{Alocações financeiras para o gerenciamento de RH}

Todas as medidas de gerenciamento de recurso humanos que necessitem de gastos adicionais deverão ser alocadas dentro das reservas gerenciais do projeto, desde que dentro da alçada do gerente do projeto, conforme autonomias descritas no plano de gerenciamento dos custos (ver capítulo \ref{ch:cost-management-plan}). 

\section{Administração do plano de gerenciamento de recursos humanos}

\subsection{Responsável pelo plano}

\begin{itemize}
	\item \projectManagerName{}, gerente de projeto, será o responsável direto pelo plano de gerenciamento de RH.
\end{itemize}

\subsection{Frequência de atualização do plano de gerenciamento de RH}

O plano de gerenciamento de RH será reavaliado mensalmente durante a reunião do CCM, juntamente com os outros planos de gerenciamento do projeto.

\section{Outros assuntos relacionados ao gerenciamento de RH do projeto não previstos neste plano}

Solicitações não previstas neste plano deverão passar pela aprovação do CCM. Após aprovada o plano deve ser atualizado pelo gerente do projeto.

\section{Controle de versão}

\begin{table}[H]
	\begin{tabularx}{\textwidth}{| c | c | X | X |}
		\hline
		\textbf{Versão} & \textbf{Data} & \textbf{Autor}        & \textbf{Notas de Revisão} \\
		\hline
		1                &               & \projectManagerName{} & Criação do documento     \\
		\hline
	\end{tabularx}
	\centering
\end{table}

\section{Aprovações}

\begin{table}[H]
	\begin{tabularx}{\textwidth}{| c | c | X | c |}
		\hline
		\textbf{Função}  & \textbf{Nome}         & \textbf{Assinatura}        & \textbf{Data} \\
		\hline
		Patrocinador       & \projectSponsorName{} & \projectSponsorSignature{} &               \\
		\hline
		Gerente de projeto & \projectManagerName{} & \projectManagerSignature{} &               \\
		\hline
	\end{tabularx}
	\centering
\end{table}