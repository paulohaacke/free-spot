% Descrição do método de gerenciamento do custo a ser empregado no projeto incluindo: Sistema de alocação de custos, Relatórios de controle, Periodicidade de atualização
% Método de estimativa dos custos
% Apresentação da composição de valores que totalizam o custo do projeto incluindo: 
%   Custo com pessoal
%   Aquisições
%   Reservas Gerenciais
%   Reservas de Contingência (oriunda dos riscos)
%   Outros
% EVA – Análise de Valor Agregado do Projeto (simulação em momento projetado):
%   Aplicação do método de EVA no projeto
%   Detalhamento dos índices utilizados e análise dos resultados
%   Curvas de desempenho
%   Apresentação de análise comparativa entre o orçamento inicial aprovado do projeto e o custo orçado inicialmente para o projeto: Justificar em caso de diferença 
% Modelo de análise e controle de custos do projeto:
%   Comparação do realizado x linha de base de custos do projeto
%   Ações caso existam diferenças
%   Fluxo de caixa
%   Custo final projetado considerando-se o desempenho atual do projeto
% Apresentação dos custos correntes ou potenciais que não serão apropriados ao projeto
% Descrição do processo de controle dos custos
%   Frequência das ações de controle
%   Eventos listados na Matriz de Comunicações
%   Remete ao Controle Integrado de Mudanças em caso de divergência planejado x realizado

\chapter{Plano de gerenciamento dos custos}
\label{ch:cost-management-plan}

%\todo[inline,color=red]{Revisar plano de gerenciamento dos custos.}

%\section{Visão geral}

%O plano de gerenciamento dos custos descreve como os custos do projeto serão planejados, estruturados e controlados fornecendo detalhes sobre os processos e ferramentas utilizados para gerenciar questões relacionadas a custos.

%Os processos e ferramentas descritos neste plano orientam o gerenciamento das atividades vistas na figura \ref{fig:cost-management-plan-flowchart}.

%\begin{figure}[h]
%	\centering
%	\begin{tikzpicture}[node distance = 0.7cm and 0.7cm, auto]
%		\node (estimate) [process] {Estimar Custos};
%		\node (determine-budget) [process, right= of estimate] {Determinar Orçamento};
%		\node (control-costs) [process, right= of determine-budget] {Controlar Custos};
%		\path [line] (estimate) -- (determine-budget);
%		\path [line] (determine-budget) -- (control-costs);
%	\end{tikzpicture}
%	\caption{Processo para gerenciamento dos custos do projeto.}
%	\label{fig:cost-management-plan-flowchart}
%\end{figure}

\section{Descrição dos processos de gerenciamento de custos}

\begin{itemize}
	% Estimar custos
	% Ver custo da qualidade (PMBOK 8.1.2.2)
	\item Utilizando-se o método \foreign{bottom-up} devem ser estimados os custos para cada uma das atividades presentes na EAP.
	\item A estimativativa de custos será conduzida pelo gerente de projetos e apoiada pela opinião especializada da equipe técnica.
	\item Custos que possuam incertezas e riscos significantes deverão utilizar a análise de PERT, conforme equação \ref{eq:cost-pert}.
	      \begin{equation}\label{eq:cost-pert}
		      CE = \frac{CO+4CMP+CP}{6}
	      \end{equation}
	      \begin{description}
		      \item[Custo Estimado (DE):] a melhor estimativa do custo necessário para completar a atividade, levando em consideração o fato de que o projeto nem sempre corre conforme o planejado.
		      \item[Custo Mais Provável (MP):] custo da atividade baseado em um esforço de avaliação realista para o trabalho necessário e quaisquer outros gastos previstos.
		      \item[Custo Otimista (O):] custo da atividade baseado na análise do melhor cenário possível para a atividade.
		      \item[Custo Pessimista (P):] custo da atividade baseado na análise do pior cenário para a atividade.
	      \end{description}
	\item A atualização e gerenciamento dos custos do projeto será realizada utilizando o software \projectManagementSoftwareName{}.
	      % Determinar o orçamento
	\item Custos indiretos não serão considerados nos custos deste projeto.
	      % Controlar custos
	\item A avaliação dos custos será feito por meio da comparação entre 3 estimativas: valor planejado, valor real e valor agregado.
	\item O controle dos custos será feito tomando como base a estimativa de valor agregado (ver apêndice \ref{project-monitoring-report}).
	\item Custos recorrentes de tempo e material, como a contratação de um servidor em nuvem, serão considerados de forma proporcional no período de sua contratação até o encerramento do projeto. Mais detalhes podem ser encontrados no plano de gerenciamento das aquisições, localizado no apêndice \ref{ch:procurement-management-plan}.
\end{itemize}

\section{Frequência de avaliação do orçamento do projeto e das reservas gerenciais}

O orçamento, assim como o uso de reservas gerenciais, do projeto será avaliado semanalmente pelo gerente do projeto.

\section{Orçamento do projeto}

 A linha de base de custos é formada por \resourcesBudget{} de gastos internos com recursos e aquisições e \contingencyBudget{} para reservas de contingência (calculada de acordo com o plano de gerenciamento de riscos, apresentado no capítulo \ref{ch:risk-management-plan}). Sobre a linha de base foi adicionado o valor de \managementBudget{} utilizado para reserva gerencial, conforme descrito na seção \ref{ch:management-budget}.

 Somando o valor da linha de base de custos e a reserva gerencial forma-se o orçamento do projeto, que é de \realBudget{}.

\section{Taxa padrão dos recursos}

\begin{longtable}{ p{0.2\textwidth} p{0.2\textwidth} l l l }
	\toprule
	\textbf{Recurso}      & \textbf{Cargo}              & \textbf{Experência} & \textbf{Tipo} & \textbf{Taxa Padrão} \\
	\midrule
	\endhead
	\multicolumn{5}{c}{{\textit{Continua na próxima página.}}} \\
	\caption{Taxa padrão dos recursos.}
	\endfoot
	\endlastfoot
	\ceoName{}            & CEO                         & N/A                  & Trabalho      & R\$ 0,00/hr           \\
	\midrule
	\projectManagerName{} & Gerente do projeto          & Pleno                & Trabalho      & R\$ 83,00/hr          \\
	\midrule
	\mobDevOneName{}      & Desenvolvedor mobile        & Pleno                & Trabalho      & R\$ 65,00/hr          \\
	\midrule
	\mobDevTwoName{}      & Desenvolvedor mobile        & Pleno                & Trabalho      & R\$ 65,00/hr          \\
	\midrule
	\frontWebDevName{}    & Desenvolvedor web front-end & Pleno                & Trabalho      & R\$ 61,00/hr          \\
	\midrule
	\backWebDevName{}     & Desenvolvedor web back-end  & Pleno                & Trabalho      & R\$ 66,00/hr          \\
	\midrule
	\softEngName{}        & Engenheiro de software      & Sênior              & Trabalho      & R\$ 83,00/hr          \\
	\midrule
	\softArcName{}        & Arquiteto de software       & Pleno                & Trabalho      & R\$ 71,00/hr          \\
	\midrule
	\testAnalOneName{}    & Analista de testes          & Sênior              & Trabalho      & R\$ 60,00/hr          \\
	\midrule
	\testAnalTwoName{}    & Analista de testes          & Pleno                & Trabalho      & R\$ 60,00/hr          \\
	\midrule
	\dbAnalName{}         & Analista de banco de dados  & Sênior              & Trabalho      & R\$ 81,00/hr          \\
	\bottomrule
	\caption{Taxa padrão dos recursos.}
	\centering
\end{longtable}

\section{Custos dos recursos e aquisições do projeto por atividade}

O documento descrevendo o custo de cada atividade no projeto encontra-se no apêndice \ref{ch:activities-cost}.

\section{Aquisições}

Conforme descrito no plano de gerenciamento das aquisições do projeto, estima-se que o haverá um gasto de \procurementBudget{} para a contratação do servidor em nuvem durante o período do projeto. Este valor estará incluso no orçamento do projeto.

\section{Reservas gerenciais}
\label{ch:management-budget}

Além da linha de base de custos, o gerente do projeto contará com uma reserva gerencial de 10\% sobre o valor da linha de base, a qual estará diponível em casos de riscos desconhecidos e atividades não identificadas no escopo.

\section{Estrutura Analítica de Custos}

Para estimar o custo dos pacotes de trabalho foi utilizado como ferramenta a estimativa \foreign{bottom-up}. O diagrama gerado como resultado da estimativa é a estrutura analítica de custos, a qual pode ser vista na figura \ref{fig:cbs}.

\begin{figure}[H]
\centering
\begin{tikzpicture}[node distance = 0.18cm and 0.1cm, auto, scale=0.75, every node/.style={scale=0.75}]
	% Project management
	\node (project-management) [wbsblock] {Gerenciamento do projeto\\R\$ 48.187,17};
	\node (initiation) [wbsblock, below right= of project-management, xshift=-6em] {Iniciação\\R\$ 2.576,33};
	\node (planning) [wbsblock, below = of initiation] {Planejamento\\R\$ 14.774,00};
	\node (scope-approval) [wbsblock, below = of planning] {Aprovação do escopo\\R\$ 539,50};
	\node (project-monitoring) [wbsblock, below = of scope-approval] {Monitoramento do projeto\\R\$ 24.298,00};
	\node (closure) [wbsblock, below = of project-monitoring] {Encerramento\\R\$ 5.999,33};
	% Infraestrutura
	\node (infrastructure) [wbsblock, right = of project-management, xshift=1.5em] {Infraestrutura\\R\$ 10.067,67};
	\node (dev-env) [wbsblock, below right = of infrastructure, xshift=-6em] {Ambiente de desenvolvimento\\R\$ 2.369,67};
	\node (test-env) [wbsblock, below = of dev-env] {Ambiente de testes\\R\$ 2.292,00};
	\node (prod-env) [wbsblock, below = of test-env] {Ambiente de produção\\R\$ 5.406,00};
	% Banco de dados
	\node (webapi) [wbsblock, right = of infrastructure, xshift=1.5em] {Web API\\R\$ 31.689,33};
	\node (cloudplat) [wbsblock, below right = of webapi, xshift=-6em] {Contratar plataforma na nuvem\\R\$ 13.634,33};
	\node (database) [wbsblock, below = of cloudplat] {Banco de dados\\R\$ 6.211,00};
	\node (apistruct) [wbsblock, below = of database] {Definir estrutura da API\\R\$ 1.176,00};
	\node (resroute) [wbsblock, below = of apistruct] {Criar rotas para recursos\\R\$ 3.168,00};
	\node (wa-tests) [wbsblock, below = of resroute] {Testes\\R\$ 7.500,00};
	% Aplicativo do motorista
	\node (drivers-app) [wbsblock, right = of webapi, xshift=1.5em] {Aplicativo do motorista\\R\$ 33.180,67};
	\node (authentication) [wbsblock, below right = of drivers-app, xshift=-6em] {Autenticação\\R\$ 3.764,00};
	\node (spot-reservation) [wbsblock, below = of authentication] {Reserva de vagas\\R\$ 5.129,00};
	\node (parking-list) [wbsblock, below = of spot-reservation] {Lista de estacionamentos\\R\$ 3.172,00};
	\node (payment-process) [wbsblock, below = of parking-list] {Processamento de pagamento\\R\$ 5.259,00};
	\node (user-management) [wbsblock, below = of payment-process] {Gerência de usuário\\R\$ 3.190,00};
	\node (am-tests) [wbsblock, below = of user-management] {Testes\\R\$ 12.666,67};
	% Aplicativo do estacionamento
	\node (parking-app) [wbsblock, right = of drivers-app, xshift=1.5em] {Aplicativo do estacionamento\\R\$ 52.436,52};
	\node (user-auth-management) [wbsblock, below right = of parking-app, xshift=-6em] {Autenticação e gerência de usuários\\R\$ 5.872,50};
	\node (client-management) [wbsblock, below = of user-auth-management] {Gerenciamento dos clientes\\R\$ 4.264,00};
	\node (system-register) [wbsblock, below = of client-management] {Solicitação de cadastro no sistema\\R\$ 2.401,50};
	\node (parking-management) [wbsblock, below = of system-register] {Gerenciamento do estacionamento\\R\$ 14.820,00};
	\node (garage-register) [wbsblock, below right = of parking-management, xshift=-6em] {Cadastro e gerência de garagens\\R\$ 4.467,50};
	\node (spot-state) [wbsblock, below = of garage-register] {Situação das vagas\\R\$ 4.010,00};
	\node (automation-integration) [wbsblock, below = of spot-state] {Integração com sistema de automação\\R\$ 4.957,00};
	\node (parking-configuration) [wbsblock, below = of automation-integration] {Configurações do estacionamento\\R\$ 1385,50};
	\node (reports) [wbsblock, below left = of parking-configuration, xshift=6em] {Relatórios\\R\$ 4.962,50};
	\node (DETRAN-integration) [wbsblock, below = of reports] {Integração com DETRAN\\R\$ 3.153,50};
	\node (ae-tests) [wbsblock, below = of DETRAN-integration] {Testes\\R\$ 16.962,52};
	% Garantia da qualidade
	\node (quality-assurance) [wbsblock, right = of parking-app, xshift=1.5em] {Auditoria da qualidade\\R\$ 4.560,00};
	% WBS root
	\node (free-spot) [wbsblock, above = of webapi, yshift=2em] {Vaga livre\\R\$ 180.121,35};

	\path [simpleline] (free-spot.south) |- ++(0,-0.7cm) -| (project-management);
	\path [simpleline] (free-spot.south) |- ++(0,-0.7cm) -| (infrastructure);
	\path [simpleline] (free-spot.south) |- ++(0,-0.7cm) -| (webapi);
	\path [simpleline] (free-spot.south) |- ++(0,-0.7cm) -| (drivers-app);
	\path [simpleline] (free-spot.south) |- ++(0,-0.7cm) -| (parking-app);
	\path [simpleline] (free-spot.south) |- ++(0,-0.7cm) -| (quality-assurance);

	\path [simpleline] (project-management.215) |- (initiation);
	\path [simpleline] (project-management.215) |- (planning);
	\path [simpleline] (project-management.215) |- (scope-approval);
	\path [simpleline] (project-management.215) |- (project-monitoring);
	\path [simpleline] (project-management.215) |- (closure);

	\path [simpleline] (infrastructure.215) |- (dev-env);
	\path [simpleline] (infrastructure.215) |- (test-env);
	\path [simpleline] (infrastructure.215) |- (prod-env);

	\path [simpleline] (webapi.215) |- (cloudplat);
	\path [simpleline] (webapi.215) |- (database);
	\path [simpleline] (webapi.215) |- (apistruct);
	\path [simpleline] (webapi.215) |- (resroute);
	\path [simpleline] (webapi.215) |- (wa-tests);

	\path [simpleline] (drivers-app.215) |- (authentication);
	\path [simpleline] (drivers-app.215) |- (spot-reservation);
	\path [simpleline] (drivers-app.215) |- (parking-list);
	\path [simpleline] (drivers-app.215) |- (payment-process);
	\path [simpleline] (drivers-app.215) |- (user-management);
	\path [simpleline] (drivers-app.215) |- (am-tests);

	\path [simpleline] (parking-app.215) |- (user-auth-management);
	\path [simpleline] (parking-app.215) |- (client-management);
	\path [simpleline] (parking-app.215) |- (system-register);
	\path [simpleline] (parking-app.215) |- (parking-management);
	\path [simpleline] (parking-app.215) |- (reports);
	\path [simpleline] (parking-app.215) |- (DETRAN-integration);
	\path [simpleline] (parking-app.215) |- (ae-tests);

	\path [simpleline] (parking-management.215) |- (garage-register);
	\path [simpleline] (parking-management.215) |- (spot-state);
	\path [simpleline] (parking-management.215) |- (automation-integration);
	\path [simpleline] (parking-management.215) |- (parking-configuration);

\end{tikzpicture}
\caption{Estrutura analítica de custos.}
\label{fig:cbs}
\end{figure}

\section{Autonomias}

O gerente do projeto possui autonomia total para uso da reserva contingencial.

O gerente do projeto possui autonomia sobre o uso da reserva gerencial em diferentes eventos. Para cada evento em que haja necessidade de utilizar a reserva gerencial o gerente do projeto possui autonomia conforme descrito na tabela \ref{tab:project-manager-cost-autonomy}.

\begin{table}[h]
	\centering
	\begin{tabular}{| l | l |}
		\hline
		\textbf{Uso da Reserva Gerencial} & \textbf{Autonomia}                                   \\
		\hline
		Até R\$ 3.000,00                 & Gerente do projeto                                   \\
		\hline
		Até R\$ 7.000,00                 & Gerente do projeto com autorização do patrocinador \\
		\hline
		Acima de R\$ 7.000,00             & Apenas o patrocinador                                \\
		\hline
	\end{tabular}
	\caption{Autonomias do gerente de projetos sobre a reserva gerencial.}
	\label{tab:project-manager-cost-autonomy}
\end{table}

A reserva gerencial pode ser utilizada de forma cumulativa. Quando for atingido o limite de 50\% de uso da reserva gerencial o gerente do projeto deve informar o patrocinador.

Para medidas prioritárias ou urgentes que estejam fora da alçada do gerente do projeto, ou na ausência de reserva gerencial disponível o patrocinador deverá ser acionado para a tomada da decisão.

\section{Alocação financeira das mudanças no orçamento}

Conforme descrito no plano de gerenciamento das mudanças (ver capítulo ch:change-management-plan), a reserva gerencial pode ser utilizada caso o gerente do projeto necessite realizar mudanças de orçamento de prioridade média.

Todas as mudanças no orçamento do projeto deverão ser classificadas de acordo com o modelo de priorização integrada de mudanças (ver seção \ref{sec:integrated-change-priorization}), além disso devem seguir conforme o fluxo apresentado pelo sistema de controle integrado das mudanças (ver seção \ref{sec:change-control-system}).

\section{Administração do plano de gerenciamento de custos}

\subsection{Responsável pelo plano}

\begin{itemize}
	\item \projectManagerName{}, gerente de projeto, será o responsável direto pelo plano de gerenciamento do tempo.
\end{itemize}

\subsection{Frequência de atualização do plano de gerenciamento de custos}

O plano de gerenciamento dos custos será reavaliado semanalmente durante a reunião do CCM, juntamente com os outros planos de gerenciamento do projeto.

\section{Outros assuntos relacionados ao gerenciamento de custos do projeto não previstos neste plano}

Solicitações não previstas neste plano deverão passar pela aprovação do CCM. Após aprovada o plano deve ser atualizado pelo gerente do projeto.

\section{Controle de versão}

\begin{table}[H]
	\begin{tabularx}{\textwidth}{| c | c | X | X |}
		\hline
		\textbf{Versão} & \textbf{Data} & \textbf{Autor}        & \textbf{Notas de Revisão}           \\
		\hline
		1                & 23/04/2017    & \projectManagerName{} & Criação do documento               \\
		\hline
		2                & 29/04/2017    & \projectManagerName{} & Corrigindo taxa padrão dos recursos \\
		\hline
		3                & 04/05/2017    & \projectManagerName{} & Adicionando autonomias               \\
		\hline
		4                & 05/05/2017    & \projectManagerName{} & Revisão do documento                \\
		\hline
	\end{tabularx}
	\centering
\end{table}

\section{Aprovações}

\begin{table}[H]
	\begin{tabularx}{\textwidth}{| c | c | X | c |}
		\hline
		\textbf{Função}  & \textbf{Nome}         & \textbf{Assinatura}        & \textbf{Data} \\
		\hline
		Patrocinador       & \projectSponsorName{} & \projectSponsorSignature{} & 06/06/2017    \\
		\hline
		Gerente de projeto & \projectManagerName{} & \projectManagerSignature{} & 06/06/2017    \\
		\hline
	\end{tabularx}
	\centering
\end{table}

