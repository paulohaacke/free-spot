% Descrição do método de gerenciamento do custo a ser empregado no projeto incluindo: Sistema de alocação de custos, Relatórios de controle, Periodicidade de atualização
% Método de estimativa dos custos
% Apresentação da composição de valores que totalizam o custo do projeto incluindo: 
%   Custo com pessoal
%   Aquisições
%   Reservas Gerenciais
%   Reservas de Contingência (oriunda dos riscos)
%   Outros
% EVA – Análise de Valor Agregado do Projeto (simulação em momento projetado):
%   Aplicação do método de EVA no projeto
%   Detalhamento dos índices utilizados e análise dos resultados
%   Curvas de desempenho
%   Apresentação de análise comparativa entre o orçamento inicial aprovado do projeto e o custo orçado inicialmente para o projeto: Justificar em caso de diferença 
% Modelo de análise e controle de custos do projeto:
%   Comparação do realizado x linha de base de custos do projeto
%   Ações caso existam diferenças
%   Fluxo de caixa
%   Custo final projetado considerando-se o desempenho atual do projeto
% Apresentação dos custos correntes ou potenciais que não serão apropriados ao projeto
% Descrição do processo de controle dos custos
%   Frequência das ações de controle
%   Eventos listados na Matriz de Comunicações
%   Remete ao Controle Integrado de Mudanças em caso de divergência planejado x realizado

\chapter{Plano de gerenciamento dos custos}
\label{ch:cost-management-plan}

%\todo[inline,color=red]{Revisar plano de gerenciamento dos custos.}

%\section{Visão geral}

%O plano de gerenciamento dos custos descreve como os custos do projeto serão planejados, estruturados e controlados fornecendo detalhes sobre os processos e ferramentas utilizados para gerenciar questões relacionadas a custos.

%Os processos e ferramentas descritos neste plano orientam o gerenciamento das atividades vistas na figura \ref{fig:cost-management-plan-flowchart}.

%\begin{figure}[h]
%	\centering
%	\begin{tikzpicture}[node distance = 0.7cm and 0.7cm, auto]
%		\node (estimate) [process] {Estimar Custos};
%		\node (determine-budget) [process, right= of estimate] {Determinar Orçamento};
%		\node (control-costs) [process, right= of determine-budget] {Controlar Custos};
%		\path [line] (estimate) -- (determine-budget);
%		\path [line] (determine-budget) -- (control-costs);
%	\end{tikzpicture}
%	\caption{Processo para gerenciamento dos custos do projeto.}
%	\label{fig:cost-management-plan-flowchart}
%\end{figure}

\section{Descrição dos processos de gerenciamento de custos}

\begin{itemize}
	% Estimar custos
	% Ver custo da qualidade (PMBOK 8.1.2.2)
	\item Utilizando-se o método \foreign{bottom-up} devem ser estimados os custos para cada uma das atividades presentes na EAP.
	\item A estimativativa de custos será conduzida pelo gerente de projetos e apoiada pela opinião especializada da equipe técnica.
	\item Custos que possuam incertezas e riscos significantes deverão utilizar a análise de PERT, conforme equação \ref{eq:cost-pert}.
	      \begin{equation}\label{eq:cost-pert}
		      CE = \frac{CO+4CMP+CP}{6}
	      \end{equation}
	      \begin{description}
		      \item[Custo Estimado (DE):] a melhor estimativa do custo necessário para completar a atividade, levando em consideração o fato de que o projeto nem sempre corre conforme o planejado.
		      \item[Custo Mais Provável (MP):] custo da atividade baseado em um esforço de avaliação realista para o trabalho necessário e quaisquer outros gastos previstos.
		      \item[Custo Otimista (O):] custo da atividade baseado na análise do melhor cenário possível para a atividade.
		      \item[Custo Pessimista (P):] custo da atividade baseado na análise do pior cenário para a atividade.
	      \end{description}
	\item A atualização e gerenciamento dos custos do projeto será realizada utilizando o software \projectManagementSoftwareName{}.
	      % Determinar o orçamento
	\item Custos indiretos não serão considerados nos custos deste projeto.
	      % Controlar custos
	\item A avaliação dos custos será feito por meio da comparação entre 3 estimativas: valor planejado, valor real e valor agregado.
	\item O controle dos custos será feito tomando como base a estimativa de valor agregado (ver apêndice \ref{project-monitoring-report}).
	\item Custos recorrentes, como a contratação de um servidor em nuvem, serão considerados custos indiretos e não entrarão no orçamentos deste projeto.
\end{itemize}

\section{Frequência de avaliação do orçamento do projeto e das reservas gerenciais}

O orçamento, assim como o uso de reservas gerenciais, do projeto será avaliado semanalmente pelo gerente do projeto.

\section{Orçamento do projeto}

O orçamento do projeto é de \realBudget{}, o que abrange os gastos internos com recursos, aquisições e reservas gerenciais.

\section{Taxa padrão dos recursos}

\begin{longtable}{ p{0.2\textwidth} p{0.2\textwidth} l l l }
	\toprule
	\textbf{Recurso}      & \textbf{Cargo}              & \textbf{Experência} & \textbf{Tipo} & \textbf{Taxa Padrão} \\
	\midrule
	\ceoName{}            & CEO                         & N/A                  & Trabalho      & R\$ 0,00/hr           \\
	\midrule
	\projectManagerName{} & Gerente do projeto          & Pleno                & Trabalho      & R\$ 67,00/hr          \\
	\midrule
	\mobDevOneName{}      & Desenvolvedor mobile        & Pleno                & Trabalho      & R\$ 50,00/hr          \\
	\midrule
	\mobDevTwoName{}      & Desenvolvedor mobile        & Pleno                & Trabalho      & R\$ 50,00/hr          \\
	\midrule
	\frontWebDevName{}    & Desenvolvedor web front-end & Pleno                & Trabalho      & R\$ 44,00/hr          \\
	\midrule
	\backWebDevName{}     & Desenvolvedor web back-end  & Pleno                & Trabalho      & R\$ 51,00/hr          \\
	\midrule
	\softEngName{}        & Engenheiro de software      & Sênior              & Trabalho      & R\$ 67,00/hr          \\
	\midrule
	\softArcName{}        & Arquiteto de software       & Pleno                & Trabalho      & R\$ 52,00/hr          \\
	\midrule
	\testAnalOneName{}    & Analista de testes          & Sênior              & Trabalho      & R\$ 45,00/hr          \\
	\midrule
	\testAnalTwoName{}    & Analista de testes          & Pleno                & Trabalho      & R\$ 45,00/hr          \\
	\midrule
	\dbAnalName{}         & Analista de banco de dados  & Sênior              & Trabalho      & R\$ 66,00/hr          \\
	\bottomrule
	\caption{Taxa padrão dos recursos.}
	\centering
\end{longtable}

\section{Custos dos recursos do projeto por atividade}

O documento descrevendo o custo de cada atividade no projeto encontra-se no apêndice \ref{ch:activities-cost}.

\section{Aquisições}

Conforme descrito no plano de gerenciamento das aquisições o projeto, estima-se que o projeto terá um gasto de \procurementBudget{} para a contratação do servidor em nuvem durante o período do projeto. Este valor estará incluso no orçamento do projeto.

\section{Reservas gerenciais}

Além da linha de base de custos, o gerente do projeto contará com uma reserva gerencial de 10\% sobre o valor da linha de base, a qual estará diponível em casos de riscos desconhecidos e atividades não identificadas no escopo.

\section{Autonomias}

O gerente do projeto possui autonomia total para uso da reserva contingencial.

O gerente do projeto possui autonomia sobre o uso da reserva gerencial em diferentes eventos. Para cada evento em que haja necessidade de utilizar a reserva gerencial o gerente do projeto possui autonomia conforme descrito na tabela \ref{tab:project-manager-cost-autonomy}.

\begin{table}[h]
	\centering
	\begin{tabular}{| l | l |}
		\hline
		\textbf{Uso da Reserva Gerencial} & \textbf{Autonomia}                                   \\
		\hline
		Até R\$ 3.000,00                 & Gerente do projeto                                   \\
		\hline
		Até R\$ 7.000,00                 & Gerente do projeto com autorização do patrocinador \\
		\hline
		Acima de R\$ 7.000,00             & Apenas o patrocinador                                \\
		\hline
	\end{tabular}
	\caption{Autonomias do gerente de projetos sobre a reserva gerencial.}
	\label{tab:project-manager-cost-autonomy}
\end{table}

A reserva gerencial pode ser utilizada de forma cumulativa. Quando for atingido o limite de 50\% de uso da reserva gerencial o gerente do projeto deve informar o patrocinador.

\section{Alocação financeira das mudanças no orçamento}

Conforme descrito no plano de gerenciamento das mudanças (ver capítulo ch:change-management-plan), a reserva gerencial pode ser utilizada caso o gerente do projeto necessite realizar mudanças de orçamento de prioridade média.

Todas as mudanças no orçamento do projeto deverão ser classificadas de acordo com o modelo de priorização integrada de mudanças (ver seção \ref{sec:integrated-change-priorization}), além disso devem seguir conforme o fluxo apresentado pelo sistema de controle integrado das mudanças (ver seção \ref{sec:change-control-system}).

\section{Administração do plano de gerenciamento de custos}

\subsection{Responsável pelo plano}

\begin{itemize}
	\item \projectManagerName{}, gerente de projeto, será o responsável direto pelo plano de gerenciamento do tempo.
\end{itemize}

\subsection{Frequência de atualização do plano de gerenciamento de custos}

O plano de gerenciamento dos custos será reavaliado semanalmente durante a reunião do CCM, juntamente com os outros planos de gerenciamento do projeto.

\section{Outros assuntos relacionados ao gerenciamento de custos do projeto não previstos neste plano}

Solicitações não previstas neste plano deverão passar pela aprovação do CCM. Após aprovada o plano deve ser atualizado pelo gerente do projeto.

\section{Controle de versão}

\begin{table}[H]
	\begin{tabularx}{\textwidth}{| c | c | X | X |}
		\hline
		\textbf{Versão} & \textbf{Data} & \textbf{Autor}        & \textbf{Notas de Revisão} \\
		\hline
		1                &               & \projectManagerName{} & Criação do documento     \\
		\hline
	\end{tabularx}
	\centering
\end{table}

\section{Aprovações}

\begin{table}[H]
	\begin{tabularx}{\textwidth}{| c | c | X | c |}
		\hline
		\textbf{Função}  & \textbf{Nome}         & \textbf{Assinatura}        & \textbf{Data} \\
		\hline
		Patrocinador       & \projectSponsorName{} & \projectSponsorSignature{} &               \\
		\hline
		Gerente de projeto & \projectManagerName{} & \projectManagerSignature{} &               \\
		\hline
	\end{tabularx}
	\centering
\end{table}

