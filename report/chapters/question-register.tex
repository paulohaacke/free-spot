\begin{landscape}

\chapter{Registro das questões}
\label{ch:question-register}

A tabela \ref{tab:question-register} deverá ser utilizada para registrar questões e problemas do projeto.

\begin{longtable}{>{\columncolor{gray!10}}>{\centering\arraybackslash}p{0.14\textwidth} >{\centering\arraybackslash}p{0.14\textwidth} >{\columncolor{gray!10}}>{\centering\arraybackslash}p{0.14\textwidth} >{\centering\arraybackslash}p{0.14\textwidth} >{\columncolor{gray!10}}>{\centering\arraybackslash}p{0.14\textwidth} >{\centering\arraybackslash}p{0.14\textwidth} >{\columncolor{gray!10}}>{\centering\arraybackslash}p{0.14\textwidth} >{\centering\arraybackslash}p{0.18\textwidth} >{\columncolor{gray!10}}>{\centering\arraybackslash}p{0.16\textwidth}}
	\toprule
	\textbf{Quem Identificou} & \textbf{Data} & \textbf{Descrição} & \textbf{Tipo} & \textbf{Responsável} & \textbf{Situação} & \textbf{Ação Requerida} & \textbf{Data de Resolução Planejada} & \textbf{Comentários} \\
	\hline
	\endhead
	\multicolumn{5}{c}{{\textit{Continua na próxima página.}}} \\
	\caption{Tabela para registro das questões do projeto.}
	\endfoot
	\endlastfoot
    &&&&&&&&\\
    \hline
    &&&&&&&&\\
    \hline
    &&&&&&&&\\
    \hline
    &&&&&&&&\\
    \hline
    &&&&&&&&\\
    \hline
    &&&&&&&&\\
    \hline
    &&&&&&&&\\
    \hline
    &&&&&&&&\\
    \hline
    &&&&&&&&\\
    \hline
    &&&&&&&&\\
    \hline
    &&&&&&&&\\
    \hline
    &&&&&&&&\\
    \bottomrule
	\caption{Tabela para registro das questões do projeto.}
    \label{tab:question-register}
	\centering
\end{longtable}

\end{landscape}