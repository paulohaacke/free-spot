%Descrição do processo de garantia da qualidade do projeto e do produto desenvolvido incluindo:
%   Padrões de referência a serem utilizados/ check lists
%   Processos
%   Auditorias
%   Ações em caso de Não Conformidade
%   Ferramentas da qualidade
%Identificação dos indicadores de qualidade do projeto 
%Descrição do processo de verificação da qualidade do projeto
%   Métodos de verificação
%   Frequência das ações de controle
%   Eventos listados na Matriz de Comunicações
%   Remete ao Controle Integrado de Mudanças em caso de divergência planejado x realizado
%   Plano de ações corretivas e preventivas
%Descrição do processo de verificação da qualidade do produto
%   Métodos de verificação
%   Frequência das ações de controle
%   Eventos listados na Matriz de Comunicações
%   Remete ao Controle Integrado de Mudanças em caso de divergência planejado x realizado 
%   Plano de ações corretivas e preventivas

\chapter{Plano de gerenciamento da qualidade}

\section{Responsabilidades na garantia da qualidade}

\todo[inline,color=red]{Terminar definição dos responsáveis pela garantia da qualidade.}

\begin{table}[H]
	\begin{tabularx}{\textwidth}{| c | X |}
		\hline
		\textbf{Papel}     & \textbf{Responsabilidade}                            \\
		\hline
		Gerente de projeto & Especificar o cronograma do projeto incluindo testes \\
		\hline
	\end{tabularx}
	\centering
    \caption{Tabela de responsabilidades na garantia da qualidade.}
\end{table}

\section{Responsabilidades no controle da qualidade}

\todo[inline,color=red]{Terminar definição dos responsáveis pelo controle da qualidade.}

\begin{table}[H]
	\begin{tabularx}{\textwidth}{| c | X |}
		\hline
		\textbf{Papel}     & \textbf{Responsabilidade}                            \\
		\hline
		Gerente de projeto & Especificar o cronograma do projeto incluindo testes \\
		\hline
	\end{tabularx}
	\centering
    \caption{Tabela de responsabilidades no controle da qualidade.}
\end{table}

\section{Processo de gerenciamento da qualidade}

\todo[inline,color=red]{Criar processo para gerenciamento da qualidade.}

\section{Política de gerenciamento da qualidade}

\begin{itemize}
	\item O sistema de controle de mudanças da qualidade deve ser utilizado para verificar e classificar alterações nos requisitos de qualidade.
	\item Mudanças corretivas que impactem no sucesso do proejto serão incorporadas ao plano.
	\item Todas as solicitações de mudanças devem ser realizadas de acordo com o plano de gerenciamento das mudanças (ver capítulo \ref{ch:change-management-plan}).
\end{itemize}

\section{Planejamento de avaliação de qualidade}

\todo[inline,color=red]{Criar planejamento de avaliação da qualidade.}

\section{Planejamento de testes/inspeções}

\todo[inline,color=red]{Criar planejamento de testes e inspeções.}

\section{Frequência de atualização do plano de gerenciamento da qualidade}

Os requisitos de qualidade do projeto devem ser avaliados semanalmente durante a reunião do CCM, prevista no plano de gerenciamento das comunicações.

\section{Alocação financeira das mudanças nos requisitos de qualidade}

As mudanças nos requisitos de qualidade podem ser alocadas dentro das reservas gerenciais do projeto.

Em caso de mudanças prioritárias nos requisitos de qualidade do projeto, quando não existem reservas gerenciais disponíveis, deverá ser acionado o patrocinador.

\section{Priorização das mudanças nos requisitos de qualidade e respostas}

As mudanças nos requisitos de qualidade deverão ser classificadas de acordo com o modelo de priorização integrada de mudanças (ver seção \ref{sec:integrated-change-priorization}).

\section{Sistema de controle de mudanças da qualidade}

Todas as mudanças nos requisitos de qualidade do projeto devem ser tratados segundo o fluxo apresentado pelo sistema de controle integrado das mudanças (ver seção \ref{sec:change-control-system}).

\section{Métricas de qualidade}

As métricas de qualidade encontram-se no apêndice \ref{quality-metrics}.

\section{Administração do plano de gerenciamento da qualidade}

\subsection{Responsável}

\begin{itemize}
	\item \projectManagerName, gerente de projeto, será o responsável direto pelo plano de gerenciamento da qualidade.
\end{itemize}

\subsection{Frequência de atualização}

O plano de gerenciamento da qualidade será reavaliado mensalmente durante a reunião do CCM, juntamente com os outros planos de gerenciamento do projeto.

\section{Outros assuntos relacionados ao gerenciamento da qualidade do projeto não previstos neste plano}

\section{Controle de Versão}

\begin{table}[H]
	\begin{tabularx}{\textwidth}{| c | c | X | X |}
		\hline
		\textbf{Versão} & \textbf{Data} & \textbf{Autor}      & \textbf{Notas de Revisão} \\
		\hline
		1                &               & \projectManagerName & Criação do documento     \\
		\hline
	\end{tabularx}
	\centering
\end{table}

\section{Aprovações}

\begin{table}[H]
	\begin{tabularx}{\textwidth}{| c | c | X | c |}
		\hline
		\textbf{Função}  & \textbf{Nome}       & \textbf{Assinatura}      & \textbf{Data} \\
		\hline
		Patrocinador       & \projectSponsorName & \projectSponsorSignature &               \\
		\hline
		Gerente de projeto & \projectManagerName & \projectManagerSignature &               \\
		\hline
	\end{tabularx}
	\centering
\end{table}