\chapter{Plano de gerenciamento da qualidade}

\section{Descrição dos processos de gerenciamento da qualidade}

\section{Priorização das mudanças nos requisitos de qualidade e respostas}

\section{Requisitos de qualidade}

\section{Padrões de qualidade}

\section{Sistema de controle de mudanças da qualidade}

\section{Frequência de avaliação dos requisitos de qualidade do projeto}

\section{Alocação financeira das mudanças nos requisitos de qualidade}

\section{Administração do plano de gerenciamento da qualidade}

\subsection{Responsável}

\begin{itemize}
	\item \projectManagerName, gerente de projeto, será o responsável direto pelo plano de gerenciamento da qualidade.
\end{itemize}
\todo[inline,color=orange]{Verificar necessidade de adicionar suplente responsável.}

\subsection{Frequência de atualização}

O plano de gerenciamento da qualidade será reavaliado mensalmente durante a reunião do CCM, juntamente com os outros planos de gerenciamento do projeto.

\section{Outros assuntos relacionados ao gerenciamento da qualidade do projeto não previstos neste plano}

\section{Controle de Versão}

\begin{table}[H]
	\begin{tabularx}{\textwidth}{| c | c | X | X |}
		\hline
		\textbf{Versão} & \textbf{Data} & \textbf{Autor}      & \textbf{Notas de Revisão} \\
		\hline
		1                &               & \projectManagerName & Criação do documento     \\
		\hline
	\end{tabularx}
	\centering
\end{table}

\section{Aprovações}

\begin{table}[H]
	\begin{tabularx}{\textwidth}{| c | c | X | c |}
		\hline
		\textbf{Função}  & \textbf{Nome}       & \textbf{Assinatura}      & \textbf{Data} \\
		\hline
		Gerente de projeto & \projectManagerName & \projectManagerSignature &               \\
		\hline
	\end{tabularx}
	\centering
\end{table}

\todo[inline,color=red]{Criar plano da qualidade.}