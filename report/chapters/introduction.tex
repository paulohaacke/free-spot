\part{Apresentação do projeto}

\chapter[Introdução]{Introdução}

Guia PMBOK ... \cite{project2013guia}, apresentação sucinta da empresa onde será desenvolvido o projeto, com ramo de atividade, porte, localização, missão, visão e valores.

Hoje em dia muitos estacionamentos privados ainda operam sem qualquer sistema computadorizado. Eles precisam de empregados para verificar manualmente a ocupação de vagas individualmente. Os donos destes locais estão preocupados que o gerenciamento ineficiente das vagas de estacionamento pode estar afetando seus lucros. Enquanto algumas garagens já monitoram a ocupação usando um sistema de sensores que contabiliza quantos carros entram e saem do estacionamento, ainda há espaço para melhorias em termos de automação e eficiência.

Um dos grandes problemas encontrados em estacionamentos privados é o congestionamento dentro da garagem causado por motoristas em busca de uma vaga para estacionar. Em casos que o cliente não encontra uma vaga livre e desiste de continuar a procura, saindo do estacionamento, é provável que ele evite usar os serviços deste estacionamento em outro momento no futuro. Isto porque se ele ao menos soubesse que não iria encontrar uma vaga, ele não teria perdido seu tempo buscando uma vaga que não existia. Este tipo de ocorrência pode ter uma impacto enorme na empresa de estacionamento, visto que a retenção de clientes é um ponto de extrema importância para uma empresa alcançar o sucesso. Ao sair do estacionamento os clientes precisam pagar a taxa de uso. Isto poderia resultar em congestionamento na saída do estacionamento e atrasar a saída dos clientes. Se estes problemas puderem ser gerenciados eficientemente tanto a satisfação do cliente quanto a ocupação da garagem poderiam aumentar.

Outra maneira de aumentar a ocupação em estacionamentos privados é usar estratégias de preço que atraiam mais clientes, nesse contexto o histórico de ocupação é muito importante. Para que o uso deste tipo de estratégia seja possível o estacionamento precisa ter disponível dados com precisão suficiente para que os níveis de preços possam ser determinados de acordo com o volume de ocupação.

Neste contexto, percebendo as \"dores\" de consumidores e donos de estacionamentos privados, surgiu a idéia de criar uma solução que atendesse a maioria destes problemas. Decidiu-se lidar com estes problemas em duas frentes: uma solução de software, tanto para o consumidor quanto para os estacionamentos, em conjunto com uma solução de hardware (sensores, câmeras e infraestrutura). A idéia inicial do negócio é oferecer o hardware e infraestrutura a preço de custo, em contrapartida o retorno sobre o investimento será obtido através de uma mensalidade no uso do software e/ou através de um valor para anunciar o estacionamento na plataforma que será utilizada pelo cliente.

A \investorCompanyName\ acreditou na idéia e investiu \ventureBudget\ para o aporte inicial necessário. Através deste investimento criou-se a empresa \startupCompanyName, que tem por objetivo a criação do projeto de mesmo nome, e a inserção deste projeto no mercado.

\ceoName, CEO da recém criada empresa \startupCompanyName\, disponibilizou \maximumBudget\ e alocou o gerente de projeto \projectManagerName\ para gerenciar o projeto que envolve a criação da solução de software e hardware, que devem estar prontos até a data de lançamento do produto em \maximumDeadline.

\section{Planejamento estratégico}

\todo[inline,color=orange]{Criar planejamento estratégico.}
Apresentação dos fundamentos estratégicos da organização com explanação sobre o enquadramento do projeto alvo neste cenário.

\section{Projeto}

O projeto consiste na criação de uma solução de software que envolve diferentes interfaces para dois diferentes usuários alvos: o cliente e o dono de estacionamentos privados. É imprescindível que as soluções de software funcionem de maneira genérica e independente de características específicas para cada estacionamento. 

Para o cliente será criado um aplicativo para celular onde o mesmo poderá buscar estacionamentos vagos nas proximidades, reservar vagas e gerenciar o uso de estacionamentos.

O dono do estacionamento terá acesso a um aplicativo web, onde poderá gerenciar suas vagas, a ocupação do estacionamento, o histórico de ocupação, o cadastro de clientes, a estratégia de preços e a oferta de seus produtos na plataforma.

Este projeto será responsável também por especificar dois tipos de plano para disponibilização de sensores, câmeras e infraestrutura, uma plano contendo a menor quantidade de equipamento necessária para utilizar a solução de software, e um outro plano com a solução que proporcione o nível de precisão óptimo para gerenciar a ocupação do estacionameto. Esta parte projeto deve detalhar os equipamentos e custos de cada plano.

Um ponto importante é que alguns estacionamentos já possuem soluções de sensores em seus estacionamentos, para estes casos é preciso prever o custo de integração com os sensores já existentes.

Este projeto não inclui a implantação da solução ou estratégias de marketing, está contido apenas o desenvolvimento do software e a especificação do sistema de sensores.