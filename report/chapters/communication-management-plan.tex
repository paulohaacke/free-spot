% Descrição de como as Comunicações serão gerenciadas e controladas
% Identificação dos canais de comunicação e sua respectiva análise de complexidade do plano de comunicação do projeto
% Descrever os principais eventos e canais de comunicação envolvidos no projeto (matriz de comunicação)
% Apresentar o modelo dos principais documentos de comunicação empregados no projeto
% Descrição do processo de gerenciamento das comunicações projeto
% Descrição do processo de controle das comunicações projeto
%   Métodos de controle
%   Frequência das ações de controle
%   Eventos listados na própria Matriz de Comunicações
%   Remete ao Controle Integrado de Mudanças em caso de divergência planejado x realizado 

\chapter{Plano de gerenciamento das comunicações}
\label{ch:communication-management-plan}

\section{Descrição dos processos de gerenciamento das comunicações}

\todo[inline,color=red]{Finalizar processos para gerenciamento das comunicações.}

\begin{itemize}
\item Para realizar o gerenciamento das comunicações serão utilizados processos de comunicação formal, incluídos nesta categoria estão:
\begin{itemize}
\item e-mail,
\item publicações web,
\item memorandos,
\item documentos impressos,
\item reuniões com ata lavrada.
\end{itemize}
\item 
\end{itemize}

\section{Eventos de comunicação}

O projeto terá os seguintes eventos de comunicação:

\begin{description}
\item Reunião de \foreign{kick-off}
\item Reunião do CCM
\item Reunião de encerramento do projeto
\end{description}

\todo[inline,color=red]{Revisar eventos de comunicação.}

\section{Atas de reunião}

Todos os eventos do projeto, com exceção da reunião de \foreign{kick-off}, deverão apresentar ata de reunião contendo, no mínimo, os seguintes dados:

\begin{itemize}
\item Lista de presença.
\item Pauta.
\item Pendências não solucionadas.
\item Aprovações.
\item Decisões tomadas.
\end{itemize}

\todo[inline,color=red]{Revisar atas de reunião.}

\section{Relatórios do projeto}

O gerente do projeto deverá gerar um relatório de enviar semanalmente, por e-mail, 

\todo[inline,color=red]{Criar relatórios do projeto.}

\section{Ambiente técnico e estrutura de armazenamento e distribuição da informação}



\todo[inline,color=red]{Criar seção de ambiente técnico e estrutura de armazenamento para distribuição da infomarção.}

\section{Alocação financeira para o gerenciamento das comunicações}

\todo[inline,color=red]{Criar método para alocação financeira no ger das comunicações.}

\section{Administração do plano de gerenciamento das comunicações}

\subsection{Responsável pelo plano}

\begin{itemize}
	\item \projectManagerName{}, gerente de projeto, será o responsável direto pelo plano de gerenciamento das comunicações.
\end{itemize}

\subsection{Frequência de atualização do plano de gerenciamento das comunicações}

O plano de gerenciamento das comunicações será reavaliado mensalmente durante a reunião do CCM, juntamente com os outros planos de gerenciamento do projeto.

\section{Outros assuntos relacionados ao gerenciamento das comunicações do projeto não previstos neste plano}

Solicitações não previstas neste plano deverão passar pela aprovação do CCM. Após aprovada o plano deve ser atualizado pelo gerente do projeto.

\section{Controle de versão}

\begin{table}[H]
	\begin{tabularx}{\textwidth}{| c | c | X | X |}
		\hline
		\textbf{Versão} & \textbf{Data} & \textbf{Autor}        & \textbf{Notas de Revisão} \\
		\hline
		1                &               & \projectManagerName{} & Criação do documento     \\
		\hline
	\end{tabularx}
	\centering
\end{table}

\section{Aprovações}

\begin{table}[H]
	\begin{tabularx}{\textwidth}{| c | c | X | c |}
		\hline
		\textbf{Função}  & \textbf{Nome}         & \textbf{Assinatura}        & \textbf{Data} \\
		\hline
		Patrocinador       & \projectSponsorName{} & \projectSponsorSignature{} &               \\
		\hline
		Gerente de projeto & \projectManagerName{} & \projectManagerSignature{} &               \\
		\hline
	\end{tabularx}
	\centering
\end{table}