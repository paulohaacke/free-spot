% Descrição de como as Comunicações serão gerenciadas e controladas
% Identificação dos canais de comunicação e sua respectiva análise de complexidade do plano de comunicação do projeto
% Descrever os principais eventos e canais de comunicação envolvidos no projeto (matriz de comunicação)
% Apresentar o modelo dos principais documentos de comunicação empregados no projeto
% Descrição do processo de gerenciamento das comunicações projeto
% Descrição do processo de controle das comunicações projeto
%   Métodos de controle
%   Frequência das ações de controle
%   Eventos listados na própria Matriz de Comunicações
%   Remete ao Controle Integrado de Mudanças em caso de divergência planejado x realizado 

\chapter{Plano de gerenciamento das comunicações}
\label{ch:communication-management-plan}

\section{Descrição dos processos de gerenciamento das comunicações}

%\todo[inline,color=red]{Finalizar processos para gerenciamento das comunicações.}

\begin{itemize}
	\item Para realizar o gerenciamento das comunicações serão utilizados processos de comunicação formal, incluídos nesta categoria estão:
	      \begin{itemize}
		      \item e-mail.
		      \item relatório de acompanhamento do projeto.
		      \item documentos impressos.
		      \item reuniões com ata lavrada.
	      \end{itemize}
	\item Todas as informações do projeto devem ser mantidas atualizadas e disponíveis através de uma pasta específica com controle de versão, onde são armazenados os documentos relacionados aos projetos ativos da empresa.
	\item Todas as solicitações de mudança no processo de comunicação deverão utilizar o formulário de requisição de mudança (ver apêndice \ref{ch:change-request-form}). A solicitação de mudança deverá ser submetida por escrita ou através de e-mail direcionado ao gerente do projeto.
	\item Mudanças relacionadas a comunicação deverão ser tratadas conforme o plano de gerenciamento das mudanças, apresentado no capítulo \ref{ch:change-management-plan}.
\end{itemize}

\section{Eventos de comunicação}

A seguir estão descritos os eventos de comunicação do projeto.

\subsection{Reunião de \foreign{kick-off}}

\begin{description}
	\item[Objetivo] Reunião representando o início do projeto, onde serão expostos o objetivo e a importância do projeto para a empresa. Também serão apresentados as principais entregas, os elementos de alto nível da EAP, os riscos, os custos e os prazos. Além disso, é objetivo desta reunião motivar e integrar a equipe do projeto. 
	\item[Metodologia] Reunião com os integrantes da equipe e apresentação em projetor.
	\item[Responsável] Gerente do projeto.
	\item[Envolvidos] Todos os membros da equipe do projeto, patrocinador e convidados.
	\item[Data e horário] Dia 21/04/2017 às 10:30.
	\item[Duração] 1 hora.
	\item[Local] Sala de reuniões principal.
	\item[Outros] Não se aplica.
\end{description}

\subsection{Reunião para apresentação do escopo e de planos do projeto}

\begin{description}
	\item[Objetivo] Reunião para apresentação do escopo assim como dos planos de projeto ainda não aprovados para o patrocinador. Baseado nessa reunião o patrocinador deve aprovar os planos e escopo do projeto.
	\item[Metodologia] Reunião entre gerente do projeto baseado na utilização de projeto para apresentação.
	\item[Responsável] Gerente do projeto.
	\item[Envolvidos] Patrocinador e gerente do projeto.
	\item[Data e horário] Dia 05/06/2017 às 14:30.
	\item[Duração] 1 hora e 30 minutos.
	\item[Local] Sala do patrocinador do projeto.
	\item[Outros] Ata de reunião requerida.
\end{description}

\subsection{Reunião do CCM}

\begin{description}
	\item[Objetivo] Discutir, analisar e aprovar as solicitações de mudança. Avaliar o prazo, escopo, riscos, processo de aquisição e orçamento do projeto. Reavaliar e atualizar planos do projeto. 
	\item[Metodologia] Reunião com a utilização de computador.
	\item[Responsável] Gerente do projeto.
	\item[Envolvidos] Todos os participantes do CCM.
	\item[Frequência] Semanal.
	\item[Duração] 1 h.
	\item[Local] Sala de reuniões secundária.
	\item[Outros] Ata de reunião requerida.
\end{description}

\subsection{Reunião de lições aprendidas (reunião de encerramento)}

\begin{description}
	\item[Objetivo] Apresentar as lições aprendidas do projeto.
	\item[Metodologia] Reunião com apresentação no projetor para toda equipe, incluindo o patrocinador.
	\item[Responsável] Gerente do projeto.
	\item[Envolvidos] Equipe do projeto, patrocinador e gerente do projeto.
	\item[Duração] 2 horas.
	\item[Local] Sala de reuniões principal.
	\item[Outros] Ata de reunião requerida.
\end{description}

\subsection{Confraternização com a equipe}

\begin{description}
	\item[Objetivo] Comemorar o encerramento do projeto.
	\item[Metodologia] Confraternização informal.
	\item[Responsável] Gerente do projeto.
	\item[Envolvidos] Equipe do projeto, patrocinador e gerente do projeto.
	\item[Data e horário] Dia 01/01/2018 às 16:00.
	\item[Duração] 4 horas.
	\item[Local] Salão de festas do prédio da sede da empresa. 
	\item[Outros] Não se aplica.
\end{description}

%\todo[inline,color=red]{Revisar eventos de comunicação.}

\section{Atas de reunião}

Todos os eventos do projeto, com exceção da reunião de \foreign{kick-off}, deverão apresentar ata de reunião contendo, no mínimo, os seguintes dados:

\begin{itemize}
	\item Lista de presença.
	\item Pauta.
	\item Pendências não solucionadas.
	\item Aprovações.
	\item Decisões tomadas.
\end{itemize}

%\todo[inline,color=red]{Adicionar modelo de ata para reunião?}

\section{Relatórios do projeto}

O gerente do projeto deverá enviar a cada 15 dias o relatório de acompanhamento do projeto para o patrocinador, conforme modelo descrito no apêndice \ref{ch:status-report}.

\section{Ambiente técnico e estrutura de armazenamento e distribuição da informação}

Todos documentos do projeto deverão estar em um diretório específico com controle de versão, acessível através da rede local com permissões de acesso específicas para os envolvidos no projeto, conforme padrão já estabelecido pela empresa. Por ser um padrão da empresa, a estrutura de armazenamento e de rede necessárias já estão prontas para utilização.

\section{Alocação financeira para o gerenciamento das comunicações}

Caso haja necessidade de despesas adicionais no processo de comunicação, as mesmas deverão ser alocadas dentro das reservas gerenciais do projeto, conforme descrito no plano de gernciamento de custos (ver apêndice \ref{ch:cost-management-plan}).

\section{Administração do plano de gerenciamento das comunicações}

\subsection{Responsável pelo plano}

\begin{itemize}
	\item \projectManagerName{}, gerente de projeto, será o responsável direto pelo plano de gerenciamento das comunicações.
\end{itemize}

\subsection{Frequência de atualização do plano de gerenciamento das comunicações}

O plano de gerenciamento das comunicações será reavaliado semanalmente durante a reunião do CCM, juntamente com os outros planos de gerenciamento do projeto.

\section{Outros assuntos relacionados ao gerenciamento das comunicações do projeto não previstos neste plano}

Solicitações não previstas neste plano deverão passar pela aprovação do CCM. Após aprovada o plano deve ser atualizado pelo gerente do projeto.

\section{Controle de versão}

\begin{table}[H]
	\begin{tabularx}{\textwidth}{| c | c | X | X |}
		\hline
		\textbf{Versão} & \textbf{Data} & \textbf{Autor}        & \textbf{Notas de Revisão} \\
		\hline
		1                &               & \projectManagerName{} & Criação do documento     \\
		\hline
	\end{tabularx}
	\centering
\end{table}

\section{Aprovações}

\begin{table}[H]
	\begin{tabularx}{\textwidth}{| c | c | X | c |}
		\hline
		\textbf{Função}  & \textbf{Nome}         & \textbf{Assinatura}        & \textbf{Data} \\
		\hline
		Patrocinador       & \projectSponsorName{} & \projectSponsorSignature{} &               \\
		\hline
		Gerente de projeto & \projectManagerName{} & \projectManagerSignature{} &               \\
		\hline
	\end{tabularx}
	\centering
\end{table}