% Descrição do processo de aquisições a ser empregado no projeto
% Apresentação da análise de make or buy com justificativas
% Apresentação da Declaração de Trabalho dos itens a serem adquiridos
% Apresentar o fluxo do processo de aquisições
% Apresentação dos modelos de formulários a serem utilizados no processo de aquisição
% Organograma de aquisições
% Processo de identificação e seleção de fornecedores
% Processo de qualificação de propostas
% Identificar os tipos de contratos a serem empregados no projeto
% Apresentar os sistemas preferenciais de garantia
% Apresentação do procedimento de Controle das Aquisições
%   Métodos de controle
%   Frequência das ações de controle
%   Eventos listados na Matriz de Comunicações
%   Remete ao Controle Integrado de Mudanças em caso de divergência planejado x realizado 
% Apresentação do procedimento de Encerramento das Aquisições
 
\chapter{Plano de gerenciamento das aquisições}

\section{Descrição dos processos de gerenciamento das aquisições}

\begin{itemize}
	\item O gerenciamento das aquisições estará focado na contratação de um servidor em nuvem.
	\item 
\end{itemize}

\todo[inline,color=red]{Criar descrição dos processos de aquisição.}

\section{Gerenciamento e tipos de contratos}

\todo[inline,color=red]{Criar gerenciamento e tipos de contratos.}

\section{Critérios de avaliação de cotações e propostas.}

\todo[inline,color=red]{Criar critérios de avaliação de cotações e propostas.}

\section{Avaliação de fornecedores}

\todo[inline,color=red]{Criar método para avaliação de fornecedores.}

\section{Frequência de avaliação dos processos de aquisições}

\todo[inline,color=red]{Descrever frequência de avaliação dos processos de aquisições.}

\section{Alocação financeira para o gerenciamento das aquisições}

\todo[inline,color=red]{Descrever método para aloação financeira do plano de aquisições.}

\section{Administração do plano de gerenciamento das aquisições}

\subsection{Responsável pelo plano}

\begin{itemize}
	\item \projectManagerName{}, gerente de projeto, será o responsável direto pelo plano de gerenciamento das aquisições.
\end{itemize}

\subsection{Frequência de atualização do plano de gerenciamento das aquisições}

O plano de gerenciamento das aquisições será reavaliado mensalmente durante a reunião do CCM, juntamente com os outros planos de gerenciamento do projeto.

\section{Outros assuntos relacionados ao gerenciamento das aquisições do projeto não previstos neste plano}

Solicitações não previstas neste plano deverão passar pela aprovação do CCM. Após aprovada o plano deve ser atualizado pelo gerente do projeto.

\section{Controle de versão}

\begin{table}[H]
	\begin{tabularx}{\textwidth}{| c | c | X | X |}
		\hline
		\textbf{Versão} & \textbf{Data} & \textbf{Autor}        & \textbf{Notas de Revisão} \\
		\hline
		1                &               & \projectManagerName{} & Criação do documento     \\
		\hline
	\end{tabularx}
	\centering
\end{table}

\section{Aprovações}

\begin{table}[H]
	\begin{tabularx}{\textwidth}{| c | c | X | c |}
		\hline
		\textbf{Função}  & \textbf{Nome}         & \textbf{Assinatura}        & \textbf{Data} \\
		\hline
		Patrocinador       & \projectSponsorName{} & \projectSponsorSignature{} &               \\
		\hline
		Gerente de projeto & \projectManagerName{} & \projectManagerSignature{} &               \\
		\hline
	\end{tabularx}
	\centering
\end{table}