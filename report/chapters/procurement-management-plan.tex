% Descrição do processo de aquisições a ser empregado no projeto
% Apresentação da análise de make or buy com justificativas
% Apresentação da Declaração de Trabalho dos itens a serem adquiridos
% Apresentar o fluxo do processo de aquisições
% Apresentação dos modelos de formulários a serem utilizados no processo de aquisição
% Organograma de aquisições
% Processo de identificação e seleção de fornecedores
% Processo de qualificação de propostas
% Identificar os tipos de contratos a serem empregados no projeto
% Apresentar os sistemas preferenciais de garantia
% Apresentação do procedimento de Controle das Aquisições
%   Métodos de controle
%   Frequência das ações de controle
%   Eventos listados na Matriz de Comunicações
%   Remete ao Controle Integrado de Mudanças em caso de divergência planejado x realizado 
% Apresentação do procedimento de Encerramento das Aquisições

\chapter{Plano de gerenciamento das aquisições}

\section{Descrição dos processos de gerenciamento das aquisições}

\begin{itemize}
	\item O gerenciamento das aquisições estará focado na contratação de um servidor em nuvem.
	\item Durante a execução do projeto todos os custos com o servidor em nuvem deverão estar incluídos no orçamento do projeto.
	\item A autonomia sobre os contratos é de exclusiva competência do gerente do projeto, que deve assinar todos os contratos e medições de serviços.
	\item Os aspectos éticos do processo de aquisição serão rigorosamente acompanhados, respeitado os seguintes princípios:
	      \begin{itemize}
		      \item Legalidade.
		      \item Igualdade.
		      \item Publicidade.
		      \item Impessoalidade.
		      \item Imparcialidade.
		      \item Moralidade.
		      \item Probidade administrativa.
		      \item Lealdade à empresa.
	      \end{itemize}
	\item Quaisquer infrações aos aspectos éticos serão consideradas faltas gravíssimas pelo gerente do projeto e pelo patrocinador.
	\item No gerenciamento das aquisições serão considerados apenas as aquisições diretamente relacionadas ao escopo do projeto. Inovações e novos recursos não serão tratados pelo gerenciamento das aquisições.
	\item Quaisquer solicitações de mudança no processo de aquisições ou nos objetos a serem adquiridos devem ser feitas de acordo com plano de gerenciamento das mudanças (ver capítulo \ref{ch:change-management-plan}).
\end{itemize}

\section{Gerenciamento e tipos de contratos}

\begin{itemize}
	\item Todas as cláusulas contratuais devem ser respeitadas rigorosamente, principalmente no que diz respeito a atendimento dos requisitos solicitados.
	\item A elaboração e avaliação de contratos é de responsabilidade da área jurídica da empresa.
	\item Os contratos deste projeto são de tempo e material, visto que os recursos em nuvem são disponibilizados conforme uma cotação fixa e são cobrados de forma proporcional ao seu uso.
\end{itemize}

\section{Critérios de avaliação de cotações e propostas.}

\begin{itemize}
	\item Para toda aquisição deverão ser avaliadas um mínimo de 3 propostas de diferentes fornecedores.
	\item As propostas deverão ser priorizadas e cada fornecedor receberá um pontuação, de acordo com os requisitos desejados, o custo e outros aspectos, conforme tabela \ref{procurement-evaluation-criteria}.
	\item O valor máximo para ser gasto com o servidor em nuvem durante o projeto é de \procurementBudget{}.
\end{itemize}

\begin{longtable}{ll}
	\toprule
	\thead[c]{Critério de Avaliação} & \thead[c]{Peso} \\
	\midrule
	Custo                               & 30\%            \\
	\midrule
	Atendimento aos requisitos          & 50\%            \\
	\midrule
	Suporte                             & 10\%            \\
	\midrule
	Condição de Pagamento             & 10\%            \\
	\bottomrule
	\caption{Critérios para avaliação de fornecedores.}
	\label{procurement-evaluation-criteria}
	\centering
\end{longtable}

\section{Avaliação de fornecedores}

O gerente do projeto deve acompanhar, próximo a equipe do projeto, se a qualidade do serviço oferecido está de acordo com os níveis de serviço acordados.

Em caso de não cumprimento de itens do contrato, devem ser tomadas as seguintes medidas:

\begin{description}
\item[Advertência] Em casos pouco graves e que não comprometam o andamento do projeto.
\item[Suspensão] Em casos de desvios de gravidade média que comprometam parcialmente o andamento do projeto, ou para fornecedores já advertidos.
\item[Cancelamento] Em casos graves que comprometam o projeto e que tornem necessário a intervenção do gerente do projeto e do patrocinador.
\end{description}

\section{Frequência de avaliação dos processos de aquisições}

Os processos de aquisição serão avaliados semanalmente durante a reunião do CCM, prevista no plano de gerenciamento das comunicações (ver capítulo \ref{ch:communication-management-plan}).

\section{Alocação financeira para o gerenciamento das aquisições}

Necessidades de aquisições não previstas no orçamento deve ser alocada dentro das reservas gerenciais do projeto, conforme plano de gerenciamento dos custos (ver capítulo \ref{ch:cost-management-plan}).

\section{Administração do plano de gerenciamento das aquisições}

\subsection{Responsável pelo plano}

\begin{itemize}
	\item \projectManagerName{}, gerente de projeto, será o responsável direto pelo plano de gerenciamento das aquisições.
\end{itemize}

\subsection{Frequência de atualização do plano de gerenciamento das aquisições}

O plano de gerenciamento das aquisições será reavaliado semanalmente durante a reunião do CCM, juntamente com os outros planos de gerenciamento do projeto.

\section{Outros assuntos relacionados ao gerenciamento das aquisições do projeto não previstos neste plano}

Solicitações não previstas neste plano deverão passar pela aprovação do CCM. Após aprovada o plano deve ser atualizado pelo gerente do projeto.

\section{Controle de versão}

\begin{table}[H]
	\begin{tabularx}{\textwidth}{| c | c | X | X |}
		\hline
		\textbf{Versão} & \textbf{Data} & \textbf{Autor}        & \textbf{Notas de Revisão} \\
		\hline
		1                &               & \projectManagerName{} & Criação do documento     \\
		\hline
	\end{tabularx}
	\centering
\end{table}

\section{Aprovações}

\begin{table}[H]
	\begin{tabularx}{\textwidth}{| c | c | X | c |}
		\hline
		\textbf{Função}  & \textbf{Nome}         & \textbf{Assinatura}        & \textbf{Data} \\
		\hline
		Patrocinador       & \projectSponsorName{} & \projectSponsorSignature{} &               \\
		\hline
		Gerente de projeto & \projectManagerName{} & \projectManagerSignature{} &               \\
		\hline
	\end{tabularx}
	\centering
\end{table}