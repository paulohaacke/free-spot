\chapter{Plano de gerenciamento de escopo}

\section{Objetivo do documento}

O objetivo deste documento é descrever como será gerenciado o projeto \projectname

\section{Descrição dos processos de gerenciamento de escopo}

\begin{itemize}
	\item O gerenciamento do escopo do projeto será realizado com base em 3 documentos: declaração de escopo para o escopo funcional do projeto, EAP para o escopo das atividades a serem realizadas pelo projeto e dicionário da EAP para descrever os pacotes de trabalho.
\end{itemize}

\section{Priorização das mudanças de escopo e respostas}

\section{Gerenciamento de configuração}

\section{Frequência de avaliação do escopo do projeto}

\section{Alocação financeira das mudanças de escopo}

\section{Administração do plano de gerenciamento do escopo}

\section{Outros assuntos relacionados ao gerenciamento do escopo do projeto não previstos neste plano}

\section{Controle de Versão}

\begin{table}[H]
	\begin{tabularx}{.9\textwidth}{| c | c | X | X |}
		\hline
		\textbf{Versão} & \textbf{Data} & \textbf{Autor}      & \textbf{Notas de Revisão} \\
		\hline
		1                &               & Paulo André Haacke & Criação do documento     \\
		\hline
	\end{tabularx}
	\centering
	\caption{Tabela para controle de versão do termo de abertura.}
\end{table}

\section{Aprovações}

\begin{table}[H]
	\begin{tabularx}{\textwidth}{| c | c | X | c |}
		\hline
		\textbf{Função} & \textbf{Nome}       & \textbf{Assinatura}      & \textbf{Data} \\
		\hline
		Patrocinador      & \projectSponsorName & \projectSponsorSignature &               \\
		\hline
	\end{tabularx}
	\centering
\end{table}

\todo[inline,color=red]{Criar plano de gerenciamento do escopo.}