%---Escopo
%Descrição do processo utilizado para Gerenciamento do Escopo
%Descrição dos processos de coleta de requisitos
%	Descrição de como os requisitos serão coletados
%	Frequência/ número de eventos de coleta/ limite para cessar a coleta
%	Eventos listados na Matriz de Comunicações
%Descrição do processo de validação e controle do escopo do projeto
%	Descrição de como os requisitos serão validados e controlados
%	Frequência das validações e controle
%	Eventos listados na Matriz de Comunicações
%	Remete ao Controle Integrado de Mudanças em caso de divergência planejado x realizado
%---


\chapter{Plano de gerenciamento de escopo}

\section{Objetivo do documento}

O objetivo deste documento é descrever como será gerenciado o escopo, descrevendo quais ferramentas, técnicas e artefatos serão utilizados para determinar o que deve ser abordado durante o projeto \projectName.

\section{Descrição dos processos de gerenciamento de escopo}

\begin{itemize}
	\item O gerenciamento do escopo do projeto será realizado com base em 3 documentos: declaração de escopo para o escopo funcional do projeto, EAP para o escopo das atividades a serem realizadas pelo projeto e dicionário da EAP para descrever os pacotes de trabalho.
	\item O escopo inicial será validado em reunião juntamente com o patrocinador do projeto. Sua aprovação será realizada mediante a assinatura do termo de aceite.
	\item O controle do escopo será realizado pelo gerente de projeto utilizando informações de desempenho do trabalho e comparando as mesmas com os documentos gerados na definição do escopo (linha de base do escopo). Deverão ser avaliados os desvios em relação a linha de base de escopo do projeto e identificadas as necessidades de ações corretivas ou preventivas.
	\item Serão consideradas mudanças de escopo apenas as medidas corretivas. Todas as mudanças de escopo devem ser avaliadas e gerenciadas de acordo com o plano de gerenciamento de mudanças (ver capítulo \ref{ch:change-management-plan}).
	\item Todas as mudanças de escopo deverão ser submetidas por escrito ou através de e-mail, conforme descrito no plano de comunicações do projeto.
\end{itemize}

\section{Coletar os requisitos}

Os requisitos deste projeto serão identificados e coletados através das seguintes técnicas e ferramentas:

\begin{description}
	\item[Entrevistas:] serão realizadas entrevistas com pessoas que tenham contato com donos de estacionamentos privados, ou que já tiveram contato com o desenvolvimento de tecnologias para este setor.
	\item[Brainstorming:] será reunida a equipe para fazer uma sessão de brainstorming para melhorar o detalhamento dos possíveis requisitos.
	\item[Questionários e pesquisas:] serão realizados questionários com consumidores e donos de estacionamentos privados para ajudar na identificação dos requisitos.
	\item[Observação:] na busca de melhores requisitos observar o funcionamento de estacionamentos privados e o comportamento de seus consumidores poderá trazer ideías ainda não relatadas.
\end{description}

\section{Frequência para coleta de requisitos}

\todo[inline,color=red]{Frequência para coleta de requisitos.}

\section{Priorização das mudanças de escopo e respostas}

As mudanças no escopo deverão ser classificadas de acordo com o modelo de priorização integrada de mudanças (ver seção \ref{sec:integrated-change-priorization}).

\section{Gerenciamento de configuração}

Todas as mudanças no escopo do projeto devem ser tratadas de acordo com o sistema de controle integrado das mudanças (ver seção \ref{sec:change-control-system}). Seus resultados devem ser apresentados na reunião semanal do CCM com suas conclusões, prioridades e ações relacionadas.

\section{Frequência de avaliação do escopo do projeto}

O escopo deve ser avaliado semanalmente dentro da reunião do CCM, prevista no plano de gerenciamento das comunicações (ver capítulo \ref{ch:communication-management-plan}).

\section{Alocação financeira das mudanças de escopo}

As mudanças de escopo podem ser alocadas dentro das reservas gerenciais do projeto de acordo com as necessidades do gerente de projeto.

Para mudanças de escopo prioritárias, em momentos que não existam mais reservas gerenciais disponíveis, deverá ser acionado o patrocinador, já que o gerente de projeto não possui autonomia para decidir utilizar a reserva de contingência de riscos para mudanças de escopo.

\section{Administração do plano de gerenciamento do escopo}

\subsection{Responsável}

\begin{itemize}
	\item \projectManagerName, gerente de projeto, será o responsável direto pelo plano de gerenciamento de escopo.
\end{itemize}
\todo[inline,color=orange]{Verificar necessidade de adicionar suplente responsável.}

\subsection{Frequência de atualização}

O plano de gerenciamento do escopo será reavaliado mensalmente durante a reunião do CCM, juntamente com os outros planos de gerenciamento do projeto.

\section{Outros assuntos relacionados ao gerenciamento do escopo do projeto não previstos neste plano}

As solicitações não previstas neste plano deverão ser submetidas a reunião do CCM para aprovação.
\todo[inline,color=orange]{Adicionar menção ao plano de mudanças}

\section{Controle de Versão}

\begin{table}[H]
	\begin{tabularx}{\textwidth}{| c | c | X | X |}
		\hline
		\textbf{Versão} & \textbf{Data} & \textbf{Autor}      & \textbf{Notas de Revisão} \\
		\hline
		1                &               & \projectManagerName & Criação do documento     \\
		\hline
	\end{tabularx}
	\centering
\end{table}

\section{Aprovações}

\begin{table}[H]
	\begin{tabularx}{\textwidth}{| c | c | X | c |}
		\hline
		\textbf{Função}  & \textbf{Nome}       & \textbf{Assinatura}      & \textbf{Data} \\
		\hline
		Patrocinador       & \projectSponsorName & \projectSponsorSignature &               \\
		\hline
		Gerente de projeto & \projectManagerName & \projectManagerSignature &               \\
		\hline
	\end{tabularx}
	\centering
\end{table}