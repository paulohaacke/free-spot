\documentclass[
	12pt,				% tamanho da fonte
	openright,			% capítulos começam em pág ímpar (insere página vazia caso preciso)
	twoside,			% para impressão em recto e verso. Oposto a oneside
	a4paper,			% tamanho do papel.
	final,
	% -- opções da classe abntex2 --
	%chapter=TITLE,		% títulos de capítulos convertidos em letras maiúsculas
	%section=TITLE,		% títulos de seções convertidos em letras maiúsculas
	%subsection=TITLE,	% títulos de subseções convertidos em letras maiúsculas
	%subsubsection=TITLE,% títulos de subsubseções convertidos em letras maiúsculas
	% -- opções do pacote babel --
    french,             % idioma adicional para hifenização
	spanish,			% idioma adicional para hifenização
	english,			% idioma adicional para hifenização
	brazil				% o último idioma é o principal do documento
	]{abntex2}

\usepackage{lmodern}
\usepackage[T1]{fontenc}
\usepackage[utf8]{inputenc}
\usepackage{indentfirst}
\usepackage{color}
\usepackage{graphicx}
\usepackage{microtype}

\usepackage[brazilian,hyperpageref]{backref}
\usepackage[alf]{abntex2cite}

\usepackage{amssymb}
\usepackage{multirow}
\usepackage{float}
\usepackage[obeyFinal,colorinlistoftodos,portuguese]{todonotes}
\usepackage{ifdraft}
\usepackage{pdfpages}
\usepackage{booktabs}
\usepackage{longtable,lscape}
\usepackage{pdflscape}
\usepackage{scrhack}
\usepackage{tikz}
\usepackage{colortbl}
\usepackage{xcolor}
\usepackage{makecell}
\usepackage{enumitem}
\def\theadfont{\bfseries}

% Makes figure count by section
%\usepackage{chngcntr}
%\counterwithin{figure}{chapter}

\usepackage{styles/vagalivre}

\usetikzlibrary{shapes.geometric, arrows}
\tikzstyle{startstop} = [rectangle, draw=black, fill=red!30, text width=5em, text centered, rounded corners, minimum height=2em]
\tikzstyle{io} = [trapezium, trapezium left angle=70, trapezium right angle=110, minimum width=3cm, minimum height=1cm, text centered, draw=black, fill=blue!30]
\tikzstyle{process} = [rectangle, draw, fill=blue!20, text width=8em, text centered, rounded corners, minimum height=4em]
\tikzstyle{decision} = [diamond, draw, fill=blue!20, text width=6.5em, text badly centered, inner sep=0pt]
\tikzstyle{arrow} = [thick,->,>=stealth]
\tikzstyle{line} = [draw, thick, ->, >=stealth, -latex']
\tikzstyle{simpleline} = [draw, thick, >=stealth]
\tikzstyle{wbsblock} = [rectangle, draw, fill=blue!20, text width=7em, text centered, rounded corners, minimum height=4em]

\addtocontents{toc}{\protect\setcounter{tocdepth}{1}}

\newcommand*\rot{\rotatebox{90}}
\newcommand\foreign[1]{\emph{#1}}

\renewcommand{\backrefpagesname}{Citado na(s) página(s):~}
\renewcommand{\backref}{}
\renewcommand*{\backrefalt}[4]{
	\ifcase #1 %
		Nenhuma citação no texto.%
	\or
		Citado na página #2.%
	\else
		Citado #1 vezes nas páginas #2.%
	\fi}%

% ---
% Cover and front pages information
% ---
\titulo{\projectName{}: Planejamento do Projeto}
\autor{Paulo André Haacke}
\local{Porto Alegre}
\data{\today}
\orientador{Prof. Me. Cristiano Tonietto Galina}
%\coorientador{Nenhum}
\instituicao{%
  Pontifícia Universidade Católica do Rio Grande do Sul -- PUCRS
  \par
  Faculdade de Informática
  \par
  Especialização em Gerenciamento de Projetos com Ênfase em Tecnologia da Informação}
\tipotrabalho{Trabalho de Conclusão de Curso (Pós-Graduação)}
% O preambulo deve conter o tipo do trabalho, o objetivo, 
% o nome da instituição e a área de concentração 
\preambulo{Trabalho de conclusão de curso de pós-graduação apresentado à Faculdade de Informática da Pontifícia Universidade Católica do Rio Grande do Sul, como requisito parcial para obtenção do grau de Especialista em Gerenciamento de Projetos com ênfase em Tecnologia da Informação.}
% ---

% ---
% PDF appearence configurations

\definecolor{blue}{RGB}{41,5,195}

% PDF information
\makeatletter
\hypersetup{
     	%pagebackref=true,
		pdftitle={\@title}, 
		pdfauthor={\@author},
    	pdfsubject={\imprimirpreambulo},
	    pdfcreator={LaTeX with abnTeX2},
        pdfkeywords={project management}{gerenciamento de projetos}{tecnologia da informação}{estacionamento privado}{automação}, 
		colorlinks=true,       		% false: boxed links; true: colored links
    	linkcolor=blue,          	% color of internal links
    	citecolor=blue,        		% color of links to bibliography
    	filecolor=magenta,      		% color of file links
		urlcolor=blue,
		bookmarksdepth=4
}
\makeatother
% ---

% --- 
% Line and paragraph space
% --- 

% Paragraph size
\setlength{\parindent}{1.3cm}

% Space between paragraphs
\setlength{\parskip}{0.2cm}  % try also \onelineskip


% ---
% Compile index
% ---
\makeindex
% ---

\begin{document}

% Select document language (according to babel packages)
%\selectlanguage{english}
\selectlanguage{brazil}

% Remove extra space between phrases
\frenchspacing

% ----------------------------------------------------------
% PRETEXTUAL ELEMENTS
% ----------------------------------------------------------
%\pretextual
\listoftodos
\clearpage

% ---
% Cover
% ---
\imprimircapa
% ---

% ---
% Front page
% (The * imply in the existence of a bibliography)
% ---
\imprimirfolhaderosto*
% ---

% ---
% Cataloguing data
% ---

% Isto é um exemplo de Ficha Catalográfica, ou ``Dados internacionais de
% catalogação-na-publicação''. Você pode utilizar este modelo como referência. 
% Porém, provavelmente a biblioteca da sua universidade lhe fornecerá um PDF
% com a ficha catalográfica definitiva após a defesa do trabalho. Quando estiver
% com o documento, salve-o como PDF no diretório do seu projeto e substitua todo
% o conteúdo de implementação deste arquivo pelo comando abaixo:
%
% \begin{fichacatalografica}
%     \includepdf{fig_ficha_catalografica.pdf}
% \end{fichacatalografica}

\begin{fichacatalografica}
	\sffamily
	\vspace*{\fill}					% Posição vertical
	\begin{center}					% Minipage Centralizado
		\fbox{\begin{minipage}[c][8cm]{13.5cm}		% Largura
				\small
				\imprimirautor
				%Sobrenome, Nome do autor

				\hspace{0.5cm} \imprimirtitulo  / \imprimirautor. --
				\imprimirlocal, \imprimirdata-

				\hspace{0.5cm} \pageref{LastPage} p. : il. (algumas color.) ; 30 cm.\\

				\hspace{0.5cm} \imprimirorientadorRotulo~\imprimirorientador\\

				\hspace{0.5cm}
				\parbox[t]{\textwidth}{\imprimirtipotrabalho~--~\imprimirinstituicao,
					\imprimirdata.}\\

				\hspace{0.5cm}
				1. Palavra-chave1.
				2. Palavra-chave2.
				2. Palavra-chave3.
				I. Orientador.
				II. Universidade xxx.
				III. Faculdade de xxx.
				IV. Título
			\end{minipage}}
	\end{center}
\end{fichacatalografica}
% ---

% ---
% Approval page
% ---

% Isto é um exemplo de Folha de aprovação, elemento obrigatório da NBR
% 14724/2011 (seção 4.2.1.3). Você pode utilizar este modelo até a aprovação
% do trabalho. Após isso, substitua todo o conteúdo deste arquivo por uma
% imagem da página assinada pela banca com o comando abaixo:
%
% \includepdf{folhadeaprovacao_final.pdf}
%
\begin{folhadeaprovacao}

	\begin{center}
		{\ABNTEXchapterfont\large\imprimirautor}

		\vspace*{\fill}\vspace*{\fill}
		\begin{center}
			\ABNTEXchapterfont\bfseries\Large\imprimirtitulo
		\end{center}
		\vspace*{\fill}

		\hspace{.45\textwidth}
		\begin{minipage}{.5\textwidth}
			\imprimirpreambulo
		\end{minipage}%
		\vspace*{\fill}
	\end{center}

	Trabalho aprovado. \imprimirlocal, 02 de agosto de 2017:

	\assinatura{\textbf{\imprimirorientador} \\ Orientador}
	\assinatura{\textbf{Prof. Osmar A. M. Pedrozo} \\ Avaliador}
	\assinatura{\textbf{Professor} \\ Avaliador}
	%\assinatura{\textbf{Professor} \\ Convidado 3}
	%\assinatura{\textbf{Professor} \\ Convidado 4}

	\begin{center}
		\vspace*{0.5cm}
		{\large\imprimirlocal}
		\par
		{\large\imprimirdata}
		\vspace*{1cm}
	\end{center}

\end{folhadeaprovacao}
% ---

% ---
% Dedication
% ---
\begin{dedicatoria}
	\vspace*{\fill}
	\centering
	\noindent
	\textit{ Este trabalho é dedicado à meus pais, Ilvo e Liane, meu irmão, Anderson, e minha namorada, Larissa.} \vspace*{\fill}
\end{dedicatoria}
% ---

% ---
% Acknowledgment
% ---
\begin{agradecimentos}
	Agradeço ao meu orientador, Cristiano Galina, pelos seus conselhos e apoio durante o desenvolvimento do trabalho de conclusão.

	Agradeço a meus pais e meu irmão pelo apoio e compreensão sempre ofertados de forma gratuita, me ajudando a superar mais esse desafio.

	Agradeço a minha namorada pela compreensão e incentivo, pois sem estes eu não chegaria aqui.

	Agradeço a todos meus amigos que me apoiaram, mesmo que indiretamente.

	Agradeço aos colegas e professores que conheci durante o curso, através dos quais pude não só adquirir conhecimento, mas também evoluir como profissional.

\end{agradecimentos}
% ---

% ---
% Epigraph
% ---
\begin{epigrafe}
	\vspace*{\fill}
	\begin{flushright}
		\textit{``To be honest, the fact that people trust \\
				  you gives you a lot of power over people. \\
		Having another person's trust is more powerful than\\ 
		 all other management techniques put together.'' \\
			(Linus Torvalds, 2004)}
		% Sugestoes:
		% - "Não há nada permanente, exceto as mudanças." Heráclito, 500 a.C.
		% - "Lost time is never found again." Benjamin Franklin
		% - "Se eu tivesse oito horas para derrubar uma árvore, passaria seis afiando meu machado." Abraham Lincoln
		% - "To be honest, the fact that people trust you gives you a lot of power over people. Having another person's trust is more powerful than all other management techniques put together." - Linus Torvalds
		%
	\end{flushright}
\end{epigrafe}
% ---

% ---
% ABSTRACTS
% ---

% brazilian portuguese abstract
\setlength{\absparsep}{18pt} % ajusta o espaçamento dos parágrafos do resumo
\begin{resumo}

Ao observar e conversar com donos de estacionamentos privados, assim como com motoristas, percebeu-se dois grandes problemas enfrentados por estes grupos. Os donos e gerentes de estacionamentos privados reportaram a necessidade de aumentar a eficiência e diminuir custos em seus negócios. Os motoristas apontam o tempo gasto com a busca de vagas de garagem e a falta de possibilidade de reservar um vaga como os principais motivos para a insatisfação com o uso de serviços oferecidos por estacionamentos privados. 

Observando-se estes problemas, percebeu-se que eles complementam um ao outro, e para resolver o problema de um é preciso resolver o problema do outro. Este fato levou à percepção de que a criação de um espécie de leilão de vagas para estacionamentos pode ser uma boa oportunidade de negócio. O plano de projeto para implementação desta idéia é o proposto neste trabalho. 

A solução encontrada aborda a implementação de dois aplicativos. Um para ser utilizado pelo motorista, onde é possível realizar reservas e encontrar estacionamentos vagos. O outro aplicativo foca no dono ou gerente do estacionamento privado, e permite verificar a ocupação do estacionamento, seus horários de pico, controlar entrada e saída de clientes, gerar relatórios de tendências, entre outras funcionalidades.

Este plano de projeto foi baseado nos conhecimentos propostos na quinta edição do \cite{project2013guia}. A organização deste trabalho segue conforme os grupos de processos e as áreas de conhecimento para cada documento criado. O plano foca em apresentar e descrever os processos utilizados durante o gerenciamento deste projeto.

 \textbf{Palavras-chave}: vaga-livre, estacionamento privado, plano de projeto, PMBOK.
\end{resumo}

% english abstract
\begin{resumo}[Abstract]
 \begin{otherlanguage*}{english}
   This is the english abstract.
   \todo[inline,color=green]{Criar abstract.}

   \vspace{\onelineskip}
 
   \noindent 
   \textbf{Keywords}: latex. abntex. text editoration.
 \end{otherlanguage*}
\end{resumo}

% ---
% illustrations list
% ---
\pdfbookmark[0]{\listfigurename}{lof}
\listoffigures*
\cleardoublepage
% ---

% ---
% tables list
% ---
\pdfbookmark[0]{\listtablename}{lot}
\listoftables*
\cleardoublepage
% ---

% ---
% Abbreviations and acronyms list
% ---
\begin{siglas}
	\item[CCM] Conselho de Controle de Mudanças
	\item[CR] Custo Real
	\item[DETRAN] Departamento Estadual de Trânsito
	\item[EAP] Estrutura Analítica de Projetos
	\item[EAR] Estrutura Analítica de Riscos
	\item[GP] Gerente de Projeto
	\item[IDC] Índice de Desempenho de Custo
	\item[IDP] Índice de Desempenho de Prazo
	\item[PERT] \foreign{Program, Evaluation and Review Technique} (Técnica de Programa, Avaliação e Revisão)
	\item[PMBOK] \foreign{Project Management Book of Knowledge} (Um Guia do Conhecimento em Gerenciamento de Projetos)
	\item[PMI] \foreign{Project Management Institute} (Instituto de Gerenciamento de Projetos)
	\item[RACI] \foreign{Responsible, Accountable, Consulted, Informed} (Responsável, Aprovador, Consultado, Informado)
	\item[VME] Valor Monetário Esperado
	 
\end{siglas}
% ---

% ---
% Symbols list
% ---
%\begin{simbolos}
%	\item[$ \Gamma $] Letra grega Gama
%	\item[$ \Lambda $] Lambda
%	\item[$ \zeta $] Letra grega minúscula zeta
%	\item[$ \in $] Pertence
%\end{simbolos}
% ---

% ---
% Summary
% ---
\pdfbookmark[0]{\contentsname}{toc}
\tableofcontents*
\cleardoublepage
% ---

% ----------------------------------------------------------
% TEXTUAL ELEMENTS
% ----------------------------------------------------------
\textual

% ----------------------------------------------------------
% Introdução (exemplo de capítulo sem numeração, mas presente no Sumário)
% ----------------------------------------------------------
\chapter*[Introdução]{Introdução}
\addcontentsline{toc}{chapter}{Introdução}
% ----------------------------------------------------------

Teste

\part{Processos de Iniciação}

\section{Termo de Abertura}

\subsection{Nome do Projeto}

Vaga Livre

\subsection{Patrocinador}

\projectSponsorName

%Poderia colocar a minha empresa ou uma empresa contratante, provisoriamente deixei como sendo minha empresa
%Nome e autoridade do patrocinador ou outra(s) pessoa(s) que autoriza(m) o termo de abertura do projeto.

\subsection{Gerente do Projeto}

\projectManagerName

%Gerente do projeto, responsabilidade, nível de autoridade designados.

\subsection{Descrição}

Habilitar gerentes de estacionamentos privados a controlar a ocupação de vagas em seu estacionamento, incluindo visualizar o histórico de ocupação e demanda atual, oferecer um serviço de reserva com rígido controle.
O consumidor será beneficiado através de um aplicativo móvel, onde poderá realizar reservas e encontrar vagas em estacionamentos diversos.

%Descrição de alto nível do projeto e seus limites.

\subsection{Justificativa}

O mercado de estacionamentos privados tem evoluído muito no Brasil e no Mundo, essa evolução trouxe grandes benefícios para o usuário, entretanto ainda existe demanda para a melhoria nos serviços em áres pouco exploradas, como a busca de vagas e o congestionamento das cidades.
Por outro lado estacionamentos privados encontram dificuldades em alocar de forma eficiente seus estacionamentos.
\todo[inline,color=red]{Melhorar descrição da oportunidade de mercado.}
Através do desenvolvimento deste sistema pretende-se:
\begin{itemize}
	\item Maximizar a ocupação dos estacionamentos;% O que significa isso?
	\item Maximizar o lucro;
	\item Atingir objetivos estratégicos;
	\item Desenvolver estratégias de marketing de acordo com ocupação em períodos anteriores;
	\item Oferecer um melhor serviço aos seus consumidores: permitindo a reserva e busca de vagas.
\end{itemize}

%Finalidade ou justificativa do projeto.

\subsection{Objetivos}

%%Diminuir o tempo médio de espera em fila para 10 minutos. Diminuir o número de vagas não ocupadas nos piores horários para 50.

\begin{itemize}
	\item Desenvolver um aplicativo de celular portável tanto para Android quanto para IOS para encontrar e reservar vagas de estacionamento;
	\item Desenvolver um software que permita aos estacionamento privados controlar a demanda e ocupação de seus estacionamentos;
	\item O aplicativo e o software devem estar prontos até \maximumDeadline;
	\item O aplicativo deve suportar até \minimumUsersAmount usuários;
	\item O orçamento é de até \maximumBudget;
\end{itemize}

%Objetivos mensuráveis do projeto e critérios de sucesso relacionados.

\subsection{Requisitos}

Requisitos de alto nível.

\subsection{Premissas}

Premissas e restrições.

\subsection{Restrições}

Premissas e restrições.

\subsection{Riscos}

Riscos de Alto Nível

\subsection{Marcos}

Resumo do Cronograma de Marcos

\subsection{Orçamento}

Resumo do Orçamento

\subsection{Partes Interessadas}

Lista das Partes Interessadas.

\subsection{Requisitos para aprovação do projeto}

Requisitos para aprovação do projeto (ou seja, o que constitui o sucesso do projeto, quem decide se
o projeto é bem sucedido e quem assina o projeto).

\subsection{Controle de Versão}

\subsection{Aprovações}

\todo[inline,color=red]{Finalizar termo de abertura.}

%\section{Identificar Partes Interessadas}

\subsection{Análise das Partes Interessadas}

Opcional.

\subsection{Registro das Partes Interessadas}

\begin{tabular}{l*{6}{c}}
	Grupo & Nome & Posição/Função & Interesse & Influência & Força/Impacto & Expectativas \\
	\hline
	Cliente & Parte Interessada 1 & Interesse & Função & Influência & Impacto & Expectativas \\
\end{tabular}

\todo[inline,color=red]{Identificar partes interessadas.}

\chapter{Processos de Planejamento}

%---Escopo
%Descrição do processo utilizado para Gerenciamento do Escopo
%Descrição dos processos de coleta de requisitos
%	Descrição de como os requisitos serão coletados
%	Frequência/ número de eventos de coleta/ limite para cessar a coleta
%	Eventos listados na Matriz de Comunicações
%Descrição do processo de validação e controle do escopo do projeto
%	Descrição de como os requisitos serão validados e controlados
%	Frequência das validações e controle
%	Eventos listados na Matriz de Comunicações
%	Remete ao Controle Integrado de Mudanças em caso de divergência planejado x realizado
%---


\chapter{Plano de gerenciamento de escopo}

\section{Objetivo do documento}

O objetivo deste documento é descrever como será gerenciado o escopo, descrevendo quais ferramentas, técnicas e artefatos serão utilizados para determinar o que deve ser abordado durante o projeto \projectName.

\section{Descrição dos processos de gerenciamento de escopo}

\begin{itemize}
	\item O gerenciamento do escopo do projeto será realizado com base em 3 documentos: declaração de escopo para o escopo funcional do projeto, EAP para o escopo das atividades a serem realizadas pelo projeto e dicionário da EAP para descrever os pacotes de trabalho.
	\item Serão consideradas mudanças de escopo apenas as medidas corretivas. Inovações e novas características do produto ou projeto deverão ser tratados de acordo com o plano de gerenciamento da configuração (ver capítulo \ref{ch:configuration-management-plan}).
	\item Todas as mudanças de escopo deverão ser submetidas por escrito ou através de e-mail, conforme descrito no plano de comunicações do projeto.
\end{itemize}

\section{Priorização das mudanças de escopo e respostas}

\todo[inline,color=orange]{Criar níveis de priorização para as mudanças de escopo ou referenciar modelo de priorização integrado de mudanças.}

\section{Gerenciamento de configuração}

\todo[inline,color=orange]{Referenciar fluxo de controle integrado de mudanças.}

\section{Frequência de avaliação do escopo do projeto}

O escopo deve ser avaliado semanalmente dentro da reunião do CCM, prevista no plano de gerenciamento das comunicações (ver capítulo \ref{ch:communication-management-plan}).

\section{Alocação financeira das mudanças de escopo}

As mudanças de escopo podem ser alocadas dentro das reservas gerenciais do projeto de acordo com as necessidades do gerente de projeto.

Para mudanças de escopo prioritárias, em momentos que não existam mais reservas gerenciais disponíveis, deverá ser acionado o patrocinador, já que o gerente de projeto não possui autonomia para decidir utilizar a reserva de contingência de riscos para mudanças de escopo.

\section{Administração do plano de gerenciamento do escopo}

\subsection{Responsável}

\begin{itemize}
	\item \projectManagerName, gerente de projeto, será o responsável direto pelo plano de gerenciamento de escopo.
\end{itemize}
\todo[inline,color=orange]{Verificar necessidade de adicionar suplente responsável.}

\subsection{Frequência de atualização}

O plano de gerenciamento do escopo será reavaliado mensalmente durante a reunião do CCM, juntamente com os outros planos de gerenciamento do projeto.

\section{Outros assuntos relacionados ao gerenciamento do escopo do projeto não previstos neste plano}

As solicitações não previstas neste plano deverão ser submetidas a reunião do CCM para aprovação.
\todo[inline,color=orange]{Adicionar menção ao plano de mudanças}

\section{Controle de Versão}

\begin{table}[H]
	\begin{tabularx}{\textwidth}{| c | c | X | X |}
		\hline
		\textbf{Versão} & \textbf{Data} & \textbf{Autor}      & \textbf{Notas de Revisão} \\
		\hline
		1                &               & \projectManagerName & Criação do documento     \\
		\hline
	\end{tabularx}
	\centering
\end{table}

\section{Aprovações}

\begin{table}[H]
	\begin{tabularx}{\textwidth}{| c | c | X | c |}
		\hline
		\textbf{Função}  & \textbf{Nome}       & \textbf{Assinatura}      & \textbf{Data} \\
		\hline
		Gerente de projeto & \projectManagerName & \projectManagerSignature &               \\
		\hline
	\end{tabularx}
	\centering
\end{table}

%Tempo
%Descrição do processo de definição das atividades (base na decomposição dos pacotes da EAP normalmente)
%Descrição do processo e técnicas de sequenciamento das atividades
%Descrição do processo de estimativa de recursos para as atividades
%Descrição do método de gerenciamento do tempo que será empregado no projeto (PERT/CPM, corrente crítica, SCRUM, ...) e a justificativa para sua aplicação no projeto
%Elementos visuais:
%   Visão executiva do cronograma em formato gráfico (Gantt, Burndown, ...), deve ocupar apenas uma página
%   Principais marcos do projeto na visão do cliente final, patrocinador ou gerência imediata / tem no escopo
%   Cronograma detalhado do projeto com identificação do caminho crítico no formato Gantt contendo: Atividade, Duração, Data de início, Data de fim, Percentual de progresso
%   Lista de atividades do caminho crítico com datas de início e fim
%Apresentação do processo empregado para gerenciamento da linha de base de tempo
%Descrição do processo de controle do cronograma
%   Frequência das ações de controle
%   Eventos listados na Matriz de Comunicações
%   Remete ao Controle Integrado de Mudanças em caso de divergência planejado x realizado

\chapter{Plano de gerenciamento do cronograma}

\todo[inline,color=red]{Criar plano de gerenciamento do cronograma.}

\section{Descrição dos processos de gerenciamento de tempo}

\begin{itemize}

\end{itemize}

\section{Priorização das mudanças nos prazos e respostas}

\section{Sistema de controle de mudanças de prazos}

\section{Mecanismo adotado para conflitos de recursos}

\section{Buffer de tempo do projeto}

\section{Frequência de avaliação dos prazos do projeto}

\section{Alocação financeira para o gerenciamento de tempo}

\section{Administração do plano de gerenciamento de tempo}

\subsection{Responsável}

\begin{itemize}
	\item \projectManagerName, gerente de projeto, será o responsável direto pelo plano de gerenciamento do cronograma.
\end{itemize}
\todo[inline,color=orange]{Verificar necessidade de adicionar suplente responsável.}

\subsection{Frequência de atualização}

O plano de gerenciamento do cronograma será reavaliado mensalmente durante a reunião do CCM, juntamente com os outros planos de gerenciamento do projeto.

\section{Outros assuntos relacionados ao gerenciamento do cronograma do projeto não previstos neste plano}

\section{Controle de Versão}

\begin{table}[H]
	\begin{tabularx}{\textwidth}{| c | c | X | X |}
		\hline
		\textbf{Versão} & \textbf{Data} & \textbf{Autor}      & \textbf{Notas de Revisão} \\
		\hline
		1                &               & \projectManagerName & Criação do documento     \\
		\hline
	\end{tabularx}
	\centering
\end{table}

\section{Aprovações}

\begin{table}[H]
	\begin{tabularx}{\textwidth}{| c | c | X | c |}
		\hline
		\textbf{Função}  & \textbf{Nome}       & \textbf{Assinatura}      & \textbf{Data} \\
		\hline
		Gerente de projeto & \projectManagerName & \projectManagerSignature &               \\
		\hline
	\end{tabularx}
	\centering
\end{table}

% Descrição do método de gerenciamento do custo a ser empregado no projeto incluindo: Sistema de alocação de custos, Relatórios de controle, Periodicidade de atualização
% Método de estimativa dos custos
% Apresentação da composição de valores que totalizam o custo do projeto incluindo: 
%   Custo com pessoal
%   Aquisições
%   Reservas Gerenciais
%   Reservas de Contingência (oriunda dos riscos)
%   Outros
% EVA – Análise de Valor Agregado do Projeto (simulação em momento projetado):
%   Aplicação do método de EVA no projeto
%   Detalhamento dos índices utilizados e análise dos resultados
%   Curvas de desempenho
%   Apresentação de análise comparativa entre o orçamento inicial aprovado do projeto e o custo orçado inicialmente para o projeto: Justificar em caso de diferença 
% Modelo de análise e controle de custos do projeto:
%   Comparação do realizado x linha de base de custos do projeto
%   Ações caso existam diferenças
%   Fluxo de caixa
%   Custo final projetado considerando-se o desempenho atual do projeto
% Apresentação dos custos correntes ou potenciais que não serão apropriados ao projeto
% Descrição do processo de controle dos custos
%   Frequência das ações de controle
%   Eventos listados na Matriz de Comunicações
%   Remete ao Controle Integrado de Mudanças em caso de divergência planejado x realizado

\chapter{Plano de gerenciamento dos custos}

\todo[inline,color=red]{Terminar plano de gerenciamento dos custos.}

\section{Objetivos do documento}

O plano de gerenciamento dos custos descreve como os custos do projeto serão planejados, estruturados e controlados fornecendo detalhes sobre os processos e ferramentas utilizados para gerenciar questões relacionadas a custos.

\section{Descrição dos processos de gerenciamento de custos}

\todo[inline,color=red]{Continuar descrição dos processos de gerenciamento de custos}

\begin{itemize}
	% Estimar custos
	% Ver custo da qualidade (PMBOK 8.1.2.2)
	\item As estimativas de custos serão realizadas utilizando o método \"bottom-up\" com base na opinião especializada.
	\item Custos que possuam incertezas e riscos significantes deverão utilizar a análise de PERT, conforme equação \ref{eq:cost-pert}.
	      \begin{equation}\label{eq:cost-pert}
		      CE = \frac{CO+4CMP+CP}{6}
	      \end{equation}

	      \begin{description}
		      \item[Custo Estimado (DE):] a melhor estimativa do custo necessário para completar a atividade, levando em consideração o fato de que o projeto nem sempre corre conforme o planejado.
		      \item[Custo Mais Provável (MP):] custo da atividade baseado em um esforço de avaliação realista para o trabalho necessário e quaisquer outros gastos previstos.
		      \item[Custo Otimista (O):] custo da atividade baseado na análise do melhor cenário possível para a atividade.
		      \item[Custo Pessimista (P):] custo da atividade baseado na análise do pior cenário para a atividade.
	      \end{description}
	\item A atualização e gerenciamento dos custos do projeto será realizada utilizando o software \projectManagementSoftwareName.
	% Determinar o orçamento
	% Controlar custos
	\item A avaliação dos custos será feito por meio da comparação entre 3 estimativas: valor planejado, valor real e valor agregado.
	\item O controle dos custos será feito tomando como base as estimativas de valor agregado.
	\item O gerente do projeto irá acompanhar a utilização de horas do projeto
\end{itemize}

\section{Frequência de avaliação do orçamento do projeto e das reservas gerenciais}

\section{Reservas gerenciais}

\section{Autonomias}

\section{Alocação financeira das mudanças no orçamento}

\section{Administração do plano de gerenciamento de custos}

\subsection{Responsável pelo plano}

\subsection{Frequência de atualização do plano de gerenciamento de custos}

\section{Outros assuntos relacionados ao gerenciamento de custos do projeto não previstos neste plano}

\section{Controle de Versão}

\begin{table}[H]
	\begin{tabularx}{\textwidth}{| c | c | X | X |}
		\hline
		\textbf{Versão} & \textbf{Data} & \textbf{Autor}      & \textbf{Notas de Revisão} \\
		\hline
		1                &               & \projectManagerName & Criação do documento     \\
		\hline
	\end{tabularx}
	\centering
\end{table}

\section{Aprovações}

\begin{table}[H]
	\begin{tabularx}{\textwidth}{| c | c | X | c |}
		\hline
		\textbf{Função}  & \textbf{Nome}       & \textbf{Assinatura}      & \textbf{Data} \\
		\hline
		Patrocinador       & \projectSponsorName & \projectSponsorSignature &               \\
		\hline
		Gerente de projeto & \projectManagerName & \projectManagerSignature &               \\
		\hline
	\end{tabularx}
	\centering
\end{table}



\chapter{Plano de gerenciamento da qualidade}

\section{Descrição dos processos de gerenciamento da qualidade}

\section{Priorização das mudanças nos requisitos de qualidade e respostas}

\section{Requisitos de qualidade}

\section{Padrões de qualidade}

\section{Sistema de controle de mudanças da qualidade}

\section{Frequência de avaliação dos requisitos de qualidade do projeto}

\section{Alocação financeira das mudanças nos requisitos de qualidade}

\section{Administração do plano de gerenciamento da qualidade}

\subsection{Responsável}

\begin{itemize}
	\item \projectManagerName, gerente de projeto, será o responsável direto pelo plano de gerenciamento da qualidade.
\end{itemize}
\todo[inline,color=orange]{Verificar necessidade de adicionar suplente responsável.}

\subsection{Frequência de atualização}

O plano de gerenciamento da qualidade será reavaliado mensalmente durante a reunião do CCM, juntamente com os outros planos de gerenciamento do projeto.

\section{Outros assuntos relacionados ao gerenciamento da qualidade do projeto não previstos neste plano}

\section{Controle de Versão}

\begin{table}[H]
	\begin{tabularx}{\textwidth}{| c | c | X | X |}
		\hline
		\textbf{Versão} & \textbf{Data} & \textbf{Autor}      & \textbf{Notas de Revisão} \\
		\hline
		1                &               & \projectManagerName & Criação do documento     \\
		\hline
	\end{tabularx}
	\centering
\end{table}

\section{Aprovações}

\begin{table}[H]
	\begin{tabularx}{\textwidth}{| c | c | X | c |}
		\hline
		\textbf{Função}  & \textbf{Nome}       & \textbf{Assinatura}      & \textbf{Data} \\
		\hline
		Gerente de projeto & \projectManagerName & \projectManagerSignature &               \\
		\hline
	\end{tabularx}
	\centering
\end{table}

\todo[inline,color=red]{Criar plano da qualidade.}

% Descrição do tipo de estrutura organizacional da empresa onde o projeto está inserido: Projetizada, matricial forte, funcional, Identificar prós e contras do modelo vigente
% Descrição das necessidades de recursos humanos do projeto: Papéis, Competências, Responsabilidades, Quantidades, Relações hierárquicas
% Organograma do projeto (ideal é ter nome + papel)
% Matriz RACI
% Descrição do plano de mobilização da equipe do projeto: Processo de alocação ou contratação, Datas de alocação e saída do projeto
% Descrição do plano de desenvolvimento de competências da equipe do projeto
% Plano de gerenciamento das pessoas do projeto:
%   Formato de acompanhamento e monitoramento da equipe
%   Frequência do acompanhamento
%   Eventos listados na Matriz de Comunicações
%   Possíveis técnicas de resolução de problemas de acordo com a cultura da empresa
%   Processo de avaliação e feedback
%   Remete ao Controle Integrado de Mudanças em caso de divergência planejado x realizado
% Descrever o modelo de gerenciamento de segurança do projeto: Integridade física, Confidencialidade das informações

\chapter{Plano de gerenciamento dos recursos humanos}

\todo[inline,color=green]{Criar plano de gerenciamento dos recursos humanos.}

\chapter{Plano de gerenciamento das comunicações}
\label{ch:communication-management-plan}

\todo[inline,color=green]{Criar plano de gerenciamento das comunicações.}

\section{Plano de Gerenciamento dos Riscos}

\todo[inline,color=green]{Criar plano de gerenciamento dos riscos.}

% Descrição do processo de aquisições a ser empregado no projeto
% Apresentação da análise de make or buy com justificativas
% Apresentação da Declaração de Trabalho dos itens a serem adquiridos
% Apresentar o fluxo do processo de aquisições
% Apresentação dos modelos de formulários a serem utilizados no processo de aquisição
% Organograma de aquisições
% Processo de identificação e seleção de fornecedores
% Processo de qualificação de propostas
% Identificar os tipos de contratos a serem empregados no projeto
% Apresentar os sistemas preferenciais de garantia
% Apresentação do procedimento de Controle das Aquisições
%   Métodos de controle
%   Frequência das ações de controle
%   Eventos listados na Matriz de Comunicações
%   Remete ao Controle Integrado de Mudanças em caso de divergência planejado x realizado 
% Apresentação do procedimento de Encerramento das Aquisições
 
\chapter{Plano de gerenciamento das aquisições}

\section{Descrição dos processos de gerenciamento das aquisições}

\begin{itemize}
	\item O gerenciamento das aquisições estará focado na contratação de um servidor em nuvem.
	\item 
\end{itemize}

\todo[inline,color=red]{Criar descrição dos processos de aquisição.}

\section{Gerenciamento e tipos de contratos}

\todo[inline,color=red]{Criar gerenciamento e tipos de contratos.}

\section{Critérios de avaliação de cotações e propostas.}

\todo[inline,color=red]{Criar critérios de avaliação de cotações e propostas.}

\section{Avaliação de fornecedores}

\todo[inline,color=red]{Criar método para avaliação de fornecedores.}

\section{Frequência de avaliação dos processos de aquisições}

\todo[inline,color=red]{Descrever frequência de avaliação dos processos de aquisições.}

\section{Alocação financeira para o gerenciamento das aquisições}

\todo[inline,color=red]{Descrever método para aloação financeira do plano de aquisições.}

\section{Administração do plano de gerenciamento das aquisições}

\subsection{Responsável pelo plano}

\begin{itemize}
	\item \projectManagerName{}, gerente de projeto, será o responsável direto pelo plano de gerenciamento das aquisições.
\end{itemize}

\subsection{Frequência de atualização do plano de gerenciamento das aquisições}

O plano de gerenciamento das aquisições será reavaliado semanalmente durante a reunião do CCM, juntamente com os outros planos de gerenciamento do projeto.

\section{Outros assuntos relacionados ao gerenciamento das aquisições do projeto não previstos neste plano}

Solicitações não previstas neste plano deverão passar pela aprovação do CCM. Após aprovada o plano deve ser atualizado pelo gerente do projeto.

\section{Controle de versão}

\begin{table}[H]
	\begin{tabularx}{\textwidth}{| c | c | X | X |}
		\hline
		\textbf{Versão} & \textbf{Data} & \textbf{Autor}        & \textbf{Notas de Revisão} \\
		\hline
		1                &               & \projectManagerName{} & Criação do documento     \\
		\hline
	\end{tabularx}
	\centering
\end{table}

\section{Aprovações}

\begin{table}[H]
	\begin{tabularx}{\textwidth}{| c | c | X | c |}
		\hline
		\textbf{Função}  & \textbf{Nome}         & \textbf{Assinatura}        & \textbf{Data} \\
		\hline
		Patrocinador       & \projectSponsorName{} & \projectSponsorSignature{} &               \\
		\hline
		Gerente de projeto & \projectManagerName{} & \projectManagerSignature{} &               \\
		\hline
	\end{tabularx}
	\centering
\end{table}

\chapter{Plano de gerenciamento das partes interessadas}

\todo[inline,color=green]{Criar plano de gerenciamento das partes interessadas.}

% ----------------------------------------------------------
% Finaliza a parte no bookmark do PDF
% para que se inicie o bookmark na raiz
% e adiciona espaço de parte no Sumário
% ----------------------------------------------------------
\phantompart

\chapter*[Conclusão]{Conclusão}
\addcontentsline{toc}{chapter}{Conclusão}

\todo[inline,color=green]{Criar conclusão.}

% ----------------------------------------------------------
% POSTEXTUAL ELEMENTS
% ----------------------------------------------------------
\postextual
% ----------------------------------------------------------

% ----------------------------------------------------------
% Bibliography
% ----------------------------------------------------------
\bibliography{bibliography}

% ----------------------------------------------------------
% Appendix
% ----------------------------------------------------------

% ---
% Appendix begin
% ---
\begin{apendicesenv}

	% Print page indicating the appendix begin
	\partapendices

	%\chapter{Pesquisa de Mercado}

\begin{enumerate}
\item Nome da empresa para a qual trabalha?
\item Cargo ocupado nesta empresa?
\item Quantas vezes por semana o estacionamento fica lotado?
\item Você sabe quais os dias de maior lotação no estacionamento?
\item Você tem conhecimento do plano estratégico da empresa? Qual seria a missão e visão da empresa?
\item Faz parte dos objetivos estratégicos da empresa a satisfação do cliente?
\item Quais informações você considera mais importantes para atingir a missão e visão da empresa?
\item Você sberia informar se existe a formação de filas? Qual o tempo médio de espera em fila? 
\end{enumerate}

\todo[inline,color=red]{Finalizar pesquisa de mercado}

	%\chapter{Especificação do Trabalho do Projeto}

\section{Necessidade de Negócios}

\section{Descrição do Escopo do Produto}

\section{Plano Estratégico}

\chapter{Proposição de Valor}

\todo[inline,color=red]{Criar proposição de valor}

\chapter{Modelo de Negócio}

\todo[inline,color=red]{Criar modelo do negócio}

\chapter{Plano de Negócios (Business Case)}

\todo[inline,color=red]{Criar plano de negócios}

\section{Sumário Executivo}

\section{Descrição Geral da Companhia}

\section{Produtos e Serviços}

\section{Plano de Marketing}

\section{Plano Operacional}

\section{Organização e Gerenciamento}

\section{Declaração de Finanças Pessoais}

\section{Gastos e Capitalização Iniciais}

\section{Planejamento Financeiro}

	%---Escopo
%Apresentação da estrutura analítica do projeto (EAP)
%	ID
%	Nome
%	Entregas de GP
%	Sem filhos únicos
%	Tarefas x Pacotes
%---

\chapter{Estrutura Analítica do Projeto}
\label{ch:wbs}

\section{Controle de Versão}

\begin{table}[H]
	\begin{tabularx}{\textwidth}{| c | c | X | X |}
		\hline
		\textbf{Versão} & \textbf{Data} & \textbf{Autor}      & \textbf{Notas de Revisão} \\
		\hline
		1                &               & \projectManagerName & Criação do documento     \\
		\hline
	\end{tabularx}
	\centering
\end{table}

\section{Aprovações}

\begin{table}[H]
	\begin{tabularx}{\textwidth}{| c | c | X | c |}
		\hline
		\textbf{Função}  & \textbf{Nome}       & \textbf{Assinatura}      & \textbf{Data} \\
		\hline
		Gerente de projeto & \projectManagerName & \projectManagerSignature &               \\
		\hline
	\end{tabularx}
	\centering
\end{table}

	%---Escopo
%Apresentação do dicionário da EAP contendo no mínimo:
%	ID
%	Nome
%	Descrição do Pacote
%	Critérios de aceitação
%---

\chapter{Dicionário da EAP}
\label{ch:wbs-dictionary}

\section{Controle de Versão}

\begin{table}[H]
	\begin{tabularx}{\textwidth}{| c | c | X | X |}
		\hline
		\textbf{Versão} & \textbf{Data} & \textbf{Autor}      & \textbf{Notas de Revisão} \\
		\hline
		1                &               & \projectManagerName & Criação do documento     \\
		\hline
	\end{tabularx}
	\centering
\end{table}

\section{Aprovações}

\begin{table}[H]
	\begin{tabularx}{\textwidth}{| c | c | X | c |}
		\hline
		\textbf{Função}  & \textbf{Nome}       & \textbf{Assinatura}      & \textbf{Data} \\
		\hline
		Gerente de projeto & \projectManagerName & \projectManagerSignature &               \\
		\hline
	\end{tabularx}
	\centering
\end{table}

	\begin{landscape}

\includepdf[landscape=true,turn=true,pages=1,pagecommand={\chapter{Gráfico Gannt}\label{ch:gannt-chart}},linktodoc=true]{include/gannt.pdf}
\includepdf[landscape=true,turn=true,pages=2-,pagecommand={}]{include/gannt.pdf}

\end{landscape}

	\chapter{Custos por atividade}
\label{ch:activities-cost}

\begin{center}
\begin{tikzpicture}[scale=0.8, every node/.style={scale=0.8}]
  \node at (current page.center) {\includegraphics[page=1]{include/activities-cost.pdf}};
\end{tikzpicture}
\end{center}

\includepdf[pages=2-,pagecommand={},scale=0.8]{include/activities-cost.pdf}

	\chapter{Análise do valor agregado}
\label{project-monitoring-report}

A técnica de análise do valor agregado é uma metodologia que combina escopo, cronograma, e medições de recursos para avaliar o desempenho e progresso do projeto.

Os dois principais indicadores da análise de valor agregado são o IDP (índice de desempenho de prazo) e o IDC (índice de desempenho de custo). Estes indicadores podem ser interpretados da seguinte maneira:

\begin{description}
	\item[IDP = 1] Projesto está atendendo as expectativas (dentro do prazo).
	\item[IDP > 1] Projeto está superando as expectativas (adiantado).
	\item[IDP < 1] Projeto está abaixo das expectativas (atrasado).
\end{description}

O gráfico da análise de valor agregado e dos índices de desempenho podem ser vistos na figura \ref{fig:earned-value-analysis}.

\begin{figure}[h]
	\centering
	\begin{tikzpicture}[scale=0.7, every node/.style={scale=0.7}]
		\node at (current page.center) {\includegraphics[page=1]{include/earned-value-chart.pdf}};
	\end{tikzpicture}
	\caption{Análise de valor agregado.}
	\label{fig:earned-value-analysis}
\end{figure}

	\chapter{Lista de verificação da qualidade}
\label{quality-checklist}

\begin{longtable}{ p{0.7\textwidth} c c }
	\toprule
	\thead[c]{\textbf{Descrição}}                                              & \thead[c]{\textbf{Peso}} & \thead[c]{\textbf{Nota}} \\
    \midrule
	\toprule
	\textbf{Iniciação do projeto}                                                   & \textbf{6}               &                          \\
    \bottomrule
	Termo de abertura desenvolvido e aprovado?                                        & 4                        &                          \\
    \midrule
	Partes interessadas identificadas?                                                & 2                        &                          \\
    \midrule
    \toprule
	\textbf{Planejamento}                                                             & \textbf{12}              &                          \\
    \bottomrule
	Plano de projeto desenvolvido e aprovado?                                         & 7                        &                          \\
    \midrule
	Escopo aprovado?                                                                  & 5                        &                          \\
    \midrule
    \toprule
	\textbf{Testes}                                                                   & \textbf{16}              &                          \\
    \bottomrule
	Existe documentação para a evidência dos testes?                               & 5                        &                          \\
    \midrule
	As evidências dos testes foram entregues ao patrocinador?                        & 5                        &                          \\
    \midrule
	Os planos de teste foram devidamente documentados?                                & 3                        &                          \\
    \midrule
	A versão final foi revisada pela equipe de testes?                               & 3                        &                          \\
    \midrule
    \toprule
	\textbf{Execução}                                                               & \textbf{59}              &                          \\
    \bottomrule
	A equipe está utilizando as melhores práticas para desenvolvimento do software? & 10                       &                          \\
    \midrule
	Os prazos estipulados no cronograma estão sendo suficientes?                     & 15                       &                          \\
    \midrule
	As estimativas de esforço do projeto estão corretas?                            & 12                       &                          \\
    \midrule
	O código fonte está no controle de versão?                                     & 12                       &                          \\
    \midrule
	A infraestrutura está adequada para o desenvolvimento?                           & 10                       &                          \\
    \midrule
    \toprule
	\textbf{Encerramento}                                                             & \textbf{7}               &                          \\
    \bottomrule
	As lições aprendidas foram devidamente documentadas?                            & 2                        &                          \\
    \midrule
	O patrocinador realizou o aceite formal?                                          & 5                        &                          \\
    \midrule
    \toprule
	\textbf{Total}                                                                    & \textbf{100}             &                          \\
	\bottomrule
	\caption{Lista de verificação da qualidade.}
	\centering
\end{longtable}

	\chapter{Métricas de qualidade}

	\chapter{Matriz de responsabilidades}
\label{ch:raci-matrix}

A matriz de responsabilidade pode ser vista na tabela \ref{tab:raci-matrix}. Utilizou-se a ferramenta chamada matriz \foreign{Responsible, Accountable, Consulted, Informed} (RACI), logo para relacionar os recursos e as atividades utiliza-se a seguinte classificação:

\begin{description}
	\item[Responsável(R)] O recurso fica responsável por executar a atividade.
	\item[Aprovador(A)] É o responsável pelo aceite formal da atividade.
	\item[Consultado(C)] É alguém que deve ser consultado, participando das decisões ou da execução da atividade.
	\item[Informado(I)] Indica a pessoa que deve ser informada sobre algo em relação à atividade.
\end{description}

\begin{longtable}{ l p{0.3\textwidth} >{\centering\arraybackslash}>{\columncolor{gray!10}}p{0.025\textwidth} >{\centering\arraybackslash}p{0.025\textwidth} >{\centering\arraybackslash}>{\columncolor{gray!10}}p{0.025\textwidth} >{\centering\arraybackslash}p{0.025\textwidth} >{\centering\arraybackslash}>{\columncolor{gray!10}}p{0.025\textwidth} >{\centering\arraybackslash}p{0.025\textwidth} >{\centering\arraybackslash}>{\columncolor{gray!10}}p{0.025\textwidth} >{\centering\arraybackslash}p{0.025\textwidth} >{\centering\arraybackslash}>{\columncolor{gray!10}}p{0.025\textwidth} >{\centering\arraybackslash}p{0.025\textwidth} >{\centering\arraybackslash}>{\columncolor{gray!10}}p{0.025\textwidth} >{\centering\arraybackslash}p{0.025\textwidth} >{\centering\arraybackslash}>{\columncolor{gray!10}}p{0.025\textwidth} >{\centering\arraybackslash}p{0.025\textwidth} }
	\toprule
	\thead[c]{\textbf{ID}} & \thead[c]{\textbf{Atividade}} & \rot{\textbf{\parbox{6cm}{Patrocinador}}} & \rot{\textbf{\parbox{6cm}{Gerente do Projeto}}} & \rot{\textbf{\parbox{6cm}{Engenheiro de software}}} & \rot{\textbf{\parbox{6cm}{Arquiteto de software}}} & \rot{\textbf{\parbox{6cm}{Desenvolvedor mobile 1}}} & \rot{\textbf{\parbox{6cm}{Desenvolvedor mobile 2}}} & \rot{\textbf{\parbox{6cm}{Desenvolvedor web front-end}}} & \rot{\textbf{\parbox{6cm}{Desenvolvedor web back-end}}} & \rot{\textbf{\parbox{6cm}{Analista de banco de dados}}} & \rot{\textbf{\parbox{6cm}{Analista de testes 1}}} & \rot{\textbf{\parbox{6cm}{Analista de testes 2}}} \\*
	\hline
	\endhead
	\multicolumn{13}{c}{{\textit{Continua na próxima página.}}} \\*
	\caption{Matriz de responsabilidades.}
	\endfoot
	\endlastfoot
	1       & \textbf{Gerenciamento do projeto}                                 &   &   &   &   &   &   &   &   &   &   &      \\*
	\hline
	1.1     & Iniciação                                              &  A &  R &  C & C  &  I &  I & I  & I  &  I &  I & I     \\*
	\hline
	1.2     & Planejamento                                             & A  & R  &  C &  C &  I &  I & I  & I  & I  & I  & I     \\*
	\hline
	1.3     & Aprovação do escopo                                    &  A &  R &   &   &   &   &   &   &   &   &      \\*
	\hline
	1.4     & Monitoramento do projeto                                 & A  & R  & I  &  I &  I &  I &  I &  I &  I &  I & I     \\*
	\hline
	1.5     & Encerramento                                             & A  & R  &  I &  I & I  &  I &  I &  I &  I &  I & I     \\*
	\hline
	2       & \textbf{Infraestrutura}                                           &   &   &   &   &   &   &   &   &   &   &      \\*
	\hline
	2.1     & Ambiente de desenvolvimento                              & I  & A  &  R &  C &  C &  C &  C &  C &  C &  C & C     \\*
	\hline
	2.2     & Ambiente de testes                                       & I  & A  & C  &  R &  I &  I &  I &  I &  I &  C &  C    \\*
	\hline
	2.3     & Ambiente de produção                                   & I & A & C &  R &  I &  I &  I &  I &  I &  I & I     \\*
	\hline
	3       & \textbf{Web API}                                                  &   &   &   &   &   &   &   &   &   &   &      \\*
	\hline
	3.1     & Contratar plataforma na nuvem                            &   &  A &   &  R &   &   &  C &   &  C &   &      \\*
	\hline
	3.2     & Banco de dados                                           &   & A  & C  & C  &  I &  I &  I &  I &  R & I  & I     \\*
	\hline
	3.3     & Definir estrutura da API                                 &   & A  & C  &  C &  I &  I &  I & R  & C  & I  & I     \\*
	\hline
	3.4     & Criar rotas para recursos                                &   & A  &  C &  C &  I &  I &  I &  R &  I &  I & I     \\*
	\hline
	3.5     & Testes                                                   &  I & A  & C  & C  & C  & C  & C  & C  & C  & R  & C     \\*
	\hline
	4       & \textbf{Aplicativo do motorista}                                  &   &   &   &   &   &   &   &   &   &   &      \\*
	\hline
	4.1     & Autenticação                                           &   & A  &  C &   &  R &   &   &   & C  &   &      \\*
	\hline
	4.2     & Reserva de vaga                                          &   &  A & C  &   & R  &   &   &   & C  &   &      \\*
	\hline
	4.3     & Lista de estacionamentos                                 &   & A  &  C &   &  R &   &   &   & C  &   &      \\*
	\hline
	4.4     & Processamento de pagamentos                              &   & A  &  C &   &   &  R &   &   & C  &   &      \\*
	\hline
	4.5     & Gerência do usuário                                    &   & A  &  C &   &   &   &  R &   & C  &   &      \\*
	\hline
	4.6     & Testes                                                   & I  &  A &  C &  C &  C &  C &  C &  C &  C & C  & R     \\*
	\hline
	5       & \textbf{Aplicativo estacionamento}                                &   &   &   &   &   &   &   &   &   &   &      \\*
	\hline
	5.1     & Autenticação e gerência de usuários do estacionamento &   &   & A  & C  & C  &   & R  & C  & C  &   &      \\*
	\hline
	5.2     & Gerenciamento dos clientes                               &   & A  & C  &  C &   & R  &  C &   &  C &   &      \\*
	\hline
	5.3     & Solicitação de cadastro no sistema                     &   & A  &  C &  C &   &   & R  & C  &  C &   &      \\*
	\hline
	5.4     & Gerenciamento do estacionamento                          &   &   &   &   &   &   &   &   &   &   &      \\*
	\hline
	5.4.1   & Gerenciamento de garagens                                &   & A  & C  &  C &   &   & R  & C  & C  &   &      \\*
	\hline
	5.4.1.1 & Cadastro e gerência de garagens                         &   & A  &  C &  C &   &   & R  & C  & C  &   &      \\*
	\hline
	5.4.1.2 & Situação das vagas                                     &   &  A & C  &  C &   &   & R  &  C & C  &   &      \\*
	\hline
	5.4.1.3 & Integração com sistema de automação                  &   & A  & C  &  C &   &   & R  & C & C  &   &      \\*
	\hline
	5.4.2   & Configurações do estacionamento                        &   & A  & C  & C  &   &   &  R & C  &  C &   &      \\*
	\hline
	5.5     & Relatórios                                              &   & A  & C  & C  &   &   & R  & C  &  C &   &      \\*
	\hline
	5.5.1   & Tendências em relação a ocupação do estacionamento   &   & A  & C  & C  &   &   & R  & C  &  C &  &    \\*
	\hline
	5.6     & Integração com DETRAN                                  &   & A  & C  &  C &   &   & R  & C  &   &   &      \\*
	\hline
	5.7     & Testes                                                   & I  & A  & C  &  C &  C &  C & C  &  C &  C & R  & R     \\*
	\hline
	6       & \textbf{Auditoria da qualidade}                                   & C &  A &  I &  I &  I &  I &  I &  I & I  &  R & R     \\*
	\bottomrule
	\caption{Matriz de responsabilidades.}
	\label{tab:raci-matrix}
	\centering
\end{longtable}

	\chapter{Formulário de avaliação de desempenho}
\label{ch:team-evaluation-form}

\section{Escala de avaliação}

O desempenho deve ser avaliado conforme segue:

\begin{description}
    \item[NA:] Não atende.
    \item[AP:] Atende parcialmente.
    \item[AE:] Atende o esperado.
    \item[SE:] Supera o esperado.
\end{description}

\section{Avaliação}

\begin{longtable}{ >{\bfseries}p{0.2\textwidth} p{0.5\textwidth} c c c c}
\toprule
\thead[c]{\textbf{Característica}} & \thead[c]{\textbf{Item de descrição do comportamento}} & \textbf{NA} & \textbf{AP} & \textbf{AE} & \textbf{SE} \\
\midrule
\endhead
\multicolumn{6}{c}{{\textit{Continua na próxima página.}}} \\
\endfoot
\endlastfoot
Liderança & É capaz de influenciar os demais membros da equipe e motivá-los a participar do projeto? & {\large$\Box$} & {\large$\Box$} & {\large$\Box$} & {\large$\Box$} \\
\cmidrule{2-6}
 & Entende a diversidade de personalidades e consegue despertar o melhor em cada uma? & {\large$\Box$} & {\large$\Box$} & {\large$\Box$} & {\large$\Box$} \\
\midrule
Integridade Moral & Possui uma postura ética nas atividades do dia-dia?  & {\large$\Box$} & {\large$\Box$} & {\large$\Box$} & {\large$\Box$} \\
\cmidrule{2-6}
& Busca ser imparcial em situações de opiniões contraditórias? & {\large$\Box$} & {\large$\Box$} & {\large$\Box$} & {\large$\Box$} \\
\midrule
Versatilidade & Incentiva as pessoas da equipe a buscarem soluções para os obstáculos encontrados nas atividades do projeto e da empresa? & {\large$\Box$} & {\large$\Box$} & {\large$\Box$} & {\large$\Box$} \\
\cmidrule{2-6}
& Enfrenta novos desafios com criatividade e entusiasmo? & {\large$\Box$} & {\large$\Box$} & {\large$\Box$} & {\large$\Box$} \\
\midrule
Relacionamento & Consegue unir os colaboradores para trabalharem em equipe? & {\large$\Box$} & {\large$\Box$} & {\large$\Box$} & {\large$\Box$} \\
\cmidrule{2-6}
& Em situações de conflitos de opinião, procura conciliar as opiniões e reaproximar as pessoas? & {\large$\Box$} & {\large$\Box$} & {\large$\Box$} & {\large$\Box$} \\
\midrule
Olhar Sistêmico & Percebe os problemas da empresa e busca o suporte dos líderes? & {\large$\Box$} & {\large$\Box$} & {\large$\Box$} & {\large$\Box$} \\
\cmidrule{2-6}
&  & {\large$\Box$} & {\large$\Box$} & {\large$\Box$} & {\large$\Box$} \\
\midrule
Trabalho em Equipe & Incentiva o diálogo e a troca de opiniões para que todos encontrem juntos a melhor solução? & {\large$\Box$} & {\large$\Box$} & {\large$\Box$} & {\large$\Box$} \\
\cmidrule{2-6}
& É capaz de trabalhar em grupo sem causar conflitos e incentivando a participação de todos? & {\large$\Box$} & {\large$\Box$} & {\large$\Box$} & {\large$\Box$} \\
\midrule
Responsabilidade & Consegue cumprir prazos e busca atingir seus objetivos no trabalho? & {\large$\Box$} & {\large$\Box$} & {\large$\Box$} & {\large$\Box$} \\
\cmidrule{2-6}
& Procura realizar seu trabalho com o nível de qualidade esperado? & {\large$\Box$} & {\large$\Box$} & {\large$\Box$} & {\large$\Box$} \\
\midrule
Comunicação & Passa as informações necessárias para a equipe? & {\large$\Box$} & {\large$\Box$} & {\large$\Box$} & {\large$\Box$} \\
\cmidrule{2-6}
& Comunica-se com lealdade, sem esconder fatos ou omitir informações? & {\large$\Box$} & {\large$\Box$} & {\large$\Box$} & {\large$\Box$} \\
\midrule
Foco em resultados & Motiva o grupo em busca de atingir os resultados esperados? & {\large$\Box$} & {\large$\Box$} & {\large$\Box$} & {\large$\Box$} \\
\cmidrule{2-6}
& Direciona seus esforços para atingir os objetivos da empresa? & {\large$\Box$} & {\large$\Box$} & {\large$\Box$} & {\large$\Box$} \\
\midrule
Organização & É capaz de desempenhar várias tarefas ao mesmo tempo, definindo prioridades e dividindo seu tempo adequadamente? & {\large$\Box$} & {\large$\Box$} & {\large$\Box$} & {\large$\Box$} \\
\cmidrule{2-6}
& Usa seu tempo de forma adequada? & {\large$\Box$} & {\large$\Box$} & {\large$\Box$} & {\large$\Box$} \\
\bottomrule
\caption{Formulário de avaliação de desempenho.}
\centering
\end{longtable}



	\begin{landscape}

\chapter{Relatório de acompanhamento do projeto}
\label{ch:status-report}

\section{Identificação do projeto}

\begin{longtable}{ p{0.2\textwidth} p{1.25\textwidth} }
    \toprule
    \endhead
	\multicolumn{2}{c}{{\textit{Continua na próxima página.}}} \\
	\caption{Identificação do projeto.}
	\endfoot
	\endlastfoot
    \textbf{Projeto:} & \foreign{\{Vaga Livre\}} \\
    \midrule
    \textbf{Data:} & \foreign{\{Data em que o relatório foi desenvolvido\}} \\
    \midrule
    \textbf{Responsável:} & \foreign{\{Nome de quem elaborou o relatório\}} \\
    \bottomrule
    \caption{Identificação do projeto.}
    \centering
\end{longtable}

\section{Marcos do projeto}

\begin{longtable}{ >{\centering\arraybackslash}p{0.65\textwidth} >{\centering\arraybackslash}p{0.25\textwidth} >{\centering\arraybackslash}p{0.25\textwidth} >{\centering\arraybackslash}p{0.25\textwidth} }
    \toprule
    \thead[c]{\textbf{Entrega}} & \thead[c]{\textbf{Data Planejada}} & \thead[c]{\textbf{Data Realizada}} & \thead[c]{\textbf{Status}} \\
    \midrule
    \endhead
	\multicolumn{4}{c}{{\textit{Continua na próxima página.}}} \\
	\caption{Marcos do projeto.}
	\endfoot
	\endlastfoot

    \foreign{\{Descrever atividade ou entrega\}} & \foreign{\{Data prevista\}} & \foreign{\{Data de realização da atividade\}} & \foreign{\{Percentual completado da atividade.\}} \\
    \midrule
    &&&\\
    \midrule
    &&&\\
    \midrule
    &&&\\
    \midrule
    &&&\\

    \caption{Marcos do projeto.}
    \centering
\end{longtable}

\section{Análise desempenho}

O relatório de análise de valor agregado, gerado pelo Microsoft Project (ver apêndice \ref{project-monitoring-report}), deverá ser anexado a este documento.

\section{Principais riscos}

\begin{longtable}{ >{\centering\arraybackslash}p{0.4\textwidth} >{\centering\arraybackslash}p{0.2\textwidth} >{\centering\arraybackslash}p{0.4\textwidth} >{\centering\arraybackslash}p{0.4\textwidth} }
    \toprule
    \thead[c]{\textbf{Risco}} & \thead[c]{\textbf{Exposição}} & \thead[c]{\textbf{Plano de Resposta}} & \thead[c]{\textbf{Comentários}} \\
    \midrule
    \endhead
	\multicolumn{4}{c}{{\textit{Continua na próxima página.}}} \\
	\caption{Marcos do projeto.}
	\endfoot
	\endlastfoot

    \foreign{\{Descrever risco\}} & \foreign{\{Exposição ao risco\}} & \foreign{\{Ação de resposta ao risco\}} & \foreign{\{Comentários sobre o risco\}} \\
    \midrule
    &&&\\
    \midrule
    &&&\\
    \midrule
    &&&\\
    \midrule
    &&&\\

    \caption{Marcos do projeto.}
    \centering
\end{longtable}

\subsection{Avaliação}

\begin{longtable}{ p{0.2\textwidth} p{1.25\textwidth} }
    \toprule
    \endhead
	\multicolumn{2}{c}{{\textit{Continua na próxima página.}}} \\
	\caption{Avaliação do andamento do projeto.}
	\endfoot
	\endlastfoot
    \textbf{Observações:} & \foreign{\{Observações sobre o andamento do projeto\}} \\
    \midrule
    \textbf{Responsável:} & \foreign{\{Nome de quem realizou a avaliação\}} \\
    \midrule
    \textbf{Assinatura:} & \foreign{\{Assinatura de quem realizou a avaliação\}} \\
    \bottomrule
    \caption{Avaliação do andamento do projeto.}
    \centering
\end{longtable}

\section{Anexos opcionais}

Os seguintes documentos poderão ser anexados a este relatório conforme necessidade:

\begin{itemize}
    \item Registro das questões (ver apêndice \ref{ch:question-register}).
    \item Mudanças no nível de engajamento das partes interessadas.
    \item Planos de ação para recuperação de atraso, engajamento das partes interessadas e problemas esperados.
    \item Métricas de qualidade.
\end{itemize}

\end{landscape}

	\chapter{Requisitos de comunicações das partes interessadas}
\label{ch:stakeholder-communication-requirements}

\section{Lista de informações distribuídas}

\subsection{Informação 1}

\begin{description}
\item[Identificador] 1
\item[Tipo] Interna / Externa.
\item[Descrição]
\item[Conteúdo]
\item[Motivo da distribuição]
\item[Idioma]
\item[Canal] Audioconferencia / banco de dados / e-mail / intranet / registro de questões / reunião / videoconferência.
\item[Formato] Ata de reunião / documento / mensagem / relatório / sistema.
\item[Método] Ativa / interativa / passiva.
\item[Periodicidade] 
\item[Recursos alocados]
\item[Formato]
\item[Local]
\end{description}

\section{Matriz de responsabilidades sobre as informações das partes interessadas}

Os responsáveis sobre cada informação terão seus papéis classificados da seguinte forma:

\begin{description}
\item[Auditor (A)] responsável por auditar ou revisar a informação.
\item[Distribuidor (D)] responsável por distribuir a informação conforme estabelecido.
\item[Gerador (G)] responsável por gerar a informação.
\item[Liberador (L)] responsável por liberar ou autorizar a distribuição da informação.
\item[Armazenador (M)] responsável por armazenar a informação.
\item[Receptor (R)] aquele que deve receber a informação.
\end{description}

\begin{longtable}{@{\extracolsep{\fill}} >{\centering\arraybackslash}p{0.05\textwidth} >{\centering\arraybackslash}p{0.05\textwidth} >{\centering\arraybackslash}p{0.05\textwidth} >{\centering\arraybackslash}p{0.05\textwidth} >{\centering\arraybackslash}p{0.05\textwidth} >{\centering\arraybackslash}p{0.05\textwidth} >{\centering\arraybackslash}p{0.05\textwidth} >{\centering\arraybackslash}p{0.05\textwidth} >{\centering\arraybackslash}p{0.05\textwidth} >{\centering\arraybackslash}p{0.05\textwidth} }
    \toprule
	\rot{\textbf{\parbox{6cm}{Identificador da informação}}} & \rot{\textbf{\parbox{6cm}{Parte Interessada 1}}} & \rot{\textbf{\parbox{6cm}{Parte Interessada 2}}} & \rot{\textbf{\parbox{6cm}{Parte Interessada 3}}} & \rot{\textbf{\parbox{6cm}{Parte Interessada 4}}} & \rot{\textbf{\parbox{6cm}{Parte Interessada 5}}} & \rot{\textbf{\parbox{6cm}{Parte Interessada 6}}} & \rot{\textbf{\parbox{6cm}{Parte Interessada 7}}} & \rot{\textbf{\parbox{6cm}{Parte Interessada 8}}} & \rot{\textbf{\parbox{6cm}{Parte Interessada 9}}} \\
	\midrule
	1.1 & A & D & G & L & M & R & A & D & G \\
    \bottomrule
	\caption{Matriz de responsabilidades sobre as informações das partes interessadas.}
	\centering
\end{longtable}

	\chapter{Classificação das partes interessadas}

\section{Grau de poder/interesse}
\label{sec:power-interest-grade}

\begin{figure}[h]
	\begin{tabularx}{\textwidth}{ c | >{\centering\arraybackslash}X  >{\centering\arraybackslash}X l}
		\textbf{Poder} & \cellcolor{blue!10!}\textbf{Manter Satisfeito} & \cellcolor{red!10!}\textbf{Gerenciar de Perto}  &                    \\
		\textbf{Alto} &
		% Poder Alto e Interesse Baixo
		\cellcolor{blue!10!}\begin{tabular}{@{}p{0.35\textwidth}@{}}
			17.Estafácil                             \\
			18.Estacionamento Quality Park            \\
			19.Garagem São Rafael                    \\
			20.Garagem Siqueira Campos                \\
			21.Indigo Center                          \\
			22.Estacionamento Rua da Praia Shopping   \\
			23.Garagem Porto Belo Ltda                \\
			24.Ok Park Estacionamento                 \\
			25.Pare Bem Estacionamentos               \\
			26.Estacionamento Piovezani               \\
			27.Estacionamento AZ Park                 \\
			28.Outros Estacionamentos em Porto Alegre \\
			37.Localiza                               \\
			38.Unidas                                 \\
			39.Alamo                                  \\
			40.Outras locadoras de automóveis        \\
			41.Governo Federal                        \\
			42.Governo Estadual                       \\
			44.Uber                                   \\
			45.Cabify                                 \\
		\end{tabular} &
		% Poder Alto e Interesse Alto
		\cellcolor{red!10!}\begin{tabular}{@{}p{0.35\textwidth}@{}}
			1.\projectSponsorName{}                          \\
			2.\projectManagerName{}                          \\
			3.Equipe do projeto                              \\
			12.Donos de estacionamentos privados             \\
			43.Ministério do Meio Ambiente                  \\
			46.Empresa Pública de Transporte e Circulação \\
			47.Governo do Município de Porto Alegre         \\
		\end{tabular} &   \\
		               & \cellcolor{green!10!}\textbf{Monitorar}        & \cellcolor{orange!10!}\textbf{Manter Informado} &                    \\
		\textbf{Baixo}&
		% Poder Baixo e Interesse Baixo
		\cellcolor{green!10!}\begin{tabular}{@{}p{0.35\textwidth}@{}}
			6.Fórum Permanente de Responsabilidade Social do Rio Grande do Sul \\
			7.Fundação CEPA                                                   \\
			8.Instituto Brasil Ambiental                                        \\
			9.Instituto Urbano Ambiental                                        \\
			10.Outras ONGs relacionadas a preservação do meio ambiente        \\
		\end{tabular} &
		% Poder Baixo e Interesse Alto
		\cellcolor{orange!10!}\begin{tabular}{@{}p{0.35\textwidth}@{}}
			4.Motoristas                                                             \\
			5.Pessoas jurídicas que contratam serviços de estacionamentos privados \\
			11.Funcionários de empresas de estacionamento privado                   \\
		\end{tabular} &    \\
		\hline
		               & \textbf{Baixo}                                 & \textbf{Alto}                                   & \textbf{Interesse} \\
	\end{tabularx}
	\caption{Matriz de classificação das partes interessadas em grau de poder/interesse.}
	\label{tab:power-interest-grade}
	\centering
\end{figure}

Utilizando o registro das partes interessadas foi realizada a classificação das mesmas de acordo com o grau de poder/interesse. O resultado obtido pode ser visto na matriz descrita pela figura \ref{tab:power-interest-grade}.

A classificação segue os seguintes graus de poder/interesse:

\begin{description}
	\item[Poder] \mbox{}
	\begin{description}
		\item[Alto] Poder entre 70\% e 100\%.
		\item[Baixo] Poder entre 30\% e 70\%.
	\end{description}
	\item[Interesse] \mbox{}
	\begin{description}
		\item[Alto] Interesse entre 70\% e 100\%.
		\item[Baixo] Interesse entre 30\% e 70\%.
	\end{description}
\end{description}

%\section{Grau de poder/influência}

%\section{Grau de influência/impacto}

%\section{Modelo de relevância}

\section{Nível de engajamento das partes interessadas}
\label{sec:stakeholder-engagement}

O nível de engajamento atual será documentado utilizando a matriz de avaliação do nível de engajamento das partes interessadas, como mostrado na tabela \ref{tab:stakeholder-engagement-level}), onde A indica o nível de engajamento atual e D indica o nível de engajamento desejado.

O nível de engajamento das partes interessadas será classificado conforme segue:

\begin{description}
	\item[Desinformado] Sem conhecimento do projeto e impactos potenciais.
	\item[Resistente] Ciente do projeto e dos impactos potenciais e resistente ao empreendimento.
	\item[Neutro] Ciente do projeto e mesmo assim não dá apoio ou resistente.
	\item[Dá apoio] Ciente do projeto e dos impactos potenciais e dá apoio ao empreendimento.
	\item[Lidera] Ciente do projeto e dos impactos potenciais e ativamente engajado em garantir o êxito do projeto.
\end{description}

Apenas as partes interessadas que devem ser gerenciadas de perto (conforme classificação realizada na seção \ref{sec:power-interest-grade}) serão analisadas em relação ao nível de engajamento.

\begin{longtable}{ l | >{\centering\arraybackslash}p{0.05\textwidth} >{\centering\arraybackslash}p{0.05\textwidth} >{\centering\arraybackslash}p{0.05\textwidth} >{\centering\arraybackslash}p{0.05\textwidth} >{\centering\arraybackslash}p{0.05\textwidth} }
	\toprule
	\thead[c]{\textbf{Parte interessada}}         & \rot{\textbf{\parbox{4cm}{Não informado}}} & \rot{\textbf{\parbox{4cm}{Resistente}}} & \rot{\textbf{\parbox{4cm}{Neutro}}} & \rot{\textbf{\parbox{4cm}{Apoiador}}} & \rot{\textbf{\parbox{4cm}{Lidera}}} \\
	\midrule
	\endhead
	\multicolumn{6}{c}{{\textit{Continua na próxima página.}}} \\
	\endfoot
	\endlastfoot
	\projectSponsorName{}                         &                                             &                                         &                                     &                                       & D A                                 \\
	\projectManagerName{}                         &                                             &                                         &                                     &                                       & D A                                 \\
	Equipe do projeto                             &                                             &                                         & A                                   &                                       & D                                   \\
	Donos de estacionamentos privados             &                                             &                                         & A                                   & D                                     &                                     \\
	Ministério do Meio Ambiente                  &                                             &                                         & A                                   & D                                     &                                     \\
	Empresa Pública de Transporte e Circulação &                                             &                                         & A                                   & D                                     &                                     \\
	Governo do Município de Porto Alegre         &                                             &                                         & A                                   & D                                     &                                     \\
	\bottomrule
	\caption{Matriz de avaliação do nível de engajamento das partes interessadas.}
	\label{tab:stakeholder-engagement-level}
	\centering
\end{longtable}
	
	\begin{landscape}

\chapter{Registro das questões}
\label{ch:question-register}

A tabela \ref{tab:question-register} deverá ser utilizada para registrar questões e problemas do projeto.

\begin{longtable}{>{\columncolor{gray!10}}>{\centering\arraybackslash}p{0.14\textwidth} >{\centering\arraybackslash}p{0.14\textwidth} >{\columncolor{gray!10}}>{\centering\arraybackslash}p{0.14\textwidth} >{\centering\arraybackslash}p{0.14\textwidth} >{\columncolor{gray!10}}>{\centering\arraybackslash}p{0.14\textwidth} >{\centering\arraybackslash}p{0.14\textwidth} >{\columncolor{gray!10}}>{\centering\arraybackslash}p{0.14\textwidth} >{\centering\arraybackslash}p{0.18\textwidth} >{\columncolor{gray!10}}>{\centering\arraybackslash}p{0.16\textwidth}}
	\toprule
	\textbf{Quem Identificou} & \textbf{Data} & \textbf{Descrição} & \textbf{Tipo} & \textbf{Responsável} & \textbf{Situação} & \textbf{Ação Requerida} & \textbf{Data de Resolução Planejada} & \textbf{Comentários} \\
	\hline
	\endhead
	\multicolumn{5}{c}{{\textit{Continua na próxima página.}}} \\
	\caption{Tabela para registro das questões do projeto.}
	\endfoot
	\endlastfoot
    &&&&&&&&\\
    \hline
    &&&&&&&&\\
    \hline
    &&&&&&&&\\
    \hline
    &&&&&&&&\\
    \hline
    &&&&&&&&\\
    \hline
    &&&&&&&&\\
    \hline
    &&&&&&&&\\
    \hline
    &&&&&&&&\\
    \hline
    &&&&&&&&\\
    \hline
    &&&&&&&&\\
    \hline
    &&&&&&&&\\
    \hline
    &&&&&&&&\\
    \bottomrule
	\caption{Tabela para registro das questões do projeto.}
    \label{tab:question-register}
	\centering
\end{longtable}

\end{landscape}

	\chapter{Formulário de Requisição de Mudança}

\end{apendicesenv}
% ---

% ----------------------------------------------------------
% Attachments
% ----------------------------------------------------------

% ---
% Attachments begin
% ---
%\begin{anexosenv}

	% Print page indicating the attachments session begin
%	\partanexos

	% ---
	%\chapter{Diferença entre Anexo e Apêndice}
	% ---
	%Segundo a NBR 14724 de dezembro de 2005, a diferença primordial entre Anexo e Apêndice é que o Anexo é um texto ou documento não elaborado pelo autor do Trabalho Científico (TC) (monografia, tese, etc.) e o Apêndice é um texto ou documento elaborado pelo autor do TC, ou seja, se foi necessário você criar uma entrevista, um relatório, ou qualquer documento com o escopo de complementar sua argumentação, deve-se utilizar o termo Apêndice e não Anexo.

	%Exemplos:

	%ANEXO A – Documento ou texto não elaborado pelo autor

	%APÊNDICE A – Documento ou texto elaborado pelo autor

%\end{anexosenv}

%---------------------------------------------------------------------
% INDICE REMISSIVO
%---------------------------------------------------------------------
\phantompart
\printindex
%---------------------------------------------------------------------

\end{document}