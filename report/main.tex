\documentclass[
	12pt,				% tamanho da fonte
	openright,			% capítulos começam em pág ímpar (insere página vazia caso preciso)
	twoside,			% para impressão em recto e verso. Oposto a oneside
	a4paper,			% tamanho do papel. 
	% -- opções da classe abntex2 --
	%chapter=TITLE,		% títulos de capítulos convertidos em letras maiúsculas
	%section=TITLE,		% títulos de seções convertidos em letras maiúsculas
	%subsection=TITLE,	% títulos de subseções convertidos em letras maiúsculas
	%subsubsection=TITLE,% títulos de subsubseções convertidos em letras maiúsculas
	% -- opções do pacote babel --
	english,			% idioma adicional para hifenização
    french,             % idioma adicional para hifenização
	spanish,			% idioma adicional para hifenização
	brazil				% o último idioma é o principal do documento
	]{abntex2}

\usepackage{lmodern}
\usepackage[T1]{fontenc}
\usepackage[utf8]{inputenc}
\usepackage{indentfirst}
\usepackage{color}
\usepackage{graphicx}
\usepackage{microtype}

\usepackage[brazilian,hyperpageref]{backref}
\usepackage[alf]{abntex2cite}

\renewcommand{\backrefpagesname}{Citado na(s) página(s):~}
\renewcommand{\backref}{}
\renewcommand*{\backrefalt}[4]{
	\ifcase #1 %
		Nenhuma citação no texto.%
	\or
		Citado na página #2.%
	\else
		Citado #1 vezes nas páginas #2.%
	\fi}%

% ---
% Cover and front pages information
% ---
\titulo{Solução para Gerenciar Eficientemente a Ocupação de Vagas em Estacionamentos Privados}
\autor{Paulo André Haacke}
\local{Brasil}
\data{\today}
\orientador{Cristiano Tonietto Galina}
\coorientador{Nenhum}
\instituicao{%
  Pontifícia Universidade Católica do Rio Grande do Sul -- PUCRS
  \par
  Faculdade de Informática
  \par
  Especialização em Gerenciamento de Projetos com ênfase em Tecnologia da Informação}
\tipotrabalho{Trabalho de Conclusão de Curso (Pós-Graduação)}
% O preambulo deve conter o tipo do trabalho, o objetivo, 
% o nome da instituição e a área de concentração 
\preambulo{Solução para Gerenciar Eficientemente a Ocupação de Vagas em Estacionamentos Privados.}
% ---

% ---
% PDF appearence configurations

\definecolor{blue}{RGB}{41,5,195}

% PDF information
\makeatletter
\hypersetup{
     	%pagebackref=true,
		pdftitle={\@title}, 
		pdfauthor={\@author},
    	pdfsubject={\imprimirpreambulo},
	    pdfcreator={LaTeX with abnTeX2},
        pdfkeywords={project management}{gerenciamento de projetos}{tecnologia da informação}{estacionamentos}{aplicativo}, 
		colorlinks=false,       		% false: boxed links; true: colored links
    	linkcolor=blue,          	% color of internal links
    	citecolor=blue,        		% color of links to bibliography
    	filecolor=magenta,      		% color of file links
		urlcolor=blue,
		bookmarksdepth=4
}
\makeatother
% ---

% --- 
% Line and paragraph space
% --- 

% Paragraph size
\setlength{\parindent}{1.3cm}

% Space between paragraphs
\setlength{\parskip}{0.2cm}  % try also \onelineskip

% ---
% Compile index
% ---
\makeindex
% ---

\begin{document}

% Select document language (according to babel packages)
%\selectlanguage{english}
\selectlanguage{brazil}

% Remove extra space between phrases
\frenchspacing 

% ----------------------------------------------------------
% PRETEXTUAL ELEMENTS
% ----------------------------------------------------------
% \pretextual

% ---
% Cover
% ---
\imprimircapa
% ---

% ---
% Front page
% (The * imply in the existence of a bibliography)
% ---
\imprimirfolhaderosto*
% ---

% ---
% illustrations list
% ---
\pdfbookmark[0]{\listfigurename}{lof}
\listoffigures*
\cleardoublepage
% ---

% ---
% tables list
% ---
\pdfbookmark[0]{\listtablename}{lot}
\listoftables*
\cleardoublepage
% ---

% ---
% Abbreviations and acronyms list
% ---
\begin{siglas}
  \item[ABNT] Associação Brasileira de Normas Técnicas
  \item[abnTeX] ABsurdas Normas para TeX
\end{siglas}
% ---

% ---
% Symbols list
% ---
\begin{simbolos}
  \item[$ \Gamma $] Letra grega Gama
  \item[$ \Lambda $] Lambda
  \item[$ \zeta $] Letra grega minúscula zeta
  \item[$ \in $] Pertence
\end{simbolos}
% ---

% ---
% Summary
% ---
\pdfbookmark[0]{\contentsname}{toc}
\tableofcontents*
\cleardoublepage
% ---

% ----------------------------------------------------------
% TEXTUAL ELEMENTS
% ----------------------------------------------------------
\textual

% ----------------------------------------------------------
% Introdução (exemplo de capítulo sem numeração, mas presente no Sumário)
% ----------------------------------------------------------
\chapter*[Introdução]{Introdução}
\addcontentsline{toc}{chapter}{Introdução}
% ----------------------------------------------------------

Teste

\end{document}