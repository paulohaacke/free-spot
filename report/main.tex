\documentclass[
	12pt,				% tamanho da fonte
	openright,			% capítulos começam em pág ímpar (insere página vazia caso preciso)
	twoside,			% para impressão em recto e verso. Oposto a oneside
	a4paper,			% tamanho do papel. 
	% -- opções da classe abntex2 --
	%chapter=TITLE,		% títulos de capítulos convertidos em letras maiúsculas
	%section=TITLE,		% títulos de seções convertidos em letras maiúsculas
	%subsection=TITLE,	% títulos de subseções convertidos em letras maiúsculas
	%subsubsection=TITLE,% títulos de subsubseções convertidos em letras maiúsculas
	% -- opções do pacote babel --
	english,			% idioma adicional para hifenização
    french,             % idioma adicional para hifenização
	spanish,			% idioma adicional para hifenização
	brazil				% o último idioma é o principal do documento
	]{abntex2}

\usepackage{lmodern}
\usepackage[T1]{fontenc}
\usepackage[utf8]{inputenc}
\usepackage{indentfirst}
\usepackage{color}
\usepackage{graphicx}
\usepackage{microtype}

\usepackage[brazilian,hyperpageref]{backref}
\usepackage[alf]{abntex2cite}

\usepackage{float}
\usepackage[colorinlistoftodos]{todonotes}
\usepackage{styles/vagalivre}

\renewcommand{\backrefpagesname}{Citado na(s) página(s):~}
\renewcommand{\backref}{}
\renewcommand*{\backrefalt}[4]{
	\ifcase #1 %
		Nenhuma citação no texto.%
	\or
		Citado na página #2.%
	\else
		Citado #1 vezes nas páginas #2.%
	\fi}%

% ---
% Cover and front pages information
% ---
\titulo{Vaga Livre: Planejamento do Projeto}
\autor{Paulo André Haacke}
\local{Brasil}
\data{\today}
\orientador{Cristiano Tonietto Galina}
%\coorientador{Nenhum}
\instituicao{%
  Pontifícia Universidade Católica do Rio Grande do Sul -- PUCRS
  \par
  Faculdade de Informática
  \par
  Especialização em Gerenciamento de Projetos com ênfase em Tecnologia da Informação}
\tipotrabalho{Trabalho de Conclusão de Curso (Pós-Graduação)}
% O preambulo deve conter o tipo do trabalho, o objetivo, 
% o nome da instituição e a área de concentração 
\preambulo{Solução para Gerenciar Eficientemente a Ocupação de Vagas em Estacionamentos Privados.}
% ---

% ---
% PDF appearence configurations

\definecolor{blue}{RGB}{41,5,195}

% PDF information
\makeatletter
\hypersetup{
     	%pagebackref=true,
		pdftitle={\@title}, 
		pdfauthor={\@author},
    	pdfsubject={\imprimirpreambulo},
	    pdfcreator={LaTeX with abnTeX2},
        pdfkeywords={project management}{gerenciamento de projetos}{tecnologia da informação}{estacionamentos}{aplicativo}, 
		colorlinks=true,       		% false: boxed links; true: colored links
    	linkcolor=blue,          	% color of internal links
    	citecolor=blue,        		% color of links to bibliography
    	filecolor=magenta,      		% color of file links
		urlcolor=blue,
		bookmarksdepth=4
}
\makeatother
% ---

% --- 
% Line and paragraph space
% --- 

% Paragraph size
\setlength{\parindent}{1.3cm}

% Space between paragraphs
\setlength{\parskip}{0.2cm}  % try also \onelineskip

% ---
% Compile index
% ---
\makeindex
% ---

\begin{document}

% Select document language (according to babel packages)
%\selectlanguage{english}
\selectlanguage{brazil}

% Remove extra space between phrases
\frenchspacing

% ----------------------------------------------------------
% PRETEXTUAL ELEMENTS
% ----------------------------------------------------------
% \pretextual

% ---
% Cover
% ---
\imprimircapa
% ---

% ---
% Front page
% (The * imply in the existence of a bibliography)
% ---
\imprimirfolhaderosto*
% ---

% ---
% Cataloguing data
% ---

% Isto é um exemplo de Ficha Catalográfica, ou ``Dados internacionais de
% catalogação-na-publicação''. Você pode utilizar este modelo como referência. 
% Porém, provavelmente a biblioteca da sua universidade lhe fornecerá um PDF
% com a ficha catalográfica definitiva após a defesa do trabalho. Quando estiver
% com o documento, salve-o como PDF no diretório do seu projeto e substitua todo
% o conteúdo de implementação deste arquivo pelo comando abaixo:
%
% \begin{fichacatalografica}
%     \includepdf{fig_ficha_catalografica.pdf}
% \end{fichacatalografica}

\begin{fichacatalografica}
	\sffamily
	\vspace*{\fill}					% Posição vertical
	\begin{center}					% Minipage Centralizado
		\fbox{\begin{minipage}[c][8cm]{13.5cm}		% Largura
				\small
				\imprimirautor
				%Sobrenome, Nome do autor

				\hspace{0.5cm} \imprimirtitulo  / \imprimirautor. --
				\imprimirlocal, \imprimirdata-

				\hspace{0.5cm} \pageref{LastPage} p. : il. (algumas color.) ; 30 cm.\\

				\hspace{0.5cm} \imprimirorientadorRotulo~\imprimirorientador\\

				\hspace{0.5cm}
				\parbox[t]{\textwidth}{\imprimirtipotrabalho~--~\imprimirinstituicao,
					\imprimirdata.}\\

				\hspace{0.5cm}
				1. Palavra-chave1.
				2. Palavra-chave2.
				2. Palavra-chave3.
				I. Orientador.
				II. Universidade xxx.
				III. Faculdade de xxx.
				IV. Título
			\end{minipage}}
	\end{center}
\end{fichacatalografica}
% ---

% ---
% Approval page
% ---

% Isto é um exemplo de Folha de aprovação, elemento obrigatório da NBR
% 14724/2011 (seção 4.2.1.3). Você pode utilizar este modelo até a aprovação
% do trabalho. Após isso, substitua todo o conteúdo deste arquivo por uma
% imagem da página assinada pela banca com o comando abaixo:
%
% \includepdf{folhadeaprovacao_final.pdf}
%
\begin{folhadeaprovacao}

	\begin{center}
		{\ABNTEXchapterfont\large\imprimirautor}

		\vspace*{\fill}\vspace*{\fill}
		\begin{center}
			\ABNTEXchapterfont\bfseries\Large\imprimirtitulo
		\end{center}
		\vspace*{\fill}

		\hspace{.45\textwidth}
		\begin{minipage}{.5\textwidth}
			\imprimirpreambulo
		\end{minipage}%
		\vspace*{\fill}
	\end{center}

	Trabalho aprovado. \imprimirlocal, 24 de novembro de 2012:

	\assinatura{\textbf{\imprimirorientador} \\ Orientador}
	\assinatura{\textbf{Professor} \\ Convidado 1}
	\assinatura{\textbf{Professor} \\ Convidado 2}
	%\assinatura{\textbf{Professor} \\ Convidado 3}
	%\assinatura{\textbf{Professor} \\ Convidado 4}

	\begin{center}
		\vspace*{0.5cm}
		{\large\imprimirlocal}
		\par
		{\large\imprimirdata}
		\vspace*{1cm}
	\end{center}

\end{folhadeaprovacao}
% ---

% ---
% Dedication
% ---
\begin{dedicatoria}
	\vspace*{\fill}
	\centering
	\noindent
	\textit{ Este trabalho é dedicado às ...} \vspace*{\fill}
\end{dedicatoria}
% ---

% ---
% Acknowledgment
% ---
\begin{agradecimentos}
	Meus agradecimentos.

\end{agradecimentos}
% ---

% ---
% Epigraph
% ---
\begin{epigrafe}
	\vspace*{\fill}
	\begin{flushright}
		\textit{``Frase inspiradora!''}
	\end{flushright}
\end{epigrafe}
% ---

% ---
% ABSTRACTS
% ---

% brazilian portuguese abstract
\setlength{\absparsep}{18pt} % ajusta o espaçamento dos parágrafos do resumo
\begin{resumo}

Ao observar e conversar com donos de estacionamentos privados, assim como com motoristas, percebeu-se dois grandes problemas enfrentados por estes grupos. Os donos e gerentes de estacionamentos privados reportaram a necessidade de aumentar a eficiência e diminuir custos em seus negócios. Os motoristas apontam o tempo gasto com a busca de vagas de garagem e a falta de possibilidade de reservar um vaga como os principais motivos para a insatisfação com o uso de serviços oferecidos por estacionamentos privados. 

Observando-se estes problemas, percebeu-se que eles complementam um ao outro, e para resolver o problema de um é preciso resolver o problema do outro. Este fato levou à percepção de que a criação de um espécie de leilão de vagas para estacionamentos pode ser uma boa oportunidade de negócio. O plano de projeto para implementação desta idéia é o proposto neste trabalho. 

A solução encontrada aborda a implementação de dois aplicativos. Um para ser utilizado pelo motorista, onde é possível realizar reservas e encontrar estacionamentos vagos. O outro aplicativo foca no dono ou gerente do estacionamento privado, e permite verificar a ocupação do estacionamento, seus horários de pico, controlar entrada e saída de clientes, gerar relatórios de tendências, entre outras funcionalidades.

Este plano de projeto foi baseado nos conhecimentos propostos na quinta edição do \cite{project2013guia}. A organização deste trabalho segue conforme os grupos de processos e as áreas de conhecimento para cada documento criado. O plano foca em apresentar e descrever os processos utilizados durante o gerenciamento deste projeto.

 \textbf{Palavras-chave}: vaga-livre, estacionamento privado, plano de projeto, PMBOK.
\end{resumo}

% english abstract
\begin{resumo}[Abstract]
 \begin{otherlanguage*}{english}
   This is the english abstract.
   \todo[inline,color=green]{Criar abstract.}

   \vspace{\onelineskip}
 
   \noindent 
   \textbf{Keywords}: latex. abntex. text editoration.
 \end{otherlanguage*}
\end{resumo}

% ---
% illustrations list
% ---
\pdfbookmark[0]{\listfigurename}{lof}
\listoffigures*
\cleardoublepage
% ---

% ---
% tables list
% ---
\pdfbookmark[0]{\listtablename}{lot}
\listoftables*
\cleardoublepage
% ---

% ---
% Abbreviations and acronyms list
% ---
\begin{siglas}
	\item[ABNT] Associação Brasileira de Normas Técnicas
\end{siglas}
% ---

% ---
% Symbols list
% ---
\begin{simbolos}
	\item[$ \Gamma $] Letra grega Gama
	\item[$ \Lambda $] Lambda
	\item[$ \zeta $] Letra grega minúscula zeta
	\item[$ \in $] Pertence
\end{simbolos}
% ---

% ---
% Summary
% ---
\pdfbookmark[0]{\contentsname}{toc}
\tableofcontents*
\cleardoublepage
% ---

% ----------------------------------------------------------
% TEXTUAL ELEMENTS
% ----------------------------------------------------------
\textual

% ----------------------------------------------------------
% Introdução (exemplo de capítulo sem numeração, mas presente no Sumário)
% ----------------------------------------------------------
\chapter*[Introdução]{Introdução}
\addcontentsline{toc}{chapter}{Introdução}
% ----------------------------------------------------------

Teste

\part{Processos de Iniciação}

\section{Termo de Abertura}

\subsection{Nome do Projeto}

Vaga Livre

\subsection{Patrocinador}

\projectSponsorName

%Poderia colocar a minha empresa ou uma empresa contratante, provisoriamente deixei como sendo minha empresa
%Nome e autoridade do patrocinador ou outra(s) pessoa(s) que autoriza(m) o termo de abertura do projeto.

\subsection{Gerente do Projeto}

\projectManagerName

%Gerente do projeto, responsabilidade, nível de autoridade designados.

\subsection{Descrição}

Habilitar gerentes de estacionamentos privados a controlar a ocupação de vagas em seu estacionamento, incluindo visualizar o histórico de ocupação e demanda atual, oferecer um serviço de reserva com rígido controle.
O consumidor será beneficiado através de um aplicativo móvel, onde poderá realizar reservas e encontrar vagas em estacionamentos diversos.

%Descrição de alto nível do projeto e seus limites.

\subsection{Justificativa}

O mercado de estacionamentos privados tem evoluído muito no Brasil e no Mundo, essa evolução trouxe grandes benefícios para o usuário, entretanto ainda existe demanda para a melhoria nos serviços em áres pouco exploradas, como a busca de vagas e o congestionamento das cidades.
Por outro lado estacionamentos privados encontram dificuldades em alocar de forma eficiente seus estacionamentos.
\todo[inline,color=red]{Melhorar descrição da oportunidade de mercado.}
Através do desenvolvimento deste sistema pretende-se:
\begin{itemize}
	\item Maximizar a ocupação dos estacionamentos;% O que significa isso?
	\item Maximizar o lucro;
	\item Atingir objetivos estratégicos;
	\item Desenvolver estratégias de marketing de acordo com ocupação em períodos anteriores;
	\item Oferecer um melhor serviço aos seus consumidores: permitindo a reserva e busca de vagas.
\end{itemize}

%Finalidade ou justificativa do projeto.

\subsection{Objetivos}

%%Diminuir o tempo médio de espera em fila para 10 minutos. Diminuir o número de vagas não ocupadas nos piores horários para 50.

\begin{itemize}
	\item Desenvolver um aplicativo de celular portável tanto para Android quanto para IOS para encontrar e reservar vagas de estacionamento;
	\item Desenvolver um software que permita aos estacionamento privados controlar a demanda e ocupação de seus estacionamentos;
	\item O aplicativo e o software devem estar prontos até \maximumDeadline;
	\item O aplicativo deve suportar até \minimumUsersAmount usuários;
	\item O orçamento é de até \maximumBudget;
\end{itemize}

%Objetivos mensuráveis do projeto e critérios de sucesso relacionados.

\subsection{Requisitos}

Requisitos de alto nível.

\subsection{Premissas}

Premissas e restrições.

\subsection{Restrições}

Premissas e restrições.

\subsection{Riscos}

Riscos de Alto Nível

\subsection{Marcos}

Resumo do Cronograma de Marcos

\subsection{Orçamento}

Resumo do Orçamento

\subsection{Partes Interessadas}

Lista das Partes Interessadas.

\subsection{Requisitos para aprovação do projeto}

Requisitos para aprovação do projeto (ou seja, o que constitui o sucesso do projeto, quem decide se
o projeto é bem sucedido e quem assina o projeto).

\subsection{Controle de Versão}

\subsection{Aprovações}

\todo[inline,color=red]{Finalizar termo de abertura.}

%\section{Identificar Partes Interessadas}

\subsection{Análise das Partes Interessadas}

Opcional.

\subsection{Registro das Partes Interessadas}

\begin{tabular}{l*{6}{c}}
	Grupo & Nome & Posição/Função & Interesse & Influência & Força/Impacto & Expectativas \\
	\hline
	Cliente & Parte Interessada 1 & Interesse & Função & Influência & Impacto & Expectativas \\
\end{tabular}

\todo[inline,color=red]{Identificar partes interessadas.}

\chapter{Processos de Planejamento}

%---Escopo
%Descrição do processo utilizado para Gerenciamento do Escopo
%Descrição dos processos de coleta de requisitos
%	Descrição de como os requisitos serão coletados
%	Frequência/ número de eventos de coleta/ limite para cessar a coleta
%	Eventos listados na Matriz de Comunicações
%Descrição do processo de validação e controle do escopo do projeto
%	Descrição de como os requisitos serão validados e controlados
%	Frequência das validações e controle
%	Eventos listados na Matriz de Comunicações
%	Remete ao Controle Integrado de Mudanças em caso de divergência planejado x realizado
%---


\chapter{Plano de gerenciamento de escopo}

\section{Objetivo do documento}

O objetivo deste documento é descrever como será gerenciado o escopo, descrevendo quais ferramentas, técnicas e artefatos serão utilizados para determinar o que deve ser abordado durante o projeto \projectName.

\section{Descrição dos processos de gerenciamento de escopo}

\begin{itemize}
	\item O gerenciamento do escopo do projeto será realizado com base em 3 documentos: declaração de escopo para o escopo funcional do projeto, EAP para o escopo das atividades a serem realizadas pelo projeto e dicionário da EAP para descrever os pacotes de trabalho.
	\item Serão consideradas mudanças de escopo apenas as medidas corretivas. Inovações e novas características do produto ou projeto deverão ser tratados de acordo com o plano de gerenciamento da configuração (ver capítulo \ref{ch:configuration-management-plan}).
	\item Todas as mudanças de escopo deverão ser submetidas por escrito ou através de e-mail, conforme descrito no plano de comunicações do projeto.
\end{itemize}

\section{Priorização das mudanças de escopo e respostas}

\todo[inline,color=orange]{Criar níveis de priorização para as mudanças de escopo ou referenciar modelo de priorização integrado de mudanças.}

\section{Gerenciamento de configuração}

\todo[inline,color=orange]{Referenciar fluxo de controle integrado de mudanças.}

\section{Frequência de avaliação do escopo do projeto}

O escopo deve ser avaliado semanalmente dentro da reunião do CCM, prevista no plano de gerenciamento das comunicações (ver capítulo \ref{ch:communication-management-plan}).

\section{Alocação financeira das mudanças de escopo}

As mudanças de escopo podem ser alocadas dentro das reservas gerenciais do projeto de acordo com as necessidades do gerente de projeto.

Para mudanças de escopo prioritárias, em momentos que não existam mais reservas gerenciais disponíveis, deverá ser acionado o patrocinador, já que o gerente de projeto não possui autonomia para decidir utilizar a reserva de contingência de riscos para mudanças de escopo.

\section{Administração do plano de gerenciamento do escopo}

\subsection{Responsável}

\begin{itemize}
	\item \projectManagerName, gerente de projeto, será o responsável direto pelo plano de gerenciamento de escopo.
\end{itemize}
\todo[inline,color=orange]{Verificar necessidade de adicionar suplente responsável.}

\subsection{Frequência de atualização}

O plano de gerenciamento do escopo será reavaliado mensalmente durante a reunião do CCM, juntamente com os outros planos de gerenciamento do projeto.

\section{Outros assuntos relacionados ao gerenciamento do escopo do projeto não previstos neste plano}

As solicitações não previstas neste plano deverão ser submetidas a reunião do CCM para aprovação.
\todo[inline,color=orange]{Adicionar menção ao plano de mudanças}

\section{Controle de Versão}

\begin{table}[H]
	\begin{tabularx}{\textwidth}{| c | c | X | X |}
		\hline
		\textbf{Versão} & \textbf{Data} & \textbf{Autor}      & \textbf{Notas de Revisão} \\
		\hline
		1                &               & \projectManagerName & Criação do documento     \\
		\hline
	\end{tabularx}
	\centering
\end{table}

\section{Aprovações}

\begin{table}[H]
	\begin{tabularx}{\textwidth}{| c | c | X | c |}
		\hline
		\textbf{Função}  & \textbf{Nome}       & \textbf{Assinatura}      & \textbf{Data} \\
		\hline
		Gerente de projeto & \projectManagerName & \projectManagerSignature &               \\
		\hline
	\end{tabularx}
	\centering
\end{table}

%Tempo
%Descrição do processo de definição das atividades (base na decomposição dos pacotes da EAP normalmente)
%Descrição do processo e técnicas de sequenciamento das atividades
%Descrição do processo de estimativa de recursos para as atividades
%Descrição do método de gerenciamento do tempo que será empregado no projeto (PERT/CPM, corrente crítica, SCRUM, ...) e a justificativa para sua aplicação no projeto
%Elementos visuais:
%   Visão executiva do cronograma em formato gráfico (Gantt, Burndown, ...), deve ocupar apenas uma página
%   Principais marcos do projeto na visão do cliente final, patrocinador ou gerência imediata / tem no escopo
%   Cronograma detalhado do projeto com identificação do caminho crítico no formato Gantt contendo: Atividade, Duração, Data de início, Data de fim, Percentual de progresso
%   Lista de atividades do caminho crítico com datas de início e fim
%Apresentação do processo empregado para gerenciamento da linha de base de tempo
%Descrição do processo de controle do cronograma
%   Frequência das ações de controle
%   Eventos listados na Matriz de Comunicações
%   Remete ao Controle Integrado de Mudanças em caso de divergência planejado x realizado

\chapter{Plano de gerenciamento do cronograma}

\todo[inline,color=red]{Criar plano de gerenciamento do cronograma.}

\section{Descrição dos processos de gerenciamento de tempo}

\begin{itemize}

\end{itemize}

\section{Priorização das mudanças nos prazos e respostas}

\section{Sistema de controle de mudanças de prazos}

\section{Mecanismo adotado para conflitos de recursos}

\section{Buffer de tempo do projeto}

\section{Frequência de avaliação dos prazos do projeto}

\section{Alocação financeira para o gerenciamento de tempo}

\section{Administração do plano de gerenciamento de tempo}

\subsection{Responsável}

\begin{itemize}
	\item \projectManagerName, gerente de projeto, será o responsável direto pelo plano de gerenciamento do cronograma.
\end{itemize}
\todo[inline,color=orange]{Verificar necessidade de adicionar suplente responsável.}

\subsection{Frequência de atualização}

O plano de gerenciamento do cronograma será reavaliado mensalmente durante a reunião do CCM, juntamente com os outros planos de gerenciamento do projeto.

\section{Outros assuntos relacionados ao gerenciamento do cronograma do projeto não previstos neste plano}

\section{Controle de Versão}

\begin{table}[H]
	\begin{tabularx}{\textwidth}{| c | c | X | X |}
		\hline
		\textbf{Versão} & \textbf{Data} & \textbf{Autor}      & \textbf{Notas de Revisão} \\
		\hline
		1                &               & \projectManagerName & Criação do documento     \\
		\hline
	\end{tabularx}
	\centering
\end{table}

\section{Aprovações}

\begin{table}[H]
	\begin{tabularx}{\textwidth}{| c | c | X | c |}
		\hline
		\textbf{Função}  & \textbf{Nome}       & \textbf{Assinatura}      & \textbf{Data} \\
		\hline
		Gerente de projeto & \projectManagerName & \projectManagerSignature &               \\
		\hline
	\end{tabularx}
	\centering
\end{table}

% Descrição do método de gerenciamento do custo a ser empregado no projeto incluindo: Sistema de alocação de custos, Relatórios de controle, Periodicidade de atualização
% Método de estimativa dos custos
% Apresentação da composição de valores que totalizam o custo do projeto incluindo: 
%   Custo com pessoal
%   Aquisições
%   Reservas Gerenciais
%   Reservas de Contingência (oriunda dos riscos)
%   Outros
% EVA – Análise de Valor Agregado do Projeto (simulação em momento projetado):
%   Aplicação do método de EVA no projeto
%   Detalhamento dos índices utilizados e análise dos resultados
%   Curvas de desempenho
%   Apresentação de análise comparativa entre o orçamento inicial aprovado do projeto e o custo orçado inicialmente para o projeto: Justificar em caso de diferença 
% Modelo de análise e controle de custos do projeto:
%   Comparação do realizado x linha de base de custos do projeto
%   Ações caso existam diferenças
%   Fluxo de caixa
%   Custo final projetado considerando-se o desempenho atual do projeto
% Apresentação dos custos correntes ou potenciais que não serão apropriados ao projeto
% Descrição do processo de controle dos custos
%   Frequência das ações de controle
%   Eventos listados na Matriz de Comunicações
%   Remete ao Controle Integrado de Mudanças em caso de divergência planejado x realizado

\chapter{Plano de gerenciamento dos custos}

\todo[inline,color=red]{Terminar plano de gerenciamento dos custos.}

\section{Objetivos do documento}

O plano de gerenciamento dos custos descreve como os custos do projeto serão planejados, estruturados e controlados fornecendo detalhes sobre os processos e ferramentas utilizados para gerenciar questões relacionadas a custos.

\section{Descrição dos processos de gerenciamento de custos}

\todo[inline,color=red]{Continuar descrição dos processos de gerenciamento de custos}

\begin{itemize}
	% Estimar custos
	% Ver custo da qualidade (PMBOK 8.1.2.2)
	\item As estimativas de custos serão realizadas utilizando o método \"bottom-up\" com base na opinião especializada.
	\item Custos que possuam incertezas e riscos significantes deverão utilizar a análise de PERT, conforme equação \ref{eq:cost-pert}.
	      \begin{equation}\label{eq:cost-pert}
		      CE = \frac{CO+4CMP+CP}{6}
	      \end{equation}

	      \begin{description}
		      \item[Custo Estimado (DE):] a melhor estimativa do custo necessário para completar a atividade, levando em consideração o fato de que o projeto nem sempre corre conforme o planejado.
		      \item[Custo Mais Provável (MP):] custo da atividade baseado em um esforço de avaliação realista para o trabalho necessário e quaisquer outros gastos previstos.
		      \item[Custo Otimista (O):] custo da atividade baseado na análise do melhor cenário possível para a atividade.
		      \item[Custo Pessimista (P):] custo da atividade baseado na análise do pior cenário para a atividade.
	      \end{description}
	\item A atualização e gerenciamento dos custos do projeto será realizada utilizando o software \projectManagementSoftwareName.
	% Determinar o orçamento
	% Controlar custos
	\item A avaliação dos custos será feito por meio da comparação entre 3 estimativas: valor planejado, valor real e valor agregado.
	\item O controle dos custos será feito tomando como base as estimativas de valor agregado.
	\item O gerente do projeto irá acompanhar a utilização de horas do projeto
\end{itemize}

\section{Frequência de avaliação do orçamento do projeto e das reservas gerenciais}

\section{Reservas gerenciais}

\section{Autonomias}

\section{Alocação financeira das mudanças no orçamento}

\section{Administração do plano de gerenciamento de custos}

\subsection{Responsável pelo plano}

\subsection{Frequência de atualização do plano de gerenciamento de custos}

\section{Outros assuntos relacionados ao gerenciamento de custos do projeto não previstos neste plano}

\section{Controle de Versão}

\begin{table}[H]
	\begin{tabularx}{\textwidth}{| c | c | X | X |}
		\hline
		\textbf{Versão} & \textbf{Data} & \textbf{Autor}      & \textbf{Notas de Revisão} \\
		\hline
		1                &               & \projectManagerName & Criação do documento     \\
		\hline
	\end{tabularx}
	\centering
\end{table}

\section{Aprovações}

\begin{table}[H]
	\begin{tabularx}{\textwidth}{| c | c | X | c |}
		\hline
		\textbf{Função}  & \textbf{Nome}       & \textbf{Assinatura}      & \textbf{Data} \\
		\hline
		Patrocinador       & \projectSponsorName & \projectSponsorSignature &               \\
		\hline
		Gerente de projeto & \projectManagerName & \projectManagerSignature &               \\
		\hline
	\end{tabularx}
	\centering
\end{table}



\chapter{Plano de gerenciamento da qualidade}

\section{Descrição dos processos de gerenciamento da qualidade}

\section{Priorização das mudanças nos requisitos de qualidade e respostas}

\section{Requisitos de qualidade}

\section{Padrões de qualidade}

\section{Sistema de controle de mudanças da qualidade}

\section{Frequência de avaliação dos requisitos de qualidade do projeto}

\section{Alocação financeira das mudanças nos requisitos de qualidade}

\section{Administração do plano de gerenciamento da qualidade}

\subsection{Responsável}

\begin{itemize}
	\item \projectManagerName, gerente de projeto, será o responsável direto pelo plano de gerenciamento da qualidade.
\end{itemize}
\todo[inline,color=orange]{Verificar necessidade de adicionar suplente responsável.}

\subsection{Frequência de atualização}

O plano de gerenciamento da qualidade será reavaliado mensalmente durante a reunião do CCM, juntamente com os outros planos de gerenciamento do projeto.

\section{Outros assuntos relacionados ao gerenciamento da qualidade do projeto não previstos neste plano}

\section{Controle de Versão}

\begin{table}[H]
	\begin{tabularx}{\textwidth}{| c | c | X | X |}
		\hline
		\textbf{Versão} & \textbf{Data} & \textbf{Autor}      & \textbf{Notas de Revisão} \\
		\hline
		1                &               & \projectManagerName & Criação do documento     \\
		\hline
	\end{tabularx}
	\centering
\end{table}

\section{Aprovações}

\begin{table}[H]
	\begin{tabularx}{\textwidth}{| c | c | X | c |}
		\hline
		\textbf{Função}  & \textbf{Nome}       & \textbf{Assinatura}      & \textbf{Data} \\
		\hline
		Gerente de projeto & \projectManagerName & \projectManagerSignature &               \\
		\hline
	\end{tabularx}
	\centering
\end{table}

\todo[inline,color=red]{Criar plano da qualidade.}

% Descrição do tipo de estrutura organizacional da empresa onde o projeto está inserido: Projetizada, matricial forte, funcional, Identificar prós e contras do modelo vigente
% Descrição das necessidades de recursos humanos do projeto: Papéis, Competências, Responsabilidades, Quantidades, Relações hierárquicas
% Organograma do projeto (ideal é ter nome + papel)
% Matriz RACI
% Descrição do plano de mobilização da equipe do projeto: Processo de alocação ou contratação, Datas de alocação e saída do projeto
% Descrição do plano de desenvolvimento de competências da equipe do projeto
% Plano de gerenciamento das pessoas do projeto:
%   Formato de acompanhamento e monitoramento da equipe
%   Frequência do acompanhamento
%   Eventos listados na Matriz de Comunicações
%   Possíveis técnicas de resolução de problemas de acordo com a cultura da empresa
%   Processo de avaliação e feedback
%   Remete ao Controle Integrado de Mudanças em caso de divergência planejado x realizado
% Descrever o modelo de gerenciamento de segurança do projeto: Integridade física, Confidencialidade das informações

\chapter{Plano de gerenciamento dos recursos humanos}

\todo[inline,color=green]{Criar plano de gerenciamento dos recursos humanos.}

\chapter{Plano de gerenciamento das comunicações}
\label{ch:communication-management-plan}

\todo[inline,color=green]{Criar plano de gerenciamento das comunicações.}

\section{Plano de Gerenciamento dos Riscos}

\todo[inline,color=green]{Criar plano de gerenciamento dos riscos.}

% Descrição do processo de aquisições a ser empregado no projeto
% Apresentação da análise de make or buy com justificativas
% Apresentação da Declaração de Trabalho dos itens a serem adquiridos
% Apresentar o fluxo do processo de aquisições
% Apresentação dos modelos de formulários a serem utilizados no processo de aquisição
% Organograma de aquisições
% Processo de identificação e seleção de fornecedores
% Processo de qualificação de propostas
% Identificar os tipos de contratos a serem empregados no projeto
% Apresentar os sistemas preferenciais de garantia
% Apresentação do procedimento de Controle das Aquisições
%   Métodos de controle
%   Frequência das ações de controle
%   Eventos listados na Matriz de Comunicações
%   Remete ao Controle Integrado de Mudanças em caso de divergência planejado x realizado 
% Apresentação do procedimento de Encerramento das Aquisições
 
\chapter{Plano de gerenciamento das aquisições}

\section{Descrição dos processos de gerenciamento das aquisições}

\begin{itemize}
	\item O gerenciamento das aquisições estará focado na contratação de um servidor em nuvem.
	\item 
\end{itemize}

\todo[inline,color=red]{Criar descrição dos processos de aquisição.}

\section{Gerenciamento e tipos de contratos}

\todo[inline,color=red]{Criar gerenciamento e tipos de contratos.}

\section{Critérios de avaliação de cotações e propostas.}

\todo[inline,color=red]{Criar critérios de avaliação de cotações e propostas.}

\section{Avaliação de fornecedores}

\todo[inline,color=red]{Criar método para avaliação de fornecedores.}

\section{Frequência de avaliação dos processos de aquisições}

\todo[inline,color=red]{Descrever frequência de avaliação dos processos de aquisições.}

\section{Alocação financeira para o gerenciamento das aquisições}

\todo[inline,color=red]{Descrever método para aloação financeira do plano de aquisições.}

\section{Administração do plano de gerenciamento das aquisições}

\subsection{Responsável pelo plano}

\begin{itemize}
	\item \projectManagerName{}, gerente de projeto, será o responsável direto pelo plano de gerenciamento das aquisições.
\end{itemize}

\subsection{Frequência de atualização do plano de gerenciamento das aquisições}

O plano de gerenciamento das aquisições será reavaliado semanalmente durante a reunião do CCM, juntamente com os outros planos de gerenciamento do projeto.

\section{Outros assuntos relacionados ao gerenciamento das aquisições do projeto não previstos neste plano}

Solicitações não previstas neste plano deverão passar pela aprovação do CCM. Após aprovada o plano deve ser atualizado pelo gerente do projeto.

\section{Controle de versão}

\begin{table}[H]
	\begin{tabularx}{\textwidth}{| c | c | X | X |}
		\hline
		\textbf{Versão} & \textbf{Data} & \textbf{Autor}        & \textbf{Notas de Revisão} \\
		\hline
		1                &               & \projectManagerName{} & Criação do documento     \\
		\hline
	\end{tabularx}
	\centering
\end{table}

\section{Aprovações}

\begin{table}[H]
	\begin{tabularx}{\textwidth}{| c | c | X | c |}
		\hline
		\textbf{Função}  & \textbf{Nome}         & \textbf{Assinatura}        & \textbf{Data} \\
		\hline
		Patrocinador       & \projectSponsorName{} & \projectSponsorSignature{} &               \\
		\hline
		Gerente de projeto & \projectManagerName{} & \projectManagerSignature{} &               \\
		\hline
	\end{tabularx}
	\centering
\end{table}

\chapter{Plano de gerenciamento das partes interessadas}

\todo[inline,color=green]{Criar plano de gerenciamento das partes interessadas.}

% ----------------------------------------------------------
% Finaliza a parte no bookmark do PDF
% para que se inicie o bookmark na raiz
% e adiciona espaço de parte no Sumário
% ----------------------------------------------------------
\phantompart

\chapter*[Conclusão]{Conclusão}
\addcontentsline{toc}{chapter}{Conclusão}

\todo[inline,color=green]{Criar conclusão.}

% ----------------------------------------------------------
% POSTEXTUAL ELEMENTS
% ----------------------------------------------------------
\postextual
% ----------------------------------------------------------

% ----------------------------------------------------------
% Bibliography
% ----------------------------------------------------------
\bibliography{bibliography}

% ----------------------------------------------------------
% Apêndices
% ----------------------------------------------------------

% ---
% Inicia os apêndices
% ---
\begin{apendicesenv}

	% Imprime uma página indicando o início dos apêndices
	\partapendices

	% ----------------------------------------------------------
	\begin{minipage}[c]{0.95\textwidth}
		\chapter{Lista de Tarefas Pendentes}
		\listoftodos[]
	\end{minipage}
	% ----------------------------------------------------------

	%\chapter{Pesquisa de Mercado}

\begin{enumerate}
\item Nome da empresa para a qual trabalha?
\item Cargo ocupado nesta empresa?
\item Quantas vezes por semana o estacionamento fica lotado?
\item Você sabe quais os dias de maior lotação no estacionamento?
\item Você tem conhecimento do plano estratégico da empresa? Qual seria a missão e visão da empresa?
\item Faz parte dos objetivos estratégicos da empresa a satisfação do cliente?
\item Quais informações você considera mais importantes para atingir a missão e visão da empresa?
\item Você sberia informar se existe a formação de filas? Qual o tempo médio de espera em fila? 
\end{enumerate}

\todo[inline,color=red]{Finalizar pesquisa de mercado}

	%\chapter{Especificação do Trabalho do Projeto}

\section{Necessidade de Negócios}

\section{Descrição do Escopo do Produto}

\section{Plano Estratégico}

\chapter{Proposição de Valor}

\todo[inline,color=red]{Criar proposição de valor}

\chapter{Modelo de Negócio}

\todo[inline,color=red]{Criar modelo do negócio}

\chapter{Plano de Negócios (Business Case)}

\todo[inline,color=red]{Criar plano de negócios}

\section{Sumário Executivo}

\section{Descrição Geral da Companhia}

\section{Produtos e Serviços}

\section{Plano de Marketing}

\section{Plano Operacional}

\section{Organização e Gerenciamento}

\section{Declaração de Finanças Pessoais}

\section{Gastos e Capitalização Iniciais}

\section{Planejamento Financeiro}

\end{apendicesenv}
% ---

% ----------------------------------------------------------
% Anexos
% ----------------------------------------------------------

% ---
% Inicia os anexos
% ---
\begin{anexosenv}

	% Imprime uma página indicando o início dos anexos
	\partanexos

	% ---
	\chapter{Diferença entre Anexo e Apêndice}
	% ---
	Segundo a NBR 14724 de dezembro de 2005, a diferença primordial entre Anexo e Apêndice é que o Anexo é um texto ou documento não elaborado pelo autor do Trabalho Científico (TC) (monografia, tese, etc.) e o Apêndice é um texto ou documento elaborado pelo autor do TC, ou seja, se foi necessário você criar uma entrevista, um relatório, ou qualquer documento com o escopo de complementar sua argumentação, deve-se utilizar o termo Apêndice e não Anexo.

	Exemplos:

	ANEXO A – Documento ou texto não elaborado pelo autor

	APÊNDICE A – Documento ou texto elaborado pelo autor

\end{anexosenv}

%---------------------------------------------------------------------
% INDICE REMISSIVO
%---------------------------------------------------------------------
\phantompart
\printindex
%---------------------------------------------------------------------

\end{document}