\documentclass[
	12pt,				% tamanho da fonte
	openright,			% capítulos começam em pág ímpar (insere página vazia caso preciso)
	twoside,			% para impressão em recto e verso. Oposto a oneside
	a4paper,			% tamanho do papel. 
	% -- opções da classe abntex2 --
	%chapter=TITLE,		% títulos de capítulos convertidos em letras maiúsculas
	%section=TITLE,		% títulos de seções convertidos em letras maiúsculas
	%subsection=TITLE,	% títulos de subseções convertidos em letras maiúsculas
	%subsubsection=TITLE,% títulos de subsubseções convertidos em letras maiúsculas
	% -- opções do pacote babel --
	english,			% idioma adicional para hifenização
    french,             % idioma adicional para hifenização
	spanish,			% idioma adicional para hifenização
	brazil				% o último idioma é o principal do documento
	]{abntex2}

\usepackage{lmodern}
\usepackage[T1]{fontenc}
\usepackage[utf8]{inputenc}
\usepackage{indentfirst}
\usepackage{color}
\usepackage{graphicx}
\usepackage{microtype}

\usepackage[brazilian,hyperpageref]{backref}
\usepackage[alf]{abntex2cite}

\usepackage{todonotes}

\renewcommand{\backrefpagesname}{Citado na(s) página(s):~}
\renewcommand{\backref}{}
\renewcommand*{\backrefalt}[4]{
	\ifcase #1 %
		Nenhuma citação no texto.%
	\or
		Citado na página #2.%
	\else
		Citado #1 vezes nas páginas #2.%
	\fi}%

% ---
% Cover and front pages information
% ---
\titulo{Solução para Gerenciar Eficientemente a Ocupação de Vagas em Estacionamentos Privados}
\autor{Paulo André Haacke}
\local{Brasil}
\data{\today}
\orientador{Cristiano Tonietto Galina}
\coorientador{Nenhum}
\instituicao{%
  Pontifícia Universidade Católica do Rio Grande do Sul -- PUCRS
  \par
  Faculdade de Informática
  \par
  Especialização em Gerenciamento de Projetos com ênfase em Tecnologia da Informação}
\tipotrabalho{Trabalho de Conclusão de Curso (Pós-Graduação)}
% O preambulo deve conter o tipo do trabalho, o objetivo, 
% o nome da instituição e a área de concentração 
\preambulo{Solução para Gerenciar Eficientemente a Ocupação de Vagas em Estacionamentos Privados.}
% ---

% ---
% PDF appearence configurations

\definecolor{blue}{RGB}{41,5,195}

% PDF information
\makeatletter
\hypersetup{
     	%pagebackref=true,
		pdftitle={\@title}, 
		pdfauthor={\@author},
    	pdfsubject={\imprimirpreambulo},
	    pdfcreator={LaTeX with abnTeX2},
        pdfkeywords={project management}{gerenciamento de projetos}{tecnologia da informação}{estacionamentos}{aplicativo}, 
		colorlinks=false,       		% false: boxed links; true: colored links
    	linkcolor=blue,          	% color of internal links
    	citecolor=blue,        		% color of links to bibliography
    	filecolor=magenta,      		% color of file links
		urlcolor=blue,
		bookmarksdepth=4
}
\makeatother
% ---

% --- 
% Line and paragraph space
% --- 

% Paragraph size
\setlength{\parindent}{1.3cm}

% Space between paragraphs
\setlength{\parskip}{0.2cm}  % try also \onelineskip

% ---
% Compile index
% ---
\makeindex
% ---

\begin{document}

% Select document language (according to babel packages)
%\selectlanguage{english}
\selectlanguage{brazil}

% Remove extra space between phrases
\frenchspacing 

% ----------------------------------------------------------
% PRETEXTUAL ELEMENTS
% ----------------------------------------------------------
% \pretextual

% ---
% Cover
% ---
\imprimircapa
% ---

% ---
% Front page
% (The * imply in the existence of a bibliography)
% ---
\imprimirfolhaderosto*
% ---

% ---
% Cataloguing data
% ---

% Isto é um exemplo de Ficha Catalográfica, ou ``Dados internacionais de
% catalogação-na-publicação''. Você pode utilizar este modelo como referência. 
% Porém, provavelmente a biblioteca da sua universidade lhe fornecerá um PDF
% com a ficha catalográfica definitiva após a defesa do trabalho. Quando estiver
% com o documento, salve-o como PDF no diretório do seu projeto e substitua todo
% o conteúdo de implementação deste arquivo pelo comando abaixo:
%
% \begin{fichacatalografica}
%     \includepdf{fig_ficha_catalografica.pdf}
% \end{fichacatalografica}

\begin{fichacatalografica}
	\sffamily
	\vspace*{\fill}					% Posição vertical
	\begin{center}					% Minipage Centralizado
	\fbox{\begin{minipage}[c][8cm]{13.5cm}		% Largura
	\small
	\imprimirautor
	%Sobrenome, Nome do autor
	
	\hspace{0.5cm} \imprimirtitulo  / \imprimirautor. --
	\imprimirlocal, \imprimirdata-
	
	\hspace{0.5cm} \pageref{LastPage} p. : il. (algumas color.) ; 30 cm.\\
	
	\hspace{0.5cm} \imprimirorientadorRotulo~\imprimirorientador\\
	
	\hspace{0.5cm}
	\parbox[t]{\textwidth}{\imprimirtipotrabalho~--~\imprimirinstituicao,
	\imprimirdata.}\\
	
	\hspace{0.5cm}
		1. Palavra-chave1.
		2. Palavra-chave2.
		2. Palavra-chave3.
		I. Orientador.
		II. Universidade xxx.
		III. Faculdade de xxx.
		IV. Título 			
	\end{minipage}}
	\end{center}
\end{fichacatalografica}
% ---

% ---
% Approval page
% ---

% Isto é um exemplo de Folha de aprovação, elemento obrigatório da NBR
% 14724/2011 (seção 4.2.1.3). Você pode utilizar este modelo até a aprovação
% do trabalho. Após isso, substitua todo o conteúdo deste arquivo por uma
% imagem da página assinada pela banca com o comando abaixo:
%
% \includepdf{folhadeaprovacao_final.pdf}
%
\begin{folhadeaprovacao}

  \begin{center}
    {\ABNTEXchapterfont\large\imprimirautor}

    \vspace*{\fill}\vspace*{\fill}
    \begin{center}
      \ABNTEXchapterfont\bfseries\Large\imprimirtitulo
    \end{center}
    \vspace*{\fill}
    
    \hspace{.45\textwidth}
    \begin{minipage}{.5\textwidth}
        \imprimirpreambulo
    \end{minipage}%
    \vspace*{\fill}
   \end{center}
        
   Trabalho aprovado. \imprimirlocal, 24 de novembro de 2012:

   \assinatura{\textbf{\imprimirorientador} \\ Orientador} 
   \assinatura{\textbf{Professor} \\ Convidado 1}
   \assinatura{\textbf{Professor} \\ Convidado 2}
   %\assinatura{\textbf{Professor} \\ Convidado 3}
   %\assinatura{\textbf{Professor} \\ Convidado 4}
      
   \begin{center}
    \vspace*{0.5cm}
    {\large\imprimirlocal}
    \par
    {\large\imprimirdata}
    \vspace*{1cm}
  \end{center}
  
\end{folhadeaprovacao}
% ---

% ---
% Dedication
% ---
\begin{dedicatoria}
   \vspace*{\fill}
   \centering
   \noindent
   \textit{ Este trabalho é dedicado às ...} \vspace*{\fill}
\end{dedicatoria}
% ---

% ---
% Acknowledgment
% ---
\begin{agradecimentos}
Meus agradecimentos.

\end{agradecimentos}
% ---

% ---
% Epigraph
% ---
\begin{epigrafe}
    \vspace*{\fill}
	\begin{flushright}
		\textit{``Frase inspiradora!''}
	\end{flushright}
\end{epigrafe}
% ---

% ---
% ABSTRACTS
% ---

% brazilian portuguese abstract
\setlength{\absparsep}{18pt} % ajusta o espaçamento dos parágrafos do resumo
\begin{resumo}

Ao observar e conversar com donos de estacionamentos privados, assim como com motoristas, percebeu-se dois grandes problemas enfrentados por estes grupos. Os donos e gerentes de estacionamentos privados reportaram a necessidade de aumentar a eficiência e diminuir custos em seus negócios. Os motoristas apontam o tempo gasto com a busca de vagas de garagem e a falta de possibilidade de reservar uma vaga como os principais motivos para a insatisfação com o uso de serviços oferecidos por estacionamentos privados. 

Observando-se estes problemas, percebeu-se que eles complementam um ao outro, e para resolver o problema de um é preciso resolver o problema do outro. Este fato levou à percepção de que a criação de um espécie de leilão de vagas para estacionamentos pode ser uma boa oportunidade de negócio. O plano de projeto para implementação desta idéia é o proposto neste trabalho. 

A solução encontrada aborda a implementação de dois aplicativos. Um para ser utilizado pelo motorista, onde é possível realizar reservas e encontrar estacionamentos vagos. O outro aplicativo foca no dono ou gerente do estacionamento privado, e permite verificar a ocupação do estacionamento, seus horários de pico, controlar entrada e saída de clientes, gerar relatórios de tendências, entre outras funcionalidades.

Este plano de projeto foi baseado nos conhecimentos propostos na quinta edição do \cite{project2013guia}. A organização deste trabalho segue conforme os grupos de processos e as áreas de conhecimento para cada documento criado. O plano foca em apresentar e descrever os processos utilizados durante o gerenciamento deste projeto.

 \textbf{Palavras-chave}: vaga-livre, estacionamento privado, plano de projeto, PMBOK.
\end{resumo}

% english abstract
\begin{resumo}[Abstract]
 \begin{otherlanguage*}{english}
   This is the english abstract.
   \todo[inline,color=green]{Criar abstract.}

   \vspace{\onelineskip}
 
   \noindent 
   \textbf{Keywords}: latex. abntex. text editoration.
 \end{otherlanguage*}
\end{resumo}

% ---
% illustrations list
% ---
\pdfbookmark[0]{\listfigurename}{lof}
\listoffigures*
\cleardoublepage
% ---

% ---
% tables list
% ---
\pdfbookmark[0]{\listtablename}{lot}
\listoftables*
\cleardoublepage
% ---

% ---
% Abbreviations and acronyms list
% ---
\begin{siglas}
  \item[ABNT] Associação Brasileira de Normas Técnicas
\end{siglas}
% ---

% ---
% Symbols list
% ---
\begin{simbolos}
  \item[$ \Gamma $] Letra grega Gama
  \item[$ \Lambda $] Lambda
  \item[$ \zeta $] Letra grega minúscula zeta
  \item[$ \in $] Pertence
\end{simbolos}
% ---

% ---
% Summary
% ---
\pdfbookmark[0]{\contentsname}{toc}
\tableofcontents*
\cleardoublepage
% ---

% ----------------------------------------------------------
% TEXTUAL ELEMENTS
% ----------------------------------------------------------
\textual

% ----------------------------------------------------------
% Introdução (exemplo de capítulo sem numeração, mas presente no Sumário)
% ----------------------------------------------------------
\chapter*[Introdução]{Introdução}
\addcontentsline{toc}{chapter}{Introdução}
% ----------------------------------------------------------

Teste

% ----------------------------------------------------------
% Finaliza a parte no bookmark do PDF
% para que se inicie o bookmark na raiz
% e adiciona espaço de parte no Sumário
% ----------------------------------------------------------
\phantompart

\chapter*[Conclusão]{Conclusão}
\addcontentsline{toc}{chapter}{Conclusão}

\todo[inline,color=green]{Criar conclusão.}

% ----------------------------------------------------------
% POSTEXTUAL ELEMENTS
% ----------------------------------------------------------
\postextual
% ----------------------------------------------------------

% ----------------------------------------------------------
% Bibliography
% ----------------------------------------------------------
\bibliography{bibliography}

\end{document}